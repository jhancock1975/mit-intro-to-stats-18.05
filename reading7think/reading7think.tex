\documentclass[a4paper,11pt]{article}

\usepackage{amsmath}
\usepackage{amssymb}

% for proofs  environment
\usepackage{amsthm}

% for 3d plots
\usepackage{pgfplots}
\usepackage{pgfplotstable}
\usepgfplotslibrary{patch
\usepackage[backend=bibtex]{biblatex}
\bibliography{reading7}

% for probability trees
\usepackage{tikz}
\usetikzlibrary{trees}

% for Venn diagrams
\usetikzlibrary{shapes,backgrounds}
% for plots
\usepackage{ pgfplots}
% inserted on suggestion in warning during compilation
\pgfplotsset{compat=1.9}

%for strikethrough text
\usepackage{soul}

%for R source code listing
\usepackage{listings}

%for block quotes
\usepackage{csquotes}

% For not indenting the first line of paragraphs:
\setlength{\parindent}{0pt}
% define the title
\author{John Hancock}
\title{MIT Introduction to Statistics 18.05 Reading 7 Think Questions}
\begin{document}
% generates the title
\maketitle
% insert the table of contents
\tableofcontents
\section{References and License}
We are answering questions in the material from MIT OpenCourseWare
course 18.05, Introduction to Probability and Statistics.

In this document we are answering questions Orloff and Bloom ask in
\cite{reading7}.

Please see the references section for detailed citation information.

The material for the course is licensed under the terms at
\url{http://ocw.mit.edu/terms}.

We use documentation in \cite{vennDiagram}, \cite{nodePos}, \cite{plotPoints},
\cite{plotFunc} to write \LaTeX source code for this document.
\section{Surface Plot}
In \cite{reading7}, Orloff and Bloom ask us
to sketch a plot of $f\left( x,y \right)
= 4xy$, and visualize the probabilty 
$P\left(A \right)$ as a volume for example
5.
We should mention that example 5 from
\cite{reading7} is the event 
$X < 0.5, Y > 0.5$.

Here is a plot of $f\left( x, y \right)$, 
with the volume of the event $A$ shaded:
\begin{tikzpicture}
\begin{axis}
  \addplot[patch, patch refines=3,
    shader=faceted interp,
    patch type=biquadratic]
    table[z expr=4*x*y] {
      x y
      -1 -1
      1 -1
      1 1
      1 -1
      0 -1
      1 0
      0 1
      -1 0
      0 0
     };
    \end{axis}
  \end{tikzpicture}
\printbibliography{}
\end{document}
