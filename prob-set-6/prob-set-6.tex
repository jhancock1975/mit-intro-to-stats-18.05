\documentclass[a5paper,11pt]{article}

%for coloring cell in a table
\usepackage[table]{xcolor}% http://ctan.org/pkg/xcolor

\usepackage{amsmath}
\usepackage{amssymb}

% for proofs  environment
\usepackage{amsthm}

% for 3d plots
\usepackage{pgfplots}
\usepackage{pgfplotstable}
\usepgfplotslibrary{patchplots}

\usepackage[backend=bibtex]{biblatex}
\bibliography{prob-set-6}

% for probability trees
\usepackage{tikz}
\usetikzlibrary{trees}

% for Venn diagrams
\usetikzlibrary{shapes,backgrounds}

% for plots
\usepackage{ pgfplots}
% inserted on suggestion in warning during compilation
\pgfplotsset{compat=1.9}

%for strikethrough text
\usepackage{soul}

%for R source code listing
\usepackage{listings}

%for block quotes
\usepackage{csquotes}

%for Theorems & Lemmas
\newtheorem{thm}{Theorem}
\newtheorem{lem}[thm]{Lemma}

% For not indenting the first line of paragraphs:
\setlength{\parindent}{0pt}

% define the title
\author{John Hancock}
\title{Problem Set 6}

\begin{document}

% generates the title
\maketitle

% insert the table of contents
\tableofcontents

\section{References and License}
We are answering questions in the material from MIT OpenCourseWare
course 18.05, Introduction to Probability and Statistics.

In this document we are answering questions Orloff and Bloom ask in
\cite{probSet6}.

Please see the references section for detailed citation information.

The material for the course is licensed under the terms at
\url{http://ocw.mit.edu/terms}.

We use documentation in to write the \LaTeX source code for this document.

Note: we find the material in this section of the course extremely difficult
we are relying heavily on the solutions posted to write our answers; 
we are really not coming up with these solutions.

\section{Beta try again}
\subsection{Bayes factor}
We define the hypothesis $H_{0}$ as, "the coin is fair."  Then the likelihood
of $H_{0}$, which is the probability of the data for this problem, given
$H_{0}$ is the ratio of the probability of spinning heads 140 out of 250 times
to the probability of spinning heads with a fair coin in 250 trials:

\begin{equation}
p\left(x \mid H_{0} \right) = 
  \frac{ P \left(x \cap H_{0} \right) }{P\left(H_{0} \right).
\end{equation}

We do not know $P\left(H_{0} \right)$, so we cannot use this definition of
conditional probability.  

The binomial probability for the data, given a fair coin is:

\begin{equation}
\binom{250} {140} \left( \frac{1}{2} \right)^{140} 
  \left( 1 - \frac{1}{2} \right)^{110}
= \binom{250} {140} \left( \frac{1}{2} \right)^{250}. 
\end{equation}

The definition of binomial probability is the same as the conditional 
probability in this case, so we use it in lieu of computing a ratio
involving $P\left(H_{0} \right)$.


\printbibliography{}

\end{document}
