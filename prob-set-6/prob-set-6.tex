\documentclass[a5paper,11pt]{article}

%for coloring cell in a table
\usepackage[table]{xcolor}% http://ctan.org/pkg/xcolor

\usepackage{amsmath}
\usepackage{amssymb}

% for proofs  environment
\usepackage{amsthm}

% for 3d plots
\usepackage{pgfplots}
\usepackage{pgfplotstable}
\usepgfplotslibrary{patchplots}

\usepackage[backend=bibtex]{biblatex}
\bibliography{prob-set-6}

% for probability trees
\usepackage{tikz}
\usetikzlibrary{trees}

% for Venn diagrams
\usetikzlibrary{shapes,backgrounds}

% for plots
\usepackage{ pgfplots}
% inserted on suggestion in warning during compilation
\pgfplotsset{compat=1.9}

%for strikethrough text
\usepackage{soul}

%for R source code listing
\usepackage{listings}

%for block quotes
\usepackage{csquotes}

%for Theorems & Lemmas
\newtheorem{thm}{Theorem}
\newtheorem{lem}[thm]{Lemma}

% For not indenting the first line of paragraphs:
\setlength{\parindent}{0pt}

% define the title
\author{John Hancock}
\title{Problem Set 6}

\begin{document}

% generates the title
\maketitle

% insert the table of contents
\tableofcontents

\section{References and License}
We are answering questions in the material from MIT OpenCourseWare
course 18.05, Introduction to Probability and Statistics.

In this document we are answering questions Orloff and Bloom ask in
\cite{probSet6}.

Please see the references section for detailed citation information.

The material for the course is licensed under the terms at
\url{http://ocw.mit.edu/terms}.

We use documentation in to write the \LaTeX source code for this document.

\section{Beta try again}
\printbibliography{}

\end{document}
