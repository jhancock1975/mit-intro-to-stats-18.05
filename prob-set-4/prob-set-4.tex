\documentclass[a4paper,11pt]{article}


%for coloring cell in a table
\usepackage[table]{xcolor}% http://ctan.org/pkg/xcolor

\usepackage{amsmath}
\usepackage{amssymb}

% for proofs  environment
\usepackage{amsthm}

% for 3d plots
\usepackage{pgfplots}
\usepackage{pgfplotstable}
\usepgfplotslibrary{patchplots}

\usepackage[backend=bibtex]{biblatex}
\bibliography{prob-set-4}

% for probability trees
\usepackage{tikz}
\usetikzlibrary{trees}

% for Venn diagrams
\usetikzlibrary{shapes,backgrounds}

% for plots
\usepackage{ pgfplots}
% inserted on suggestion in warning during compilation
\pgfplotsset{compat=1.9}

%for strikethrough text
\usepackage{soul}

%for R source code listing
\usepackage{listings}

%for block quotes
\usepackage{csquotes}

\newtheorem{thm}{Theorem}
\newtheorem{lem}[thm]{Lemma}

% For not indenting the first line of paragraphs:
\setlength{\parindent}{0pt}
% define the title
\author{John Hancock}
\title{Problem Set 4}
\begin{document}
% generates the title
\maketitle
% insert the table of contents
\tableofcontents
\section{References and License}
We are answering questions in the material from MIT OpenCourseWare
course 18.05, Introduction to Probability and Statistics.

In this document we are answering questions Orloff and Bloom ask in
\cite{slides7}.

Please see the references section for detailed citation information.

The material for the course is licensed under the terms at
\url{http://ocw.mit.edu/terms}.

We use documentation in  \cite{logicNot}, \cite{proofs}, \cite{bars},
\cite{packageClash}, \cite{curlyBrace}, \cite{cases} to write the \LaTeX source code for this document.

\section{Time to failure}
The first group of problems Orloff and Bloom have for us involve some
random variables that follow an exponential distribution.

The exponential distribution they give us to work with has probability
density function (pdf):

\begin{equation}
f\left(x \right) = \lambda e^{-\lambda x}.
\end{equation}


\subsection{$P\left(X \geq x \right)$}

In order to calculate this probability, we will do a change of variable
similar to the technique Orloff and Bloom show in section 3.4 of
\cite{reading7}.

We change the variable in the pdf $f\left(x \right)$ to $u$; therefore we
rewrite the pdf as $f\left( u \right)$:

\begin{equation}
f\left(u \right) = \lamda e^{-\lambda u}.
\end{equation}

We use this definition to write the $P\left(X \geq x \right)$.

\printbibliography{}


\end{document}
