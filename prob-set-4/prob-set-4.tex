\documentclass[a4paper,11pt]{article}


%for coloring cell in a table
\usepackage[table]{xcolor}% http://ctan.org/pkg/xcolor

\usepackage{amsmath}
\usepackage{amssymb}

% for proofs  environment
\usepackage{amsthm}

% for 3d plots
\usepackage{pgfplots}
\usepackage{pgfplotstable}
\usepgfplotslibrary{patchplots}

\usepackage[backend=bibtex]{biblatex}
\bibliography{prob-set-4}

% for probability trees
\usepackage{tikz}
\usetikzlibrary{trees}

% for Venn diagrams
\usetikzlibrary{shapes,backgrounds}

% for plots
\usepackage{ pgfplots}
% inserted on suggestion in warning during compilation
\pgfplotsset{compat=1.9}

%for strikethrough text
\usepackage{soul}

%for R source code listing
\usepackage{listings}

%for block quotes
\usepackage{csquotes}

\newtheorem{thm}{Theorem}
\newtheorem{lem}[thm]{Lemma}

% For not indenting the first line of paragraphs:
\setlength{\parindent}{0pt}
% define the title
\author{John Hancock}
\title{Problem Set 4}
\begin{document}
% generates the title
\maketitle
% insert the table of contents
\tableofcontents
\section{References and License}
We are answering questions in the material from MIT OpenCourseWare
course 18.05, Introduction to Probability and Statistics.

In this document we are answering questions Orloff and Bloom ask in
\cite{slides7}.

Please see the references section for detailed citation information.

The material for the course is licensed under the terms at
\url{http://ocw.mit.edu/terms}.

We use documentation in  \cite{logicNot}, \cite{proofs}, \cite{bars},
\cite{packageClash}, \cite{curlyBrace}, \cite{cases} to write the \LaTeX source code for this document.

\section{Time to failure}
The first group of problems Orloff and Bloom have for us involve some
random variables that follow an exponential distribution.

The exponential distribution they give us to work with has probability
density function (pdf):

\begin{equation}
f\left(x \right) = \lambda e^{-\lambda x}, x \geq 0.
\end{equation}



\subsection{$P\left(X \geq x \right)$}

We know how to calculate $P\left(X < x \right)$ as a definite integral
\cite{reading5b}, therefore we will find 
$P\left( X < x \right)$, and our final result will be to find
$P\left(X \geq x \right) = 1 - P\left( X < x \right)$.

In order to calculate this probability, we will do a change of variable
similar to the technique Orloff and Bloom show in section 3.4 of
\cite{reading7}.

We change the variable in the pdf $f\left(x \right)$ to $u$; therefore we
rewrite the pdf as $f\left( u \right)$:

\begin{equation}
f\left(u \right) = \lambda e^{-\lambda u}.
\end{equation}

We use this definition, the fact that $f$ is defined for $x \geq 0$, and the 
definition of probability of continuous random variables \cite{readingx5b} to 
write this equation:

\begin{equation}
P\left(X < x \right) = \int_0^x \lambda e^{-\lambda u} \,du.
\end{equation}

We substitute the integral on the right hand side of the previous equation 
with its antiderivative to get:

\begin{equation}
P\left(X < x \right) = -e^{-\lambda u} \bigg\rvert_{u=0}^x.
\end{equation}

We evaluate the the antiderivative at the limits of integration:

\begin{equation}
P\left(X < x \right) = -e^{-\lambda x} - -e^{-\lambda 0}.
\end{equation}

Now we simplify the previous equation:

\begin{equation}
P\left(X < x \right) = -e^{-\lambda x} + 1.
\end{equation}

Now, we apply the identity:

\begin{equation}
P\left(X \geq x \right) = 1 - P\left(X < \right).
\end{equation}

Therefore
\begin{equation}
P\left(X \geq x \right) = 1 - \left( -e^{-\lambda x} + 1 \right).
\end{equation}

The previous equation simplifies to:

\begin{equation}
P\left(X \geq x \right) = e^{-\lambda x}.
\end{equation}

subsection{CDF of Minimum of two exponential random variables}

In this section Orloff and Bloom ask us to find the cumulative distribution
function (CDF) of two independent random variables $X_1$, and $X_2$ that both
follow an exponential distribution, and that both have mean 
$\frac{1}{\lambda}$.

In \cite{reading5c} Orloff and Bloom state that the mean of a random
variable that has probability mass function (pmf) $\lambda e^{-\lambda x}$
is $\frac{1}{\lambda}$.

Therefore $X_1$, and $X_2$ both have pmf's $\lambda e^{-\lambda x}$.

For this problem, Orloff and Bloom let $T=\text{min}\left(X_1, X_2 \right)$.

They ask us for the cdf of $T$.

The cdf of $T$ is a function $F\left(t \right) = P \left(T < t \right)$.

In the previous section, we found that for a random varialbe $X$ that has pdf
$\lambda e^{-\lambda x}$, 

\begin{equation}
P\left(X < x \right)  = -e^{-\lambda x} + 1.
\end{equation)

$X_1$, and $X_2$ are in


\printbibliography{}


\end{document}
