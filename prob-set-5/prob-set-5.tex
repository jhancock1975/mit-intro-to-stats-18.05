\documentclass[a5paper,11pt]{article}

%for coloring cell in a table
\usepackage[table]{xcolor}% http://ctan.org/pkg/xcolor

\usepackage{amsmath}
\usepackage{amssymb}

% for proofs  environment
\usepackage{amsthm}

% for 3d plots
\usepackage{pgfplots}
\usepackage{pgfplotstable}
\usepgfplotslibrary{patchplots}

\usepackage[backend=bibtex]{biblatex}
\bibliography{prob-set-5}

% for probability trees
\usepackage{tikz}
\usetikzlibrary{trees}

% for Venn diagrams
\usetikzlibrary{shapes,backgrounds}

% for plots
\usepackage{ pgfplots}
% inserted on suggestion in warning during compilation
\pgfplotsset{compat=1.9}

%for strikethrough text
\usepackage{soul}

%for R source code listing
\usepackage{listings}

%for block quotes
\usepackage{csquotes}

%for Theorems & Lemmas
\newtheorem{thm}{Theorem}
\newtheorem{lem}[thm]{Lemma}

% For not indenting the first line of paragraphs:
\setlength{\parindent}{0pt}

% define the title
\author{John Hancock}
\title{Problem Set 5}

\begin{document}

% generates the title
\maketitle

% insert the table of contents
\tableofcontents

\section{References and License}
We are answering questions in the material from MIT OpenCourseWare
course 18.05, Introduction to Probability and Statistics.

In this document we are answering questions Orloff and Bloom ask in
\cite{probSet5}.

Please see the references section for detailed citation information.

The material for the course is licensed under the terms at
\url{http://ocw.mit.edu/terms}.

We use documentation in to write the \LaTeX source code for this document.

\section{Fit line to data}

In this section we answer questions about a random variable $Y$ drawn from
the random variable $Y_i \sim ax_i + b + \epsilon_i$, where $epslion_i$ is
a random variable with mean $0$ and variance $\sigma^2$. 

Orloff and Bloom grant us that the $\epsilon_i$ are independent.

\subsection{Likelihood function}

We derive the likelihood function $f\left(y_i \mid a,b,x_i, \sigma\right)$.

To derive $f$ we assume $x_i, y_i$, and $\sigma$ are known values. 

It is of paramount importance to note:
\begin{equation}
\epsilon_i \sim N\left(0, \sigma \right).
\end{equation}

We then look at the random variable:
\begin{equation}
Y_i = ax_i + b + \epsilon_i
\end{equation}


$\epsilon_i$ is a random variable that follows a normal distribution.  In
the context of this discussion, it is not a fixed value, its value depends
on what we choose for $a$, and $b$.  Keep in mind that we are trying to find
values for $a$, and $b$ that maximize the likelihood of the linear relationship
between $X$ and $Y$.

So, if $\epsilon_i \sim N\left(0, \sigma^2\right)$, then 

\begin{equation}
ax_i + b + \epsilon_i \sim N\left(ax_i+b, \sigma^2 \right).
\end{equation}

That is, since $\epsilon_i$ is a random variable with mean $0$, then 
the random variable $ax_i+b + \epsilon_i$ will have mean $ax_i+b$.  Orloff
and Bloom show this in \cite{reading6a}.  In this case we are treating 
$ax_i + b$ as constants.  This is really confusing, because we are trying
to find values for $a$ and $b$ that maximize a probability. So we are 
considering varying values of $a$ and $b$ so that we find the best values
for them.  However, assuming we choose values for $a$ and $b$, then 
$ax_i+b+\epsilon_i$ will have mean $ax_i+b$.

In order to make the leap to a probability density function that we are
going to maximize, we cite the reasoning Orloff and Bloom give in 
\cite{reading10b}, section 4.

Then the likelihood function $f_i$ for one point $\left(x_i, y_i\right)$ is:

\begin{equation}
f_i\left(y_i \mid x_i, a, b, \sigma\right)
 = \frac{1}{\sqrt{2\pi}\sigma}
	e^{-\frac{\left(y_i -\left(ax_i+b\right)\right)^2}{2\sigma^2}}.
\end{equation}

The likelihood function $f$ of all points is the product of the function above
for all values of $x_i$, and $y_i$:

\begin{equation}
f
 = \prod_{i=1}^{n} \frac{1}{\sqrt{2\pi}\sigma}
	e^{-\frac{\left(y_i -\left(ax_i+b\right)\right)^2}{2\sigma^2}}.
\end{equation}

We can rewrite the product above as:

\begin{equation}
f\left(y_i \mid x_i, a, b, \sigma\right)
 =  \frac{1}{\sqrt{2\pi}\sigma}
	e^{-\frac{\left(\sum_{i=1}^{n} y_i -\left(ax_i+b\right)\right)^2}{2\sigma^2}}.
\end{equation}

The right hand side of the equation above is the likelihood function.

\subsection{Likelihood and log-likelihood functions for particular values}

We suppose we have the following data:

$\left(1,8\right), \left(3,2\right), \left(5,1\right)$.

We write down the liklihood and log likelihood functions for these
data:

\begin{equation}
f\left(y_i \mid x_i, a, b, \sigma\right)
 =  \frac{1}{\sqrt{2\pi}\sigma}
	e^{-
		\frac{
			\left( 8 -\left(a +b\right)\right)^2
			+ \left( 2 -\left(3a +b\right)\right)^2
			+ \left( 1 -\left(5a +b\right)\right)^2
		  }
		{2\sigma^2}
	  }.
\end{equation}


\begin{equation}
ln \left( f\left(y_i \mid x_i, a, b, \sigma\right) \right)
 =  ln \left( \frac{1}{\sqrt{2\pi}\sigma}
	e^{-
		\frac{
			\left( 8 -\left(a +b\right)\right)^2
			+ \left( 2 -\left(3a +b\right)\right)^2
			+ \left( 1 -\left(5a +b\right)\right)^2
		  }
		{2\sigma^2}
	  } \right).
\end{equation}

We simplify the right hand side of the equation above in several steps:

\begin{equation}
ln \left( f\left(y_i \mid x_i, a, b, \sigma\right) \right)
 =  ln \left( \frac{1}{\sqrt{2\pi}\sigma} \right) + 
	ln \left( e^{-
		\frac{
			\left( 8 -\left(a +b\right)\right)^2
			+ \left( 2 -\left(3a +b\right)\right)^2
			+ \left( 1 -\left(5a +b\right)\right)^2
		  }
		{2\sigma^2}
	  } \right).
\end{equation}


\begin{equation}
ln \left( f\left(y_i \mid x_i, a, b, \sigma\right) \right)
 =  ln \left( \frac{1}{\sqrt{2\pi}\sigma} \right) + 
	ln \left( e^{-
		\frac{
			\left( 8 -\left(a +b\right)\right)^2
			+ \left( 2 -\left(3a +b\right)\right)^2
			+ \left( 1 -\left(5a +b\right)\right)^2
		  }
		{2\sigma^2}
	  } \right).
\end{equation}



\printbibliography{}

\end{document}
