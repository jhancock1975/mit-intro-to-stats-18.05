\documentclass[a4paper,11pt]{article}

\usepackage{amsmath}
\usepackage{amssymb}

% for proofs  environment
\usepackage{amsthm}

% for 3d plots
\usepackage{pgfplots}
\usepackage{pgfplotstable}
\usepgfplotslibrary{patchplots}

\usepackage[backend=bibtex]{biblatex}
\bibliography{slides7}

% for probability trees
\usepackage{tikz}
\usetikzlibrary{trees}

% for Venn diagrams
\usetikzlibrary{shapes,backgrounds}
% for plots
\usepackage{ pgfplots}
% inserted on suggestion in warning during compilation
\pgfplotsset{compat=1.9}

%for strikethrough text
\usepackage{soul}

%for R source code listing
\usepackage{listings}

%for block quotes
\usepackage{csquotes}

% For not indenting the first line of paragraphs:
\setlength{\parindent}{0pt}
% define the title
\author{John Hancock}
\title{MIT Introduction to Statistics 18.05 Class 7 Slides - Solutions}
\begin{document}
% generates the title
\maketitle
% insert the table of contents
\tableofcontents
\section{References and License}
We are answering questions in the material from MIT OpenCourseWare
course 18.05, Introduction to Probability and Statistics.

In this document we are answering questions Orloff and Bloom ask in
\cite{slides7}.

Please see the references section for detailed citation information.

The material for the course is licensed under the terms at
\url{http://ocw.mit.edu/terms}.

We use documentation in  \cite{logicNot}, \cite{proofs} to write \LaTeX source code for this
document.

\section{Estimate Error}
The first question Orloff and Bloom ask in \cite{slides7} is about an
accountant that rounds his cacluations (entries) to the nearest dollar.  We
assume the accountant has made 300 calculations.  Orloff and Bloom want us
to estimate the probability that the total error is greater than five
dollars.

We use the central limit theorem \cite{reading6b} and techniques for estimating
probability that Orloff and Bloom show in \cite{reading6b} in order to find
this estimate.

In order to apply the central limit theorem, we first define a random variable
$X_i$.  $X_i$ is the error the accountant makes on her $i^th$ calculation.
Orloff and Bloom tell us that $X_i$ is uniformly distributed on
$\left[-0.5, 0.5 \right]$.

We also need the mean $\mu$, and standard deviation $\sigma$ of $X_i$ in order
to make our estimate.

In \cite{reading5c} Orloff and Bloom state that a uniformly distributed random
variable on $\left[a, b \right]$ has the distribution function
$f\left(x \right) = \frac{1}{a-b}$.

In \cite{reading6a} Orloff and Bloom define the mean $E\left(X \right)$ of a
continuous random variable $X$ with pdf $f\left(x \right)$ to be:

\begin{equation}
  E\left(X \right) = \int_a^b xf\left(x \right) \,dx
\end{equation}

For this problem, $f\left( x \right) = \frac{1}{-0.5 - 0.5} = -1$.  Therefore
we apply Orloff and Bloom's definition of the mean value of a continuous random
variable to find that the mean value of $X_i$ is

\begin{equation}
  E\left(X_i \right) = \int_{-0.5}^{0.5}-x \,dx.
\end{equation}

We use the power rule for integrals from \cite{basicInt} to find the antiderivative
of the function above that we must integrate in order to find the mean value of
$X_i$.  The antiderivative of $g\left( x \right) = -x$ is $\frac{-x^2}{2}$.

We replace the integral on the right hand side of the equation above with its
antiderivative:

\begin{equation}
  E\left(X_i \right) = \frac{-x^2}{2} \bigg\rvert_{-0.5}^{0.5}.
\end{equation}

And we evaluate the antiderivative over the interval $\left[-0.5, 0.5 \right]$:


\begin{equation}
  E\left(X_i \right) = \frac{-\left(-0.5^2 \right)}{2} - \frac{-\left(0.5^2 \right)}{2}
\end{equation}

Now we do some arithmetic to simplify the right hand side of the equation above:

\begin{equation}
  E\left(X_i \right) = \frac{-1}{8} - \frac{-1}{8} = 0.
\end{equation}

In order to find the standard deviation of $X_i$, we use a property of variance
from \cite{reading6a}, for a continuous random variable $X$:
\begin{equation}
\text{Var}\left(X \right) = E\left(X^2 \right) - E\left(X \right)^2.
\end{equation}

We apply the same reasoning to find $E\left(X_i^2 \right)$ that we use to find
$E\left(X_i \right)$:

\begin{equation}
  E\left(X_i^2 \right) = \int_{-0.5}^{0.5}- \left(x^2 \right) \,dx.
\end{equation}

This implies:

\begin{equation}
  E\left(X_i^2 \right) = \frac{-x^3}{3} \bigg\rvert_{-0.5}^{0.5}.
\end{equation}

Which implies

\begin{equation}
  E\left(X_i^2 \right) = \frac{-\left(-0.5^3 \right)}{3}
  - \frac{-\left(0.5^3 \right)}{3}
\end{equation}

The right hand side of the equation above simplifies to:

\begin{equation}
  E\left(X_i^2 \right) = \frac{-\left(-1 \right)}{24}
  - \frac{-1}{24}
\end{equation}

Therefore the variance of $X_i$ is $\frac{1}{12}$.

Orloff and Bloom ask us to estimate the probability of the size of the error the accountant
makes after 300 calculations.  So, we define a random variable $S$ to be the
sum of 300 values of the $X_i$. Therefore $S$ is the total erorr that the
accountant makes after 300 calculations.

In order use the central limit theorme to estimate the probability that a random
variable is in a range we need to know its mean and standard deviation.

Thereofre we need to know the mean of $S$.  We start with:

\begin{equation}
E\left(S \right) = E\left(\sum_{i=1}^n X_i \right).
\end{equation}

We use a property of expected value from \cite{reading6a} to find that the mean value
\begin{equation}
E\left(S \right) = \sum_{i=1}^300 E\left(X_i \right).
\end{equation}

Above we found that $E\left(X_i \right)=0$.  Therefore, by the equation above,
$E\left(S \right) = 0$.

Now we turn our attention to finding the variance and standard deviaion of $S$.

In \cite{reading6a} Orloff and Bloom state that the variance of the
sum of independent random varialbes is the sum of their variances. We assume the
collection of $X_i$ are independent.

This assumption allows us to write that the variance of $S$,
\begin{equation}
  \text{Var}\left(S) = \sum_{i=1}^300 \frac{1}{12} = 25.
\end{equation}

Becausee standard deviation is the square root of variance, the standard
deviation of $S$, $\sigma_S$ is 5.

Orloff and Bloom are asking us to compute the probability that the total error
the accountant makes after 300 calculations is more than $5\$$.  The total error
the accountant makes might be a positive or negative value, so we need to
estimate the probability that $S < -5\$$ or $S > 5$. However, this probability
is equal to $1 - P\left( -5 \leq S \leq 5 \right)$.  

We standardize $S$, and apply the central limit theorem like Orloff and
Bloom do in \cite{reading6b} to get the approximation:

\begin{equation}
P \left( \frac{S-0}{300 \sqrt{\frac{1}{12}}} > \frac{5-0}{300 \sqrt{\frac{1}{12}}} \right)
  \approx P\left(Z > \frac{5}{300 \sqrt{\frac{1}{12}}} \right)
\end{equation}

We use python to approximate $\frac{5}{300 \sqrt{\frac{1}{12}}} \approx 0.0577$.

Now we use the pnorm function of R to find the probability that $Z > 0.0577$.  The
value of R's pnorm function for the input value $0.0577$ is approximately $0.523$.

Therefore the probability that the total error the account makes after 300 calculations
is $0.523$.

\printbibliography{}
\end{document}
