\documentclass[a4paper,11pt]{article}

\usepackage{amsmath}

\usepackage[backend=bibtex]{biblatex}
\bibliography{biblio}

% for probability trees
\usepackage{tikz}
\usetikzlibrary{trees}

% For not indenting the first line of paragraphs:
\setlength{\parindent}{0pt}
% define the title
\author{John Hancock}
\title{MIT Introduction to Statistics 18.05 Problem Set 1}
\begin{document}
% generates the title
\maketitle
% insert the table of contents
\tableofcontents
\section{References and License}
We are answering questions in the material from MIT opencourseware
course 18.05, Introduction to Probability and Statistics.

Please see the references section for detailed citation information.

The material for the course is licensed under the terms at 
\url{http://ocw.mit.edu/terms}.

\section{Problem 1: Poker Hands}
\subsection{Two-Pair}
We calculate the probability of the poker dealer dealing us a 
hand that is a two-pair hand. First we count the number of two-pair
hands, then we divide the number of two-pair hands by the total
number of hands to calculate the probability of the dealer dealing us
a two pair hand.

The definition of a two-pair hand is, "Two cards have one rank, 
two cards have another rank, and the remaining card has a third rank.
e.g.$\left\{ 2\heartsuit, 2\spadesuit, 2\clubsuit, 5
\clubsuit, K\diamondsuit \right\}$." \cite{probSet1}

We take a combinations approach similar to the approach Orloff and
Bloom take to calculate the probability of a one-pair hand in
\cite{classSlides2}.

First we choose the ranks of the pairs.  There are $13$ ranks, so
there are $\binom{13}{2}$ ways to choose the ranks of the pairs.

Next we choose the suits for the cards in the pairs.  There are
$\binom{4}{2}$ ways to select the suits for the cards in the first
pair, and $\binom{4}{2}$ ways to select the suits for the cards in
the second pair.

To complete the hand we select one card.  We have $11$ ranks to choose
from for the fifth card, and $\binom{4}{1}$ ways to select its suit.

We apply the rule of product to count the number of two-pair hands:

\begin{equation}
\binom{13}{2} \binom{4}{2} \binom{4}{2} 
\binom{11}{1} \binom{4}{1} = 78 \times 6 \times 6 \times 11 \times 4 = 
123552
\end{equation}

The number of all poker hands is the number of ways to select $5$ items
from a set of $52$ items.  Therefore the number of all poker hands is
$\binom{52}{5} = 2598960$.

Therefore the probability of a two-pair hand is $\frac{123552}{2598960}
\approx 0.048$.

\subsection{Three-of-a-Kind}
Orloff and Bloom give the definition of a three-of-a-kind hand as,
"Three cards have one rank and the remaining two cards
have two other ranks. e.g. $\{2\heartsuit, 2\spadesuit, 2\clubsuit,
5\clubsuit, K\diamondsuit\}$." \cite{probSet1}

We use the same approach as above.

First we select the rank for the three cards that have the same rank.

There are $13$ ranks, so there are $\binom{13}{1}$ ways to select this
rank.

Next we select the suits for the three cards that have the same rank.
There are 4 suits, and we choose one for each card, so there are
$\binom{4}{3}$ ways to select the suits for the 3 cards.

We have $\binom{12}{2}$ ways to select the ranks for the fourth 
and fifth cards, and $\binom{4}{1}^{2}$ ways to select their suits.

Now we apply the rule of product to count the number of 
three-of-a-kind hands:

\begin{equation}
\binom{13}{1} \binom{4}{3} \binom{12}{2} \binom{4}{1}^{2} =
13 \times 4 \times 66 \times 16 = 54912
\end{equation}

Therefore the probability of a three-of-a-kind hand is 
$\frac{54912}{2598960} \approx 0.021$.

\section{Problem 2: Non-transitive Dice}
\subsection{Probabilities and Ordering Dice}
\subsubsection{White vs. Green and Green vs. Red}
We follow the method Orloff and Bloom use to calculate the probability
that red beats white\cite{classSlides2}.

We write probability tables for white dice and green dice:

\begin{center}
  \begin{tabular}{ | c | c | c |}
    \hline
    Green Die & & \\ \hline
    outcomes & 1 & 4 \\ \hline
    probability & $\frac{1}{6}$ & $\frac{5}{6}$  \\ \hline
  \end{tabular}
\end{center}

\begin{center}
  \begin{tabular}{ | c | c | c |}
    \hline
    White Die & & \\ \hline
    outcomes & 2 & 5 \\ \hline
    probability & $\frac{1}{2}$ & $\frac{1}{2}$  \\ \hline
  \end{tabular}
\end{center}

Next we write the probability table for the product sample space of
white and green dice:

\begin{center}
  \begin{tabular}{ | c | c | c | c| }
    \hline
     & & Green Die & \\ \hline
    & & 1 & 4 \\ \hline
    White Die & 2 & $\frac{1}{12}$ & $\frac{5}{12}$\\ \hline
     & 5 & $\frac{1}{12}$ & $\frac{5}{12}$ \\ \hline
  \end{tabular}
\end{center}

The pairs in the table above where the outcome for white is greater
than the number for green correspond to outcomes in the product
sample space where white wins.  These are: $\{ white=2, green=1\}$,
$\{white=5, green=1\}$, and $\{white=5, green=4\}$.

We then add the corresponding probabilities for these outcomes where
white wins to to calcuate the probability that white wins:
\begin{equation}
\frac{1}{12} + \frac{1}{12} + \frac{5}{12} = \frac{7}{12} \approx 0.583
\end{equation}

\subsubsection{Ordering the Dice}
When we write one color beats another color, for example, "...red
beats white..." we mean the outcome where the number we roll for the
first color die is greater than the number we roll for the second
color die.
From \cite{classSlides2}, we know that there is a $\frac{7}{12}$
probability that red beats white.
From the previous subsection, we know that there is a $\frac{7}{12}$
probability that white beats green.

Now we are required calculate the probability that green beats red.

We do the probability calculation as we do in the previous section.

We write the probability table for the red die:

\begin{center}
  \begin{tabular}{ | c | c | c |}
    \hline
    Red Die & & \\ \hline
    outcomes & 3 & 6 \\ \hline
    probability & $\frac{5}{6}$ & $\frac{1}{6}$  \\ \hline
  \end{tabular}
\end{center}

For our convenince, we repeat the probability table for the green
die:

\begin{center}
  \begin{tabular}{ | c | c | c |}
    \hline
    Green Die & & \\ \hline
    outcomes & 1 & 4 \\ \hline
    probability & $\frac{1}{6}$ & $\frac{5}{6}$  \\ \hline
  \end{tabular}
\end{center}

Now we write the probability table for the product sample space of 
red and green dice:

\begin{center}
  \begin{tabular}{ | c | c | c | c| }
    \hline
     & & Green Die & \\ \hline
    & & 1 & 4 \\ \hline
    Red Die & 3 & $\frac{5}{36}$ & $\frac{25}{36}$\\ \hline
     & 6 & $\frac{1}{36}$ & $\frac{5}{36}$ \\ \hline
  \end{tabular}
\end{center}

The outcomes where red beats green are $\{red=3,green=1\}$, 
$\{red=6, green=1\}$, and $\{red=6, green=4\}$.  We sum 
the corresponding probabilities in the above table to compute
the probability that green beats red:

\begin{equation}
\frac{5}{36} + \frac{1}{36} + \frac{5}{36} = \frac{11}{36}
\approx 0.306
\end{equation}

Now we are armed with enough information to answer the question 
of whether or not we can order the dice from best to worst.

From \cite{classSlides2} we know the probability that red beats
white is $\frac{7}{12}$.  We have calculated here the probability that
white beats green is $\frac{7}{12}$, and the probability that red
beats green is $1-\frac{11}{36}=\frac{25}{36}$.  Therefore it is more
likely that red will beat white, white will beat green, and green will
beat red.  Therefore we cannot arrange the dice in order from best to
worst.

Note: we had to look at \cite{probSet1Solutions} to get the answer for
the ordering question.

\subsection{Rolling Two Dice}
Note: we used the example in \cite{probTreeHowTo} to render the
probability trees below.

The video at \cite{youTubeDice} 
% Set the overall layout of the tree
\tikzstyle{level 1}=[level distance=3.5cm, sibling distance=3.5cm]
\tikzstyle{level 2}=[level distance=3.5cm, sibling distance=2cm]

% Define styles for bags and leafs
\tikzstyle{bag} = [text width=4em, text centered]
\tikzstyle{end} = [circle, minimum width=3pt,fill, inner sep=0pt]

% The sloped option gives rotated edge labels. Personally
% I find sloped labels a bit difficult to read. Remove the sloped options
% to get horizontal labels. 
\begin{tikzpicture}[grow=right, sloped]
\node[bag] {Bag 1 $4W, 3B$}
    child {
        node[bag] {Bag 2 $4W, 5B$}        
            child {
                node[end, label=right:
                    {$P(W_1\cap W_2)=\frac{4}{7}\cdot\frac{4}{9}$}] {}
                edge from parent
                node[above] {$W$}
                node[below]  {$\frac{4}{9}$}
            }
            child {
                node[end, label=right:
                    {$P(W_1\cap B_2)=\frac{4}{7}\cdot\frac{5}{9}$}] {}
                edge from parent
                node[above] {$B$}
                node[below]  {$\frac{5}{9}$}
            }
            edge from parent 
            node[above] {$W$}
            node[below]  {$\frac{4}{7}$}
    }
    child {
        node[bag] {Bag 2 $3W, 6B$}        
        child {
                node[end, label=right:
                    {$P(B_1\cap W_2)=\frac{3}{7}\cdot\frac{3}{9}$}] {}
                edge from parent
                node[above] {$B$}
                node[below]  {$\frac{3}{9}$}
            }
            child {
                node[end, label=right:
                    {$P(B_1\cap B_2)=\frac{3}{7}\cdot\frac{6}{9}$}] {}
                edge from parent
                node[above] {$W$}
                node[below]  {$\frac{6}{9}$}
            }
        edge from parent         
            node[above] {$B$}
            node[below]  {$\frac{3}{7}$}
    };
\end{tikzpicture}
\printbibliography
\end{document}
