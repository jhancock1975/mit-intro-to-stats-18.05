\documentclass[a4paper,11pt]{article}

\usepackage{amsmath}
\usepackage{amssymb}

\usepackage[backend=bibtex]{biblatex}
\bibliography{biblio}

% for probability trees
\usepackage{tikz}
\usetikzlibrary{trees}

%for strikethrough text
\usepackage{soul}

%for R source code listing
\usepackage{listings}

% For not indenting the first line of paragraphs:
\setlength{\parindent}{0pt}
% define the title
\author{John Hancock}
\title{MIT Introduction to Statistics 18.05 Problem Set 1}
\begin{document}
% generates the title
\maketitle
% insert the table of contents
\tableofcontents
\section{References and License}
We are answering questions in the material from MIT OpenCourseWare
course 18.05, Introduction to Probability and Statistics.

Please see the references section for detailed citation information.

The material for the course is licensed under the terms at 
\url{http://ocw.mit.edu/terms}.

\section{Problem 1: Poker Hands}
\subsection{Two-Pair}
We calculate the probability of the poker dealer dealing us a 
hand that is a two-pair hand. First we count the number of two-pair
hands, then we divide the number of two-pair hands by the total
number of hands to calculate the probability of the dealer dealing us
a two pair hand.

The definition of a two-pair hand is, "Two cards have one rank, 
two cards have another rank, and the remaining card has a third rank.
e.g.$\left\{ 2\heartsuit, 2\spadesuit, 2\clubsuit, 5
\clubsuit, K\diamondsuit \right\}$." \cite{probSet1}

We take a combinations approach similar to the approach Orloff and
Bloom take to calculate the probability of a one-pair hand in
\cite{classSlides2}.

First we choose the ranks of the pairs.  There are $13$ ranks, so
there are $\binom{13}{2}$ ways to choose the ranks of the pairs.

Next we choose the suits for the cards in the pairs.  There are
$\binom{4}{2}$ ways to select the suits for the cards in the first
pair, and $\binom{4}{2}$ ways to select the suits for the cards in
the second pair.

To complete the hand we select one card.  We have $11$ ranks to choose
from for the fifth card, and $\binom{4}{1}$ ways to select its suit.

We apply the rule of product to count the number of two-pair hands:

\begin{equation}
\binom{13}{2} \binom{4}{2} \binom{4}{2} 
\binom{11}{1} \binom{4}{1} = 78 \times 6 \times 6 \times 11 \times 4 = 
123552
\end{equation}

The number of all poker hands is the number of ways to select $5$ items
from a set of $52$ items.  Therefore the number of all poker hands is
$\binom{52}{5} = 2598960$.

Therefore the probability of a two-pair hand is $\frac{123552}{2598960}
\approx 0.048$.

\subsection{Three-of-a-Kind}
Orloff and Bloom give the definition of a three-of-a-kind hand as,
"Three cards have one rank and the remaining two cards
have two other ranks. e.g. $\{2\heartsuit, 2\spadesuit, 2\clubsuit,
5\clubsuit, K\diamondsuit\}$." \cite{probSet1}

We use the same approach as above.

First we select the rank for the three cards that have the same rank.

There are $13$ ranks, so there are $\binom{13}{1}$ ways to select this
rank.

Next we select the suits for the three cards that have the same rank.
There are 4 suits, and we choose one for each card, so there are
$\binom{4}{3}$ ways to select the suits for the 3 cards.

We have $\binom{12}{2}$ ways to select the ranks for the fourth 
and fifth cards, and $\binom{4}{1}^{2}$ ways to select their suits.

Now we apply the rule of product to count the number of 
three-of-a-kind hands:

\begin{equation}
\binom{13}{1} \binom{4}{3} \binom{12}{2} \binom{4}{1}^{2} =
13 \times 4 \times 66 \times 16 = 54912
\end{equation}

Therefore the probability of a three-of-a-kind hand is 
$\frac{54912}{2598960} \approx 0.021$.

\section{Problem 2: Non-transitive Dice}
\subsection{Probabilities and Ordering Dice}
\subsubsection{White vs. Green and Green vs. Red}
We follow the method Orloff and Bloom use to calculate the probability
that red beats white\cite{classSlides2}.

We write probability tables for white dice and green dice:

\begin{center}
  \begin{tabular}{ | c | c | c |}
    \hline
    Green Die & & \\ \hline
    outcomes & 1 & 4 \\ \hline
    probability & $\frac{1}{6}$ & $\frac{5}{6}$  \\ \hline
  \end{tabular}
\end{center}

\begin{center}
  \begin{tabular}{ | c | c | c |}
    \hline
    White Die & & \\ \hline
    outcomes & 2 & 5 \\ \hline
    probability & $\frac{1}{2}$ & $\frac{1}{2}$  \\ \hline
  \end{tabular}
\end{center}

Next we write the probability table for the product sample space of
white and green dice:

\begin{center}
  \begin{tabular}{ | c | c | c | c| }
    \hline
     & & Green Die & \\ \hline
    & & 1 & 4 \\ \hline
    White Die & 2 & $\frac{1}{12}$ & $\frac{5}{12}$\\ \hline
     & 5 & $\frac{1}{12}$ & $\frac{5}{12}$ \\ \hline
  \end{tabular}
\end{center}

The pairs in the table above where the outcome for white is greater
than the number for green correspond to outcomes in the product
sample space where white wins.  These are: $\{ white=2, green=1\}$,
$\{white=5, green=1\}$, and $\{white=5, green=4\}$.

We then add the corresponding probabilities for these outcomes where
white wins to to calculate the probability that white wins:
\begin{equation}
\frac{1}{12} + \frac{1}{12} + \frac{5}{12} = \frac{7}{12} \approx 0.583
\end{equation}

\subsubsection{Ordering the Dice}
When we write one color beats another color, for example, "...red
beats white..." we mean the outcome where the number we roll for the
first color die is greater than the number we roll for the second
color die.
From \cite{classSlides2}, we know that there is a $\frac{7}{12}$
probability that red beats white.
From the previous subsection, we know that there is a $\frac{7}{12}$
probability that white beats green.

Now we are required calculate the probability that green beats red.

We do the probability calculation as we do in the previous section.

We write the probability table for the red die:

\begin{center}
  \begin{tabular}{ | c | c | c |}
    \hline
    Red Die & & \\ \hline
    outcomes & 3 & 6 \\ \hline
    probability & $\frac{5}{6}$ & $\frac{1}{6}$  \\ \hline
  \end{tabular}
\end{center}

For our convenience, we repeat the probability table for the green
die:

\begin{center}
  \begin{tabular}{ | c | c | c |}
    \hline
    Green Die & & \\ \hline
    outcomes & 1 & 4 \\ \hline
    probability & $\frac{1}{6}$ & $\frac{5}{6}$  \\ \hline
  \end{tabular}
\end{center}

Now we write the probability table for the product sample space of 
red and green dice:

\begin{center}
  \begin{tabular}{ | c | c | c | c| }
    \hline
     & & Green Die & \\ \hline
    & & 1 & 4 \\ \hline
    Red Die & 3 & $\frac{5}{36}$ & $\frac{25}{36}$\\ \hline
     & 6 & $\frac{1}{36}$ & $\frac{5}{36}$ \\ \hline
  \end{tabular}
\end{center}

The outcomes where red beats green are $\{red=3,green=1\}$, 
$\{red=6, green=1\}$, and $\{red=6, green=4\}$.  We sum 
the corresponding probabilities in the above table to compute
the probability that green beats red:

\begin{equation}
\frac{5}{36} + \frac{1}{36} + \frac{5}{36} = \frac{11}{36}
\approx 0.306
\end{equation}

Now we are armed with enough information to answer the question 
of whether or not we can order the dice from best to worst.

From \cite{classSlides2} we know the probability that red beats
white is $\frac{7}{12}$.  We have calculated here the probability that
white beats green is $\frac{7}{12}$, and the probability that red
beats green is $1-\frac{11}{36}=\frac{25}{36}$.  Therefore it is more
likely that red will beat white, white will beat green, and green will
beat red.  Therefore we cannot arrange the dice in order from best to
worst.

Note: we had to look at \cite{probSet1Solutions} to get the answer for
the ordering question.

\subsection{Rolling Two Dice}

The authors of the problem set \cite{probSet1} ask us, "Suppose you roll
two white dice against two red dice. What is the probability that the
sum of the white dice is greater than the sum of the red dice?"

The authors also suggest we watch the video \cite{youTubeDice}. The 
material in the video is crucial to our ability to write this
answer.  The text of \cite{probSet1} also hints that we should
follow the methods in \cite{probSet1}, and use a probability tree
to answer this question.

Note: we used the example in \cite{probTreeHowTo} to render the
probability trees below.

We start with rolling the two white dice.  The nodes linked to the root
of the probability tree represent the possible combinations
we can roll using two white dice.  The edges linking the root node
of the probability tree to the nodes representing the possible
combinations we can roll for the two white dice are labeled with the
probability of rolling a combination of dice.  

We use the multiplication rule for independent events to compute the 
probability of rolling a combination of white dice. The white die has
three faces with a value of $5$, and three faces with a value of $2$,
so we have a probability of $\frac{3}{6} = \frac{1}{2}$ for rolling 
either a $2$ or a $5$.  Therefore there is a probability of
$\frac{1}{2}\times\frac{1}{2} = \frac{1}{4}$ of rolling two $5$'s or
two $2$'s.

We use the multiplication and addition rules for independent events
to compute the probability of rolling one $5$ and one $2$: there are
two ways we can roll one $5$ and one $2$, so the probability
of rolling one $5$ and one $2$ is:

\begin{equation}
\frac{1}{2} \times \frac{1}{2} + \frac{1}{2} \times \frac{1}{2}
=\frac{1}{4} + \frac{1}{4} = \frac{1}{2}
\end{equation}

We use similar reasoning for completing the probability tree.

We cross out nodes in the tree where the path through the tree
to the leaf node represents an experiment where the sum of what
we roll for the white dice is not greater than the sum of what we
roll for the red dice.

% Set the overall layout of the tree
\tikzstyle{level 1}=[level distance=3.5cm, sibling distance=3.5cm]
\tikzstyle{level 2}=[level distance=3.5cm, sibling distance=2cm]

% Define styles for bags and leafs
\tikzstyle{bag} = [text width=4em, text centered]
\tikzstyle{end} = [circle, minimum width=3pt,fill, inner sep=0pt]

% The sloped option gives rotated edge labels. Personally
% I find sloped labels a bit difficult to read. Remove the sloped options
% to get horizontal labels. 
\begin{tikzpicture}[grow=right, sloped]
\node[bag] {Start}
    child {
        node[bag] {White Die 1$=5$ White Die 2$=2$}        
           child {
                node[end, label=right:
                    {Red Die 1=$3$, Red Die 2=$3$}] {}
                edge from parent
                node[above]  {$\frac{25}{36}$}
            }
            child {
                node[end, label=right:
                    {\st{Red Die 1=$3$, Red Die 2=$6$}}] {}
                edge from parent
                node[above]  {$\frac{10}{36}$}
            }
            child {
                node[end, label=right:
                    {\st{Red Die 1=$6$, Red Die 2=$6$}}] {}
                edge from parent
                node[above]  {$\frac{1}{36}$}
            }
            edge from parent 
            node[above]  {$\frac{1}{2}$}
    }
    child {
        node[bag] {White Die 1$=5$ White Die 2$=5$}        
        child {
                node[end, label=right:
                    {Red Die 1=$3$, Red Die 2=$3$}] {}
                edge from parent
                node[above]  {$\frac{25}{36}$}
            }
            child {
                node[end, label=right:
                    {Red Die 1=$3$, Red Die 2=$6$}] {}
                edge from parent
                node[above]  {$\frac{10}{36}$}
            }
            child {
                node[end, label=right:
                    {\st{Red Die 1=$6$, Red Die 2=$6$}}] {}
                edge from parent
                node[above]  {$\frac{1}{36}$}
            }
        edge from parent         
            node[above]  {$\frac{1}{4}$}
    }
    child {
        node[bag] {White Die 1$=2$ White Die 2$=2$}        
        child {
                node[end, label=right:
                    {\st{Red Die 1=$3$, Red Die 2=$3$}}] {}
                edge from parent
                node[above]  {$\frac{25}{36}$}
            }
            child {
                node[end, label=right:
                    {\st{Red Die 1=$3$, Red Die 2=$6$}}] {}
                edge from parent
                node[above]  {$\frac{10}{36}$}
            }
            child {
                node[end, label=right:
                    {\st{Red Die 1=$6$, Red Die 2=$6$}}] {}
                edge from parent
                node[above]  {$\frac{1}{36}$}
            }
        edge from parent         
            node[above]  {$\frac{1}{4}$}
    };
\end{tikzpicture}
We use the rule of product for independent probabilities to calculate
the probability of each outcome where the sum of the value we roll 
for the white dice is greater than the sum of the value we roll for
the red dice.

We multiply the probabilities on the edges of the probability tree above
where the edge is on a path to a leaf node where the label on
the leaf node is not crossed out.

Path 1: With the white dice we roll two $5$'s, and with the red dice
we roll one $3$ and one $6$.  The probability for this outcome is
$\frac{1}{4} \times \frac{10}{36} = \frac{10}{144} \approx 0.069$.

Path 2: We roll two $5$'s with the white dice, and we roll two $3$'s
with the red dice.  The probability of this outcome is:
$\frac{1}{4} \times \frac{25}{36} = \frac{25}{144} \approx 0.174$

Path 3: we roll one $5$ and one $2$ with the white dice,
and with the red dice, we two $3$'s.  The probability of this outcome 
is: $\frac{1}{2} \times \frac{25}{36} = \frac{25}{72} \approx 0.347$

We use the addition rule for independent outcomes to compute the
probability for the outcome of path 1, or path 2, or path 3.

The sum of the probabilities is approximately 
$0.069 + 0.347 + 0.174 = 0.59$

\section{Birthdays: counting and simulation}
Note: we did not find this problem to be trivial, so we performed a
Google search on the term, "birthday problem," to get some clues
on how to proceed with a solution.  We find the web pages
\cite{birthProbWiki}, and \cite{betterExpBirth} helpful in writing
this solution.


\subsection{Probability function for $\Omega$}\label{probOmega}
For this problem, the elements $\omega$ of $\Omega$ are sequences of $n$
birthdays, one for each person in the group.  

An example $\omega$ when $n=5$ is $\left( 1, 2, 2, 1, 1) \right)$.

All birthdays are equally likely.

Therefore we can write any sequence of birthdays by sampling the
integers ${1,2,3...365}$ with replacement.


Therefore the probability function
\begin{equation}
P\left ( \Omega \right) = \frac{1}{365^{n}}
\end{equation} 

\subsection{Careful Descriptions of Subsets}
In this problem $\Omega$ is the set
\begin{equation}
   \left\{ \omega \mid \omega = \left(x_{1}, x_{2}, ..., x_{n} \right), 
  x_{i} \in  \left\{ 1, 2, 3, ..., 365  \right\}
  \right\}
\end{equation}
\subsubsection{Someone Shares Our Birthday}

Let $S$ be the set of events where someone shares our birthday $b$,
$b \in \left\{1,2,..., 365\right\}$.

Then we define $S$ as

\begin{equation}
   \left\{ \omega \mid \omega = \left(x_{1}, x_{2}, ..., x_{n} \right), 
  x_{i} \in  \left\{ 1, 2, 3, ..., 365  \right\},
  \forall \omega \left( \exists x_{i} \mid x_{i}=b \right)
  \right\}
\end{equation}

\subsubsection{Two People Share a Birthday}

Let $S$ be the set of sequences of $n$ birthdays where at least two
elements of every sequence are equal.

Then we can express $S$ as

\begin{equation}
  \label{setB}
   \left\{ \omega \mid \omega = \left(x_{1}, x_{2}, ..., x_{n} \right), 
  x_{i} \in  \left\{ 1, 2, 3, ..., 365  \right\},
  \forall \omega \left(\exists x_{i}, x_{j} \mid x_{i}=x_{j} \right)
  \right\}
\end{equation}


Let $S$ be the set of sequences of $n$ birthdays where at least three
elements of every sequence are equal.

Then we can express $S$ as

\begin{equation}
   \left\{ \omega \mid \omega = \left(x_{1}, x_{2}, ..., x_{n} \right), 
  x_{i} \in  \left\{ 1, 2, 3, ..., 365  \right\},
  \forall \omega 
  \left( \exists x_{i}, x_{j}, x_{k} \mid x_{i}=x_{j}=x_{k} \right)
  \right\}
\end{equation}

\subsection{Exact Formula For Someone Sharing Our Birthday}
We want to compute an exact formula for $P\left( A \right)$ where
$A$ is the event that some sequence of birthdays $\omega$ contains
our birthday, and $P\left( A \right)$ is the probability that a
sequence of birthdays $\omega$ contains our birthday $b$.

Once we have the formula, we will calculate the minimum size $n$
of a group of people such that the probability of one of the 
people in the group sharing our birthday is greater than $0.5$.

Note: when we write that a sequence of birthdays contains our birthday
we mean that the sequence of birthdays is a member of the set we
defined in the section titled, "Someone Shares Our Birthday."

There are $365^{n}$ sequences of birthdays of groups of people of size
$n$ or smaller.

It will be easier to calculate $\ = 1 - P\left( A^{\complement} \right)$ .

This is because in order to calculate $P\left(A\right)$ we would have
to calculate the probability that: one person in the group
shares our birthday, or two people in the group share our birthday, or
three people in the group share our birthday, ..., and so on,  up
to the probablility that all the people in the group share our birthday.
Hence we would have to compute the sum of $n$ terrms.

There are $364^{n}$ sequences of birthdays of groups of people of
size $n$ or smaller where none of the people in the group share our
birthday.  \textit{Note}: this group is in the complement $A^{\complement}$
of the set $A$ of groups of people that share our birthday.  


Therefore the probability of a sequence of birthdays not containing
our birthday is $\frac{364^{n}}{365^{n}}$.

This impliess that we have a formula for $P \left( A \right)$:

\begin{equation}
  P\left( A \right) = 1 - \frac{364^{n}}{365^{n}}
\end{equation}

Now we wish to find a value for $n$ such that $P\left( A \right) > 0.5$

Using the formula for $P\left( A \right)$ above we can begin with:

\begin{equation}
  1 - \frac{364^{n}}{365^{n}} > 0.5
\end{equation}

and solve for $n$.

\begin{equation}
  1 - \frac{364^{n}}{365^{n}} > 0.5
\end{equation}

\begin{equation}
\iff
\end{equation}

\begin{equation}
  - \frac{364^{n}}{365^{n}} > \left( 0.5 -1 \right)
\end{equation}


\begin{equation}
\iff
\end{equation}

\begin{equation}
  - \frac{364^{n}}{365^{n}} > - 0.5
\end{equation}


\begin{equation}
\iff
\end{equation}

\begin{equation}
   \frac{364^{n}}{365^{n}} <  0.5
\end{equation}

\begin{equation}
\implies
\end{equation}

\begin{equation}
   log\left( \frac{364^{n}}{365^{n}} \right) <  log \left( 0.5 \right)
\end{equation}

\begin{equation}
\iff
\end{equation}

\begin{equation}
   log\left( 364^{n} \right) - log\left(365^{n} \right) <  
    log \left( 0.5 \right)
\end{equation}

\begin{equation}
\iff
\end{equation}

\begin{equation}
   n log\left( 364 \right) - n log\left(365 \right) <  
    log \left( 0.5 \right)
\end{equation}

\begin{equation}
\iff
\end{equation}

\begin{equation}
   n \left( log\left( 364 \right) -  log\left(365 \right) \right) <  
    log \left( 0.5 \right)
\end{equation}


\begin{equation}
\iff
\end{equation}

\begin{equation}
   n \left( log\left( 364 \right) -  log\left(365 \right) \right) <  
    log \left( 0.5 \right)
\end{equation}

\begin{equation}
\iff
\end{equation}

\begin{equation}
   n <  
    \frac
    {
      log \left( 0.5 \right){}
    }
    {
      \left( log\left( 364 \right) -  log\left(365 \right) \right){}
    }
\end{equation}

\begin{equation}
\iff
\end{equation}

\begin{equation}
   n < 252.652
\end{equation}

Since we are calculating the probability of someone in group of size $n$
having our birthday, we round up.  So, in order for us to have more than
$0.5$ probability that somone in a group of people shares our birthday,
there should be 253 people in the group.

\subsection{Heuristic Explanation of Previous Result}

In the previous section, we find that there should be $253$ people in a
gropu if there is a $0.5$ probability that one of the people in the
group share our birthday.

If there is a sequence of numbers, $\omega$, that is an element of the
compliment of the set we defined in the section entitled, "Someone 
Shares Our Birthday," then the probability that one of the people in
$\omega$ shares our birthday is greater than $0.5$.  However, it is
unlikely that in a group of $\frac{365}{2}$ people, none of the people
would share a birthday, so, we will probably need to have more than
$\frac{365}{2}$ people in the group.

\subsection{\textbf{\textsf{R}} Simulation}

We rely on the material in \cite{stackCheckEltsEq}, \cite{rtut1A}, and
\cite{rtut1B}, \cite{RDummies} to complete this simulation.

In this section, we do an \textbf{\textsf{R}} simulation to estimate the
smallest $n$ for which $P\left( B \right) > 0.9$.

We defin $B$ as in section \ref{setB}.

We define $\Omega$ as in subsection \ref{probOmega}.

Using these definitions, we write an \textbf{\textsf{R}} program that 
does the following steps:to generate elements $\omega$ of
$\Omega$:{}
\begin{enumerate}
  \item for 365 iterations, let $i$ be the current count of iterations
  \begin{enumerate}
    \item for 10 trials, let $t$ be the current trial number
    \begin{enumerate}
      \item for 10,000 iterations:
      \begin{enumerate}
        \item we generate an $\omega$ of length $n$
        \item we check if $i$ of the elements of $\omega$ are the same
        \item if so, we increment the count, $bcount$ of groups that qualify as members of $B$
      \end{enumerate}
      \item print the values of $t$, $i$ and $\frac{bcount}{10,000}$
      \item store the values of $t$, $i$ and $\frac{bcount}{10,000}$ in a list called $results$
    \end{enumerate}
  \end{enumerate}
  \item for each $t$, $i$ in $results$, compute the average $a = \frac{bcount}{10,000}$ 
  \item choose the smallest $i$ where the $a > 0.9$
\end{enumerate}

We will  
\begin{lstlisting}

\end{lstlisting}
\printbibliography
\end{document}
