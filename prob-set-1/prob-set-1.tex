\documentclass[a4paper,11pt]{article}
\usepackage{amsmath}
\usepackage[backend=bibtex]{biblatex}
\bibliography{biblio}
\setlength{\parindent}{0pt}
% define the title
\author{John Hancock}
\title{MIT Introduction to Statistics 18.05 Problem Set 1}
\begin{document}
% generates the title
\maketitle
% insert the table of contents
\tableofcontents
\section{References and License}
We are answering questions in the material from MIT opencourseware
course 18.05, Introduction to Probability and Statistics.

Please see the references section for detailed citation information.

The material for the course is licensed under the terms at 
\url{http://ocw.mit.edu/terms}.

\section{Poker Hands}
\subsection{Two-Pair}
We calculate the probability of the poker dealer dealing us a 
hand that is a two-pair hand. First we count the number of two-pair
hands, then we divide the number of two-pair hands by the total
number of hands to calculate the probability of the dealer dealing us
a two pair hand.

The definition of a two-pair hand is, "Two cards have one rank, 
two cards have another rank, and the remaining card has a third rank.
\textit{e.g.}:$\left\{ 2\heartsuit, 2\spadesuit, 2\clubsuit, 5
\clubsuit, K\diamondsuit \right\}$" \cite{probSet1}

We take a combinations approach similar to the approach Orloff and
Bloom take to calculate the probability of a one-pair hand in
\cite{classSlides2}.

First we choose the ranks of the pairs.  There are $13$ ranks, so
there are $\binom{13}{2}$ ways to choose the ranks of the pairs.

Next we choose the suits for the cards in the pairs.  There are
$\binom{4}{2}$ ways to select the suits for the cards in the first
pair, and $\binom{4}{2}$ ways to select the suits for the cards in
the second pair.

To complete the hand we select one card.  We have $11$ ranks to choose
from for the fifth card, and $\binom{4}{1}$ ways to select its suit.

We apply the rule of product to count the number of two-pair hands:

\begin{equation}
\binom{13}{2} \binom{4}{2} \binom{4}{2} 
\binom{11}{1} \binom{4}{1} = 78 \times 6 \times 6 \times 11 \times 4 = 
123552
\end{equation}

We count the number of all poker hands as the way to select $5$ items
from a set of $52$ items.  Therefore the number of all poker hands is
$\binom{52}{5} = 2598960$.

Therefore the probability of a two-pair hand is $\frac{123552}{2598960}
\approx 0.048$.
\printbibliography
\end{document}
