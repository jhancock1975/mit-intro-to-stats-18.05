
\documentclass[a4paper,11pt]{article}

\usepackage{amsmath}
\usepackage{amssymb}

\usepackage[backend=bibtex]{biblatex}
\bibliography{biblio}

% for probability trees
\usepackage{tikz}
\usetikzlibrary{trees}

%for strikethrough text
\usepackage{soul}

%for R source code listing
\usepackage{listings}

% For not indenting the first line of paragraphs:
\setlength{\parindent}{0pt}
% define the title
\author{John Hancock}
\title{MIT Introduction to Statistics 18.05 Problem Set 1}
\begin{document}
% generates the title
\maketitle
% insert the table of contents
\tableofcontents
\section{Exact Formula For $P\left(B\right)$}
An element of $\omega \in \Omega$ is a sequence of birthdays.  
A birthday is an integer from $1$ through $365$. 
All birthdays are equally likely, and two or more people can be born on the same day,
so any $\omega$ of length $n$ is a sample with replacement from the set
$\left\{ 1, 2, 3, ..., 365 \right\}$ The number of samples of length $k$
when we sample with replacement from a set of size $n$ is $n^k$. 
Therefore the total number of $\omega \in \Omega$ for a given $n$ is $365^{n}$.

Therefore $365^{n}$ is the denomenator we use when we are dividing our count
of events by the total number of events to calculate a probability.

It will be easier to calculate the probability $\bar{p\}$ that for an $\omega$ containing $n$ birthdays, 
none of the birthdays are the same.  Then the probability $p$ that some two birthdays in $\omega$ are the same 
will be $1-\hat{p}$.

It is easier to calculate $1-hat{p}$ because in order to calculate $p$ directly, we have to take into
account that there are $binom{n}{2}$ ways to select two birthdays in $\omega$ to be the same, 
and $binom{n}{3}$ ways to selct three birthdays in $\omega$ to be the same, and so on.

To calculate $1-hat{p}$ we must count the number of ways to select $n$
birthdays from a set of $365$ birthdays where no two birthdays are the
same.

We will use the rule of product for independent events.
\end{document}
