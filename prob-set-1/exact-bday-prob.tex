
\documentclass[a4paper,11pt]{article}

\usepackage{amsmath}
\usepackage{amssymb}

\usepackage[backend=bibtex]{biblatex}
\bibliography{biblio}

% for probability trees
\usepackage{tikz}
\usetikzlibrary{trees}

%for strikethrough text
\usepackage{soul}

%for R source code listing
\usepackage{listings}

% For not indenting the first line of paragraphs:
\setlength{\parindent}{0pt}
% define the title
\author{John Hancock}
\title{MIT Introduction to Statistics 18.05 Problem Set 1}
\begin{document}
% generates the title
\maketitle
% insert the table of contents
\tableofcontents
\section{Exact Formula For $P\left(B\right)}
An element of $\omega \in \Omega$ is a sequence of birthdays.  A birthday is an integer from $1$ through $365$. All birthdays are equally likely, and two or more people can be born on the same day, so any $\omega$ of length $n$ is a sampling with replacement from the set
$\left\{ 1, 2, 3, ..., 365 \right\}$
\end{document}
