\documentclass[a4paper,11pt]{article}

\usepackage{amsmath}
\usepackage{amssymb}

% for proofs  environment
\usepackage{amsthm}

\usepackage[backend=bibtex]{biblatex}
\bibliography{slides4}

% for probability trees
\usepackage{tikz}
\usetikzlibrary{trees}

%for strikethrough text
\usepackage{soul}

%for R source code listing
\usepackage{listings}

%for block quotes
\usepackage{csquotes}

% For not indenting the first line of paragraphs:
\setlength{\parindent}{0pt}
% define the title
\author{John Hancock}
\title{MIT Introduction to Statistics 18.05 Slides 4 - Questions }
\begin{document}
% generates the title
\maketitle
% insert the table of contents
\tableofcontents
\section{References and License}
We are answering questions in the material from MIT OpenCourseWare
course 18.05, Introduction to Probability and Statistics.

Please see the references section for detailed citation information.

The material for the course is licensed under the terms at 
\url{http://ocw.mit.edu/terms}.

We are answering the questions that Orloff and Bloom ask in
\cite{slides4}.

We use documentation in \cite{blockQuote} to write \LaTeX source code
for this document.
 
\section{Conditional Probability of Unknown Die}

The first question Orloff and Bloom give in \cite{slides4} is:
\begin{displayquote}

\begin{enumerate}
  \item The Randomizer holds the 6-sided die in one fist and the 8-sided
  die in the other.

  \item The Roller selects one of the Randomizer’s fists and covertly 
   takes the die.

  \item The Roller rolls the die in secret and reports the result to the
  table.
\end{enumerate}

Given the reported number, what is the probability that the 6-sided die
was chosen? 
\end{displayquote}

We draw a probability tree to get started on a solutionn.  We refer
to the section titled, "Shorthand vs. precise trees," in \cite{reading3}
for guidance on drawing the tree below.

$C$ is the event that the Roller selects the cube shaped $6$-sided die.

$O$ is the event that the Roller selects the octohedron shaped $8$-sided
die.

$C1$, $C2$, \ldots, $C6$ are the events that the Roller rolls one, 
two, or so on to six.  

$O1$, $O2$, \ldots, $O8$ are the events that the Roller rolls one, 
two, or so on to eight.

% Set the overall layout of the tree
\tikzstyle{level 1}=[level distance=1cm, sibling distance=5cm]
\tikzstyle{level 2}=[level distance=2cm, sibling distance=1cm]

% Define styles for bags and leafs
\tikzstyle{bag} = [text width=4em, text centered]
\tikzstyle{end} = [circle, minimum width=3pt,fill, inner sep=0pt]
\begin{center}
\begin{tikzpicture}[grow=down, sloped]
\node[end] {}
    child {
        node[bag] {$C$}        
           child {
                node[end, label=right:
                    {$C1$}] {}
                edge from parent
                node[above]  {$\frac{1}{6}$}
            }
            child {
                node[end, label=right:
                    {$C2$}] {}
                edge from parent
                node[above]  {$\frac{1}{6}$}
            }
            child {
                node[end, label=right:
                    {$\ldots$}] {}
                edge from parent
                node[above]  {}
            }
            child {
                node[end, label=right:
                    {$C6$}] {}
                edge from parent
                node[above]  {$\frac{1}{6}$}
            }
            edge from parent 
            node[above]  {$\frac{1}{2}$}
    }
    child {
        node[bag] {$O$}        
        child {
                node[end, label=right:
                    {$O1$}] {}
                edge from parent
                node[above]  {$\frac{5}{9}$}
            }
            child {
                node[end, label=right:
                    {$O2$}] {}
                edge from parent
                node[above]  {$\frac{4}{9}$}
            }
            child {
                node[end, label=right:
                    {$\ldots$}] {}
                edge from parent
                node[above]  {$\frac{5}{9}$}
            }
            child {
                node[end, label=right:
                    {$O8$}] {}
                edge from parent
                node[above]  {$\frac{1}{2}$}
            }
        edge from parent         
            node[above]  {$\frac{1}{2}$}
    };
\end{tikzpicture}
\end{center}
\printbibliography{}
\end{document}
