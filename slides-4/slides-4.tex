\documentclass[a4paper,11pt]{article}

\usepackage{amsmath}
\usepackage{amssymb}

% for proofs  environment
\usepackage{amsthm}

\usepackage[backend=bibtex]{biblatex}
\bibliography{slides4}

% for probability trees
\usepackage{tikz}
\usetikzlibrary{trees}

%for strikethrough text
\usepackage{soul}

%for R source code listing
\usepackage{listings}

%for block quotes
\usepackage{csquotes}

% For not indenting the first line of paragraphs:
\setlength{\parindent}{0pt}
% define the title
\author{John Hancock}
\title{MIT Introduction to Statistics 18.05 Slides 4 - Questions }
\begin{document}
% generates the title
\maketitle
% insert the table of contents
\tableofcontents
\section{References and License}
We are answering questions in the material from MIT OpenCourseWare
course 18.05, Introduction to Probability and Statistics.

Please see the references section for detailed citation information.

The material for the course is licensed under the terms at 
\url{http://ocw.mit.edu/terms}.

We are answering the questions that Orloff and Bloom ask in
\cite{slides4}.

We use documentation in \cite{blockQuote} to write \LaTeX source code
for this document.
 
\section{Conditional Probability of Unknown Die}

The first question Orloff and Bloom give in \cite{slides4} is:
\begin{displayquote}

\begin{enumerate}
  \item The Randomizer holds the 6-sided die in one fist and the 8-sided
  die in the other.

  \item The Roller selects one of the Randomizer’s fists and covertly 
   takes the die.

  \item The Roller rolls the die in secret and reports the result to the
  table.
\end{enumerate}

Given the reported number, what is the probability that the 6-sided die
was chosen? 
\end{displayquote}

Note: we needed to see the solution in \cite{slides4Ans} in order to
write the answer to this question.

We have two cases.

The first case is the Roller reports a 7 or an 8. Then the 
probability that the 6-sided die was chosen is 0.

The second case is the Roller reports a number with a value from
1 to six.

We draw a probability tree to get started on a solutionn.  We refer
to the section titled, "Shorthand vs. precise trees," in \cite{reading3}
for guidance on drawing the tree below.

$C$ is the event that the Roller selects the cube shaped $6$-sided die.

$O$ is the event that the Roller selects the octohedron shaped $8$-sided
die.

$R1$, $R2$, \ldots, $R6$ are the events that the Roller reports one, 
two, or so on to six.  

% Set the overall layout of the tree
\tikzstyle{level 1}=[level distance=1cm, sibling distance=5cm]
\tikzstyle{level 2}=[level distance=2cm, sibling distance=1cm]

% Define styles for bags and leafs
\tikzstyle{bag} = [text width=4em, text centered]
\tikzstyle{end} = [circle, minimum width=3pt,fill, inner sep=0pt]

\begin{center}
\begin{tikzpicture}[grow=down, sloped]
\node[end] {}
    child {
        node[bag] {$C$}        
           child {
                node[end, label=right:
                    {$R_{1}$}] {}
                edge from parent
                node[above]  {$\frac{1}{6}$}
            }
            child {
                node[end, label=right:
                    {$R_{2}$}] {}
                edge from parent
                node[above]  {$\frac{1}{6}$}
            }
            child {
                node[end, label=right:
                    {$\ldots$}] {}
                edge from parent
                node[above]  {}
            }
            child {
                node[end, label=right:
                    {$R_{6}$}] {}
                edge from parent
                node[above]  {$\frac{1}{6}$}
            }
            edge from parent 
            node[above]  {$\frac{1}{2}$}
    }
    child {
        node[bag] {$O$}        
        child {
                node[end, label=right:
                    {$R_{1}$}] {}
                edge from parent
                node[above]  {$\frac{1}{8}$}
            }
            child {
                node[end, label=right:
                    {$R_{2}$}] {}
                edge from parent
                node[above]  {$\frac{1}{8}$}
            }
            child {
                node[end, label=right:
                    {$\ldots$}] {}
                edge from parent
                node[above]  {$\frac{1}{8}$}
            }
            child {
                node[end, label=right:
                    {$R_{6}$}] {}
                edge from parent
                node[above]  {$\frac{1}{8}$}
            }
        edge from parent         
            node[above]  {$\frac{1}{2}$}
    };
\end{tikzpicture}
\end{center}

Since the probabilities on all edges in the tree connected to $C$ are
$\frac{1}{6}$, and the probabilities on all edges in the tree connected
to $O$ are $\frac{1}{8}$, we can calculate 
$P \left( C \mid R_{1} \right)$, and the result will be the same for
any of the other leaf nodes in the tree above.  This is because the
calculation will involve the same numbers $\frac{1}{2}$, $\frac{1}{6}$,
and $\frac{1}{8}$, and the same operations on these numbers.

Using the tree above, we can calculate $P \left( R_{1} \mid C \right)$.

Now, we use Bayes' theorem in \cite{reading3} to calculate
$P \left( C \mid R_{1} \right)$

\begin{equation} \label{bayesForCube}
P \left( C \mid R_{1} \right) = 
  \frac{ P \left( R_{1} \mid C \right) P \left( C \right)}
    { P \left( R_{1} \right)}
\end{equation}

Now, we apply definitions for values on various parts of probability
trees using the section titled "Shorthand vs. precise trees," in 
\cite{reading3} to obtain values for the numerator and denominator on
the righthand side of \ref{bayesForCube}.

From the probability tree, 
\begin{equation}
P \left( R_{1} \mid C \right) = 
   \frac{1}{6}
\end{equation}

$P \left( C \right) = \frac{1}{2}$.  Note: we are assuming the Roller
uses either die with equal probability.

We apply Bayes rule \cite{reading3} and the Law of Total Probability
\cite{reading3} to compute $P \left( R_{1} \right)$.

\begin{equation}
  P \left( R_{1} \right)
    = P \left( R_{1} \cap C \right) + P \left( R_{1} \cap O \right)
    = P \left( R_{1} \mid C \right) P \left( C \right)
      + P \left( R_{1} \mid O \right) P \left( O \right)
    = \left( \frac{1}{6} \right)  \left( \frac{1}{2} \right)
      + \left( \frac{1}{8} \right)  \left( \frac{1}{2} \right)
\end{equation}

Now we have values for the numerator and denominator of the right
hand side of \ref{bayesForCube}.

\begin{equation}
P \left( C \mid R_{1} \right) = 
  \frac{ 
    \left( \frac{1}{6} \right) \left( \frac{1}{2} \right) 
    }
    {
      \left( \frac{1}{6} \right) \left( \frac{1}{2} \right) + 
        \left( \frac{1}{8} \right) \left( \frac{1}{2} \right) 
    }
\end{equation}


\printbibliography{}
\end{document}
