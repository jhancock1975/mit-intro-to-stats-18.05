\documentclass[a4paper,11pt]{article}

\usepackage{amsmath}
\usepackage{amssymb}

% for proofs  environment
\usepackage{amsthm}

\usepackage[backend=bibtex]{biblatex}
\bibliography{slides6}

% for probability trees
\usepackage{tikz}
\usetikzlibrary{trees}

% for Venn diagrams
\usetikzlibrary{shapes,backgrounds}
% for plots
\usepackage{ pgfplots}
% inserted on suggestion in warning during compilation
\pgfplotsset{compat=1.9}

%for strikethrough text
\usepackage{soul}

%for R source code listing
\usepackage{listings}

%for block quotes
\usepackage{csquotes}

% For not indenting the first line of paragraphs:
\setlength{\parindent}{0pt}
% define the title
\author{John Hancock}
\title{MIT Introduction to Statistics 18.05 Reading 6A Think Questions }
\begin{document}
% generates the title
\maketitle
% insert the table of contents
\tableofcontents
\section{References and License}
We are answering questions in the material from MIT OpenCourseWare
course 18.05, Introduction to Probability and Statistics.

In this document we are answering questions Orloff and Bloom ask in
\cite{slides6}.

Please see the references section for detailed citation information.

The material for the course is licensed under the terms at
\url{http://ocw.mit.edu/terms}.



\section{Questions about X}

In this section we answer questions Orloff and Bloom ask in \cite{reading6a}
regarding a random variable $X$.

Orloff and Bloom specify that $X$ is defined on $\left[ 0, 1 \right]$, and
the pdf of $X$ is $cx^2$.

\subsection{Value of $c$}

Orloff and Bloom ask us to calculate the value of $c$.  We will use rules
and properties for integration from \cite{basicInt} in order to calculate
the value for $c$.

We know
\begin{equation}
  \int_0^1 cx^2 \, dx = 1.
\end{equation}

Therefore
\begin{equation}
  c \int_0^1 x^2 \, dx = 1.
\end{equation}

The anti-derivative of $x^2$ is $\frac{x^3}{3} +C$, so we can replace the
integral in the equation above with:

\begin{equation}
  c \left(\frac{x^3}{3} \bigg\rvert_0^1 \right)=1.
\end{equation}

We then evaluate the anti-derivative over the interval $\left[0, 1\right]$ to
obtain:

\begin{equation}
  c \left(\frac{1^3}{3} \right)=1.
\end{equation}

This implies $c=3$.











\printbibliography{}
\end{document}
