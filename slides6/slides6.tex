\documentclass[a4paper,11pt]{article}

\usepackage{amsmath}
\usepackage{amssymb}

% for proofs  environment
\usepackage{amsthm}

\usepackage[backend=bibtex]{biblatex}
\bibliography{slides6}

% for probability trees
\usepackage{tikz}
\usetikzlibrary{trees}

% for Venn diagrams
\usetikzlibrary{shapes,backgrounds}
% for plots
\usepackage{ pgfplots}
% inserted on suggestion in warning during compilation
\pgfplotsset{compat=1.9}

%for strikethrough text
\usepackage{soul}

%for R source code listing
\usepackage{listings}

%for block quotes
\usepackage{csquotes}

% For not indenting the first line of paragraphs:
\setlength{\parindent}{0pt}
% define the title
\author{John Hancock}
\title{MIT Introduction to Statistics 18.05 Reading 6A Think Questions }
\begin{document}
% generates the title
\maketitle
% insert the table of contents
\tableofcontents
\section{References and License}
We are answering questions in the material from MIT OpenCourseWare
course 18.05, Introduction to Probability and Statistics.

In this document we are answering questions Orloff and Bloom ask in
\cite{slides6}.

Please see the references section for detailed citation information.

The material for the course is licensed under the terms at
\url{http://ocw.mit.edu/terms}.



\section{Questions about X}

In this section we answer questions Orloff and Bloom ask in \cite{reading6a}
regarding a random variable $X$.

Orloff and Bloom specify that $X$ is defined on $\left[ 0, 1 \right]$, and
the pdf of $X$ is $cx^2$.

\subsection{Value of $c$}

Orloff and Bloom ask us to calculate the value of $c$.  We will use rules
and properties for integration from \cite{basicInt} in order to calculate
the value for $c$.

We know
\begin{equation}
  \int_0^1 cx^2 \, dx = 1.
\end{equation}

Therefore
\begin{equation}
  c \int_0^1 x^2 \, dx = 1.
\end{equation}

The anti-derivative of $x^2$ is $\frac{x^3}{3} +C$, so we can replace the
integral in the equation above with:

\begin{equation}
  c \left(\frac{x^3}{3} \bigg\rvert_0^1 \right)=1.
\end{equation}

We then evaluate the anti-derivative over the interval $\left[0, 1\right]$ to
obtain:

\begin{equation}
  c \left(\frac{1^3}{3} \right)=1.
\end{equation}

This implies $c=3$.

\subsection{Mean, variance, and standard deviation of $X$}
\subsubsection{Mean of $X$}
We use the definition of mean value that Orloff and Bloom give in
\cite{reading6a}.
The mean value of $X$ is
\begin{equation}
\mu = \int_0^1 x\left( 3x^2 \right)\,dx.
\end{equation}

We multiply the terms in the polynomial in the integral above to get:

\begin{equation}
\mu = \int_0^1 \left( 3x^3 \right)\,dx.
\end{equation}

We replace the integral above with its anti-derivative:

\begin{equation}
\mu = \frac{3x^4}{4} \bigg\revert_0^1.
\end{equation}

We evaluate the anti-derivative over the closed interval $\left[0, 1 \right]$ to
find the value of the mean of $X$:

\begin{equation}
\mu = \frac{3}{4}.
\end{equation}

\subsubsection{Variance of $X$}
We use the definition of the variance of a continuous random variable in
\cite{reading6a} to compute the variance of $X$.

The definition of Variance Orloff and Bloom give in \cite{reading6a}:

\begin{equation}
  \text{Var}\left(X \right) = E\left( \left( X-\mu\right)^2\right).
\end{equation}

We use the values for $c$ and $\mu$ that we find above to find:
\begin{equation}
  \text{Var}\left(X\right)=int_0^1 x^{2}3\left(x-\frac{3}{4}\right)^2 \,dx.
\end{equation}

Now we multiply some of the factors in the polynomial in the integral above
to get:

\begin{equation}
  \text{Var}\left(X\right)=int_0^1 x^{2}3\left(x^2-\frac{6x}{4}+\frac{9}{16}\right) \,dx.
\end{equation}


We continue multiplying factors:

\begin{equation}
  \text{Var}\left(X\right)=int_0^1 3x^4-\frac{18x^3}{4}+\frac{27x^2}{16} \,dx.
\end{equation}


Now we replace the integral above with its anti-derivative:

\begin{equation}
  \text{Var}\left(X\right)= \frac{3x^4}{5}-\frac{18x^4}{16}+\frac{27x^3}{48} \bigg\rvert_0^1.
\end{equation}

And, we evaluate the anti-derivative over  the interval $\left[0, 1\right]$:

\begin{equation}
  \text{Var}\left(X\right)= \frac{3}{5}-\frac{18}{16}+\frac{27}{48}=\frac{3}{5}-\frac{18}{16}+\frac{9}{16}.
\end{equation}


\printbibliography{}
\end{document}
