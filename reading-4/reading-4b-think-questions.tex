\documentclass[a4paper,11pt]{article}

\usepackage{amsmath}
\usepackage{amssymb}

\usepackage[backend=bibtex]{biblatex}
\bibliography{4bbib}

% for probability trees
\usepackage{tikz}
\usetikzlibrary{trees}

%for strikethrough text
\usepackage{soul}

%for R source code listing
\usepackage{listings}

% For not indenting the first line of paragraphs:
\setlength{\parindent}{0pt}
% define the title
\author{John Hancock}
\title{MIT Introduction to Statistics 18.05 Reading 4B - \textit{Think}
  Questions }
\begin{document}
% generates the title
\maketitle
% insert the table of contents
\tableofcontents
\section{References and License}
We are answering questions in the material from MIT OpenCourseWare
course 18.05, Introduction to Probability and Statistics.

Please see the references section for detailed citation information.

The material for the course is licensed under the terms at 
\url{http://ocw.mit.edu/terms}.

We are answering the questions that Orloff and Bloom ask after
the word,``think,'' in \cite{reading4b}.

We use documentation in \cite{texDoubleQuotes}, and \cite{texDollarSign}
to write \LaTeX source code of this document.
 
\label{prob1}
\section{Would We Be Willing to Play?}

Would we be willing to play a game of chance where the average expected
loss is \$ $69.44$?

No.

Over a large number of trials, we would expect to win about once
every $36$ times.  Hence loosing about \$$3,500$ before winning the
\$$1,000$; we would eventually run out of money.

If we were to play once, we would have a $\frac{1}{36}$ chance of 
winning, which we do not consider to be good odds.

\section{Expected Value of the Sum of Two Dice} \label{sumTwoDiceSect}

Orloff and Bloom state that the expected  value of rolling one die
is 3.5.

In this think question, Orloff and Bloom ask us what is the expected
value for rolling two dice.

It is $3.5 + 3.5 = 7$.

We can use the formula for expected value that Orloff and Bloom give in
\cite{reading4b} to verify.

We define $X$ to be the random variable whose value is the sum of
the values we roll with two dice. 

We give the probabilities for each of the sums we could roll:

\begin{center}
  \begin{tabular}{ | c | c | c | c | c |c |c | c | c | c | c |c |}
    \hline
    Value $a$ & 2 & 3 & 4 & 5 & 6 & 7 & 8 & 9 & 10 & 11 & 12\\ \hline
    pmf $p \left( a \right)$  & $\frac{1}{36}$ & $\frac{2}{36}$ & 
      $\frac{3}{36}$ & $\frac{4}{36}$ & $\frac{5}{36}$ & 
      $\frac{6}{36}$ & $\frac{5}{36}$ & $\frac{4}{36}$ & 
      $\frac{3}{36}$ & $\frac{2}{36}$ & $\frac{1}{36}$\\ \hline
  \end{tabular}
\end{center}

Then, the expected value, $E \left ( X \right)$ is:

\begin{equation} \label{expValSumTwoDice}
E \left( X \right)  =  
2 \cdot \frac{1}{36}   + 3 \cdot \frac{2}{36}  
+ 4 \cdot \frac{3}{36}   + 5 \cdot \frac{4}{36}   + 6 \cdot \frac{5}{36}  
+ 7 \cdot \frac{6}{36}  + 8 \cdot \frac{5}{36}  + 9 \cdot \frac{4}{36}  
+ 10 \cdot \frac{3}{36}  + 11 \cdot \frac{2}{36}  
+ 12 \cdot \frac{1}{36}
\end{equation}

We use R to do the arithmetic operations on the numbers in the right
hand side of \ref{expValSumTwoDice} to get the value 7:

\begin{lstlisting}
> 2*1/36 + 3*2/36 + 4*3/36 + 5*4/36 + 6*5/36 + 7*6/36 + 8*5/36 + 9*4/36
+ 10*3/36 + 11*2/36 +12*1/36
[1] 7
\end{lstlisting}

\section{If $Y = h \left( X \right)$ does 
  $E \left( Y \right) = h \left( E \left( X \right) \right)$?}

In \cite{reading4b} Orloff and Bloom state that this is not true in
general.

They also ask us if it is true for example 13 in \cite{reading4b}.

We see in \ref{sumTwoDiceSect} that the expected value 
$E \left( X \right)$ for the sum of rolls of two dice is 7.

In example 13 $h \left( X \right) = X^2 -6X + 1$, so
$h \left( E \left ( X \right) \right) = 49 - 42 + 1 = 8$

However, in example 13, Orloff and Bloom show that
$E \left( h \left( x \right) \right) \approx 13.833$ 

Therefore it is not true in example 13 that 
$ E \left( h \left( x \right) \right) =
h \left( E \left ( X \right) \right) $

Therefore it is not true that $E \left( Y \right) = 
h \left( E \left( X \right) \right)$ if $Y = h \left( X \right)$
  

\printbibliography{}
\end{document}
