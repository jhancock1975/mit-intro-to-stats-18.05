\documentclass[a4paper,11pt]{article}

\usepackage{amsmath}
\usepackage{amssymb}

\usepackage[backend=bibtex]{biblatex}
\bibliography{biblio}

% for probability trees
\usepackage{tikz}
\usetikzlibrary{trees}

%for strikethrough text
\usepackage{soul}

%for R source code listing
\usepackage{listings}

% For not indenting the first line of paragraphs:
\setlength{\parindent}{0pt}
% define the title
\author{John Hancock}
\title{MIT Introduction to Statistics 18.05 Reading 4 - Questions }
\begin{document}
% generates the title
\maketitle
% insert the table of contents
\tableofcontents
\section{References and License}
We are answering questions in the material from MIT OpenCourseWare
course 18.05, Introduction to Probability and Statistics.

Please see the references section for detailed citation information.

The material for the course is licensed under the terms at 
\url{http://ocw.mit.edu/terms}.

We are answering the questions that Orloff and Bloom ask in
\cite{reading4Questions}.

We use documentation in \cite{latexSymbols}, \cite{limHowTo}, and
\cite{htmlLatexSymbols} to write \LaTeX source code of this
document.
 
\label{prob1}
\section{Problem 1}
\subsection{Probability Of A Given Value of a Random Variable}

The probability mass functions for all events must sum to one,
therefore the probability that $X=20$ is $\frac{1}{2}$.

\subsection{Value of a Cumulative Distribution Function for a Given
  number}

$ F \left( 17 \right)$ is the sum of the probability mass functions
for values of $X$ that are less than or equal to $17$.  Hence 
$ F \left( 17 \right) = \frac{2}{10} + \frac{1}{10} + \frac{2}{10} = 
\frac{1}{2}$.

\subsection{Value of a Cumulative Distribution Function for another
 Given number}

$ F \left( 20 \right)$ is the sum of the probability mass functions
for values of $X$ that are less than or equal to $20$.  Hence 
$ F \left( 20 \right) = \frac{2}{10} + \frac{1}{10} + \frac{2}{10}  
 + \frac{1}{2} = 1$.

\subsection{Value of a Cumulative Distribution Function}

$ F \left( 25 \right)$ is the sum of the probability mass functions
for values of $X$ that are less than or equal to $25$.  Hence 
$ F \left( 25 \right) = \frac{2}{10} + \frac{1}{10} + \frac{2}{10}  
 + \frac{1}{2} = 1$.

\subsection{Problem 2}

Orloff and Bloom ask for the probability mass function for the
event $P\left (X = 3 \right)$ where $X$ follows the binomial
distribution $\text{binomial} \left( 6, \frac{1}{2} \right)$.

We use the probability mass function that Orloff and Bloom give in
section 3.2 of \cite{reading4} for the case where $n = 6$, 
$p = \frac{1}{2}$ and $k=3$.

Therefore the probability mass function is:
\begin{equation}
  p \left( 3 \right) = \binom{6}{3} \left(\frac{1}{2} \right) ^ {3}
    \left( \frac{1}{2} \right) ^ {\left( 6-3 \right)} = \frac{20}{64} 
    = 0.3125
\end{equation}


\printbibliography{}
\end{document}
