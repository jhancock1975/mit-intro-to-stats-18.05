\documentclass[a4paper,11pt]{article}

\usepackage{amsmath}
\usepackage{amssymb}

\usepackage[backend=bibtex]{biblatex}
\bibliography{biblio}

% for probability trees
\usepackage{tikz}
\usetikzlibrary{trees}

%for strikethrough text
\usepackage{soul}

%for R source code listing
\usepackage{listings}

% For not indenting the first line of paragraphs:
\setlength{\parindent}{0pt}
% define the title
\author{John Hancock}
\title{MIT Introduction to Statistics 18.05 Reading 4 - \textit{Think}
  Questions }
\begin{document}
% generates the title
\maketitle
% insert the table of contents
\tableofcontents
\section{References and License}
We are answering questions in the material from MIT OpenCourseWare
course 18.05, Introduction to Probability and Statistics.

Please see the references section for detailed citation information.

The material for the course is licensed under the terms at 
\url{http://ocw.mit.edu/terms}.

We are answering the questions that Orloff and Bloom ask after
the word, "think," in \cite{reading4}.

We use documentation in \cite{latexSymbols}, \cite{limHowTo}, and
\cite{htmlLatexSymbols} to write \LaTeX source code of this
document.
 
\label{prob1}
\section{The Probability Mass Function for $ Z\left( i, j \right) = i + j$}

We write the $pmf$ for the events that we roll two dice and the sum
of the values we roll is a particular value of $a$:
\begin{center}
  \begin{tabular}{ | c | c | c | c | c |c |c | c | c | c | c |c |}
    \hline
    Value $a$ & 2 & 3 & 4 & 5 & 6 & 7 & 8 & 9 & 10 & 11 & 12\\ \hline
    pmf $p \left( a \right)$  & $\frac{1}{36}$ & $\frac{2}{36}$ & 
      $\frac{3}{36}$ & $\frac{4}{36}$ & $\frac{5}{36}$ & 
      $\frac{6}{36}$ & $\frac{5}{36}$ & $\frac{4}{36}$ & 
      $\frac{3}{36}$ & $\frac{2}{36}$ & $\frac{1}{36}$\\ \hline
  \end{tabular}
\end{center}

Orloff and Bloom ask if this looks familiar.  It does not look familiar
to us at this time.

\section{Properties of Cumulative Distribution Functions (cdf's)}

\subsection{cdf's are non-decreasing}

Cdf's are non-decreasing because they are sums of probability mass
function (pmf) values.

Orloff and Bloom define probability mass functions in \cite{reading4}, 
and they state that the value of a probability mass function $p$,
for any input $a$ is always greater than or equal to 0.

If we assume that for some cdf $F$ that $F \left( b \right) <
F \left(a \right), b > a$, that would mean that for some value
$c, a < c \leq b$, $p\left( c \right) < 0$.  Our assumption thus
forces a contradiction of the definition of probability mass functions,
so it must be wrong.  Therefore cdf's are non-decreasing.

\subsection{Cdf's approach 0 as $a \rightarrow -\infty$}

Orloff and Bloom define random variables as functions $X$ that
map elements $\omega$ of a sample space $\Omega$ to elements of
$\mathbb{R}$. We denote a random variable as $X$.  Orloff and Bloom
define the mapping in symbols as $X : \Omega \rightarrow \mathbb{R}$.

Orloff and Bloom define a probability mass function as having the
value 0 for values that the random variable $X$ never takes.

Note: this reasoning assumes a finite set of values that the random
variable may take.

We can order the values that $X$ takes because they are elements of
$\mathbb{R}$.  There must be some  least value $l$ that $X$ takes.
For any real number less than $l$, the probability mass function
has value $0$.  Therefore the sums of probability mass functions 
$p\left( a \right)$ for $a < l$ will also be $0$.  We note that
these sums satisfy the definition of a cumulative distribution functions
$F \left( l \right)$. Therefore we conclude
\begin{equation}
  \lim_{a \rightarrow -\infty} F \left( a \right) = 0
\end{equation}



\subsection{Cdf's have values between 0 and 1}

We show in the previous section that Cdf's have a minimum value
of 0 for sufficiently small values of $a$.

Orloff and Bloom define probability mass functions $p \left( a \right)$
to be the probability of the event that a random variable $X$ takes 
the value $a$.

In this section we define $\Omega$ to be the set of all events
that a random variable takes on all of its possible values, and $\omega$
to be an element of $\Omega$.

We claim that the elements $\omega$ are disjoint.

We justify this claim in a proof by contradiction.  
If some elements $\omega$ were not disjoint, then two events in 
$\Omega$ would have elements in common.  This would mean that events
where $X$ takes on the same value $a$ are considered different.  This
is absurd because we cannot distinguish the events.  Therefore the
elements of $\Omega$ are disjoint.

Since the elements of $\Omega$ are disjoint, the sum of the probability
mass functions $P\left( X = a \right)$ are the sums of the probabilities
of the unions of elements of $\Omega$.

The sum of probabilities of all events in a sample space is one.

Therefore the maximum value of a Cdf is one.

\subsection{Cdf's approach 1 as $a \rightarrow \infty$}

We make a note that Orloff and Bloom define a cdf $F\left( a \right)$ 
as the sum of all pmf's $p \left( b \right)$ where $b$ is any real
number less than or equal to $a$.

In the previous section we showed that the sum of probability mass
functions for all events that a random variable attains values
is 1.

We note that as $a \rightarrow \infty$, in order to compute the
cumulative distribution function $F \left( a \right)$ we are adding
more probability mass functions $p \left( a \right)$ for events that
our random variable takes the value $a$.  At some point, we will
include all possible values that $X$ is defined to take, as $a$
grows larger and larger. We will include all events in the sample
space.

Hence, by our reasoning in the previous section:
\begin{equation}
  \lim_{a \rightarrow \infty} F \left( a \right) = 1
\end{equation}

\section{Binomial Distribution Probability Not Surprising}

We equate tossing three or more heads out of five tosses with choosing 3
or more elements from a set of 5.

There are $\binom{5}{3} + \binom{5}{4} + \binom{5}{5}$ ways to choose
these elements.  This is equal to $10 + 5 + 1 = 16$.

There are $2^{5}$ possible ways of tossing a coin, so a result of
$\frac{1}{2}$ is not surprising.

\section{Sum of Random Variables}

Orloff and Bloom give two independent variables $X$, and $Y$.

Furthermore:


\begin{equation}
X \text{\textasciitilde} \text{binomial} \left( n, \frac{1}{2}, \right)
\end{equation}

and \begin{equation}
Y \text{\textasciitilde} \text{binomial} \left( m, \frac{1}{2}, \right)
\end{equation}

$X$ and $Y$ are independent random variables so the probability
of the union of events where $X$ takes a value and $Y$ takes a value
is the sum of the probabilities of the individual events.  We learn
this in \cite{reading2}, section 3, "Some rules of probability."

We wish to know what $X + Y$ is.

Orloff and Bloom define random variables as functions with output values
in the real numbers, so it makes sense to write $X+Y$ to mean the
sum of the values of functions with output values in the real numbers.
In this case, it is the sum of the number of times we toss heads in
$m+n$ trials.

Therefore $X+Y$ is the count of the number of heads we toss
for $m+n$ tosses, with probability of $\frac{1}{2}$.

Orloff and Bloom in \cite{reading4}, section 3 state that coin tosses
are Bernoulli trials, with probability $\frac{1}{2}$.

$X+Y$ is the number of successes in $m+n$ independent Bernoulli trials,
where the  probability of success is $\frac{1}{2}$.

Therefore $X+Y$ is the same as  the experiment that the binomial distribution
models that Orloff and Bloom define in example $7$ in section 3.2 of 
\cite{reading4}, if we substitute $m+n$ for the number of coin tosses.  

Hence, $X+Y \text{\textasciitilde} \text{binomial} \left(m+n, \frac{1}{2}
\right)$

\printbibliography{}
\end{document}
