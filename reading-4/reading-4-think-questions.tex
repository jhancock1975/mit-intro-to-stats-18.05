\documentclass[a4paper,11pt]{article}

\usepackage{amsmath}
\usepackage{amssymb}

\usepackage[backend=bibtex]{biblatex}
\bibliography{biblio}

% for probability trees
\usepackage{tikz}
\usetikzlibrary{trees}

%for strikethrough text
\usepackage{soul}

%for R source code listing
\usepackage{listings}

% For not indenting the first line of paragraphs:
\setlength{\parindent}{0pt}
% define the title
\author{John Hancock}
\title{MIT Introduction to Statistics 18.05 Reading 4 - \textit{Think}
  Questions }
\begin{document}
% generates the title
\maketitle
% insert the table of contents
\tableofcontents
\section{References and License}
We are answering questions in the material from MIT OpenCourseWare
course 18.05, Introduction to Probability and Statistics.

Please see the references section for detailed citation information.

The material for the course is licensed under the terms at 
\url{http://ocw.mit.edu/terms}.

We are answering the questions that Orloff and Bloom ask after
the word, "think," in \cite{reading4}.

We use documentation in \cite{latexSymbols} \LaTeX source code of this
document.
 
\label{prob1}
\section{The Probability Mass Function for $ Z\left( i, j \right) = i + j$}

We write the $pmf$ for the events that we roll two dice and the sum
of the values we roll is a particular value of $a$:
\begin{center}
  \begin{tabular}{ | c | c | c | c | c |c |c | c | c | c | c |c |}
    \hline
    Value $a$ & 2 & 3 & 4 & 5 & 6 & 7 & 8 & 9 & 10 & 11 & 12\\ \hline
    pmf $p \left( a \right)$  & $\frac{1}{36}$ & $\frac{2}{36}$ & 
      $\frac{3}{36}$ & $\frac{4}{36}$ & $\frac{5}{36}$ & 
      $\frac{6}{36}$ & $\frac{5}{36}$ & $\frac{4}{36}$ & 
      $\frac{3}{36}$ & $\frac{2}{36}$ & $\frac{1}{36}$\\ \hline
  \end{tabular}
\end{center}

Orloff and Bloom ask if this looks familiar.  It does not look familiar
to us at this time.

\section{Properties of Cumulative Distribution Functions (cdf's)}

\subsection{cdf's are non-decreasing}

Cdf's are non-decreasing because they are sums of probability mass
function (pmf) values.

Orloff and Bloom define probability mass functions in \cite{reading4}, 
and they state that the value of a probability mass function $p$,
for any input $a$ is always greater than or equal to 0.

If we assume that for some cdf $F$ that $F \left( b \right) <
F \left(a \right), b > a$, that would mean that for some value
$c, a < c \leq b$, $p\left( c \right) < 0$.  Our assumption thus
forces a contradiction of the definition of probability mass functions,
so it must be wrong.  Therefore cdf's are non-decreasing.

\subsection{Cdf's have values between 0 and 1}

A distinct value of a pmf is the probability that a random variable 
takes a given value $\lfloor a \rfloor$.

\printbibliography{}
\end{document}
