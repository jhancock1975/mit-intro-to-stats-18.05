\documentclass[a5paper,11pt]{article}

%for coloring cell in a table
\usepackage[table]{xcolor}% http://ctan.org/pkg/xcolor

\usepackage{amsmath}
\usepackage{amssymb}

% for proofs  environment
\usepackage{amsthm}

% for 3d plots
\usepackage{pgfplots}
\usepackage{pgfplotstable}
\usepgfplotslibrary{patchplots}

\usepackage[backend=bibtex]{biblatex}
\bibliography{prob-set-7}

% for probability trees
\usepackage{tikz}
\usetikzlibrary{trees}

% for Venn diagrams
\usetikzlibrary{shapes,backgrounds}

% for plots
\usepackage{ pgfplots}
% inserted on suggestion in warning during compilation
\pgfplotsset{compat=1.9}

%for strikethrough text
\usepackage{soul}

%for R source code listing
\usepackage{listings}

%for block quotes
\usepackage{csquotes}

%for Theorems & Lemmas
\newtheorem{thm}{Theorem}
\newtheorem{lem}[thm]{Lemma}

% For not indenting the first line of paragraphs:
\setlength{\parindent}{0pt}

% define the title
\author{John Hancock}
\title{Problem Set 7}

\begin{document}

% generates the title
\maketitle

% insert the table of contents
\tableofcontents

\section{References and License}
We are answering questions in the material from MIT OpenCourseWare
course 18.05, Introduction to Probability and Statistics.

In this document we are answering questions Orloff and Bloom ask in
\cite{probSet7}.

Please see the references section for detailed citation information.

The material for the course is licensed under the terms at
\url{http://ocw.mit.edu/terms}.

\section{Problem 1}
\subsection{Confident Coin}
We define the null hypothesis $H_{0}$ as, "the coin is fair."  
The 7\% figure is the $p$-value for a significance test where the
null distribution is a binomial distribution.

\section{Problem 2}
\subsection{Error Types}
The null hypothesis $H_{0}$ is that the testee is telling the truth.  The 
members of the positive class are the members that $H_{0}$ is true for.
We have a type I error if we label someone as being in the negative class
when they should be in the positive class.  Therefore a type I error is when 
we interpret the polygraph test results to mean someone is lying, when she
or he is actually telling the truth.

On the other hand, the alternative hypothesis, $H_{A}$ is that someone is
lying.  The members of the negative class are the people such that the 
alternative hypothesis is true.  A type II error is when we label a member
of the negative class as negative, when that member is actually in the
positive class.  Therefore we commit a type II error when we interpret
the polygraph test results to mean that a person is telling the truth when
that person is lying.
interpret the 
\printbibliography{}

\end{document}
