\documentclass[a5paper,11pt]{article}

%for coloring cell in a table
\usepackage[table]{xcolor}% http://ctan.org/pkg/xcolor

\usepackage{amsmath}
\usepackage{amssymb}

% for proofs  environment
\usepackage{amsthm}

% for 3d plots
\usepackage{pgfplots}
\usepackage{pgfplotstable}
\usepgfplotslibrary{patchplots}

\usepackage[backend=bibtex]{biblatex}
\bibliography{notes}

% for probability trees
\usepackage{tikz}
\usetikzlibrary{trees}

% for Venn diagrams
\usetikzlibrary{shapes,backgrounds}

% for plots
\usepackage{ pgfplots}
% inserted on suggestion in warning during compilation
\pgfplotsset{compat=1.9}

%for strikethrough text
\usepackage{soul}

%for R source code listing
\usepackage{listings}

%for block quotes
\usepackage{csquotes}

%for Theorems & Lemmas
\newtheorem{thm}{Theorem}
\newtheorem{lem}[thm]{Lemma}

% For not indenting the first line of paragraphs:
\setlength{\parindent}{0pt}

% define the title
\author{John Hancock}
\title{Slides 17 Notes}

\begin{document}

% generates the title
\maketitle

% insert the table of contents
\tableofcontents

\section{References and License}

In this document we are recording notes on reading material in
\cite{classSlides17}.

Please see the references section for detailed citation information.

The material for the course is licensed under the terms at
\url{http://ocw.mit.edu/terms}.

\section{Significance Level}
Significance level is area under the curve in the rejection region. So if
a test has a high significance level, it means it has a large rejection
region. In turn, it means, it would be easy to fail the test - it's easy
to reject the null hypothesis in favor of the alternative because the
rejection region is so big.

\section{Board Question on Significance Testing}
There is a board question on null hypothesis significance testing in
\cite{classSlides17}.  This is our attempt to solve it.
We are given a null hypothesis $H_0$ that some data follows a normal 
distribution $N\left(5, 10^2 \right)$.

Orloff and Bloom also give us an alternative hypothesis that the data follows
a normal distribution $N\left(\mu, 10^2\right)$ where $\mu  \neq 5$.

They give a test statistic $z$ which is equal to the standardized mean of the
data $\bar{x}$.

Finally they give a significance level of $\alpha = 0.05$.  We pause to 
remember that the significance level is the area under the probability 
density function where the $x$ axis is in the rejection region.

Orloff and Bloom give us that the test statistic $z$ is the standardized
sample mean $\bar{x}$.  Therefore, given the null hypothesis $H_0$ above 
\begin{equation}
z=\frac{\bar{x} -\mu}{\frac{\sigma}{\sqrt{64}}}\,.
\end{equation} 

We us the values Orloff and Bloom give for the symbols in the equation above 
to calculate the value
\begin{equation}
z=\frac{6.25 -5}{\frac{10}{8}}\,,
\end{equation} 
The equation above simplifies to 
\begin{equation}
z=\frac{\frac{5}{4}}{\frac{5}{4}}\,.
\end{equation}
Hence, $z=1$.

We use R's qnorm function to find $z_0.025 \approx 1.96$.  $z$ is less than
1.96, so our test statistic does not fall in the rejection region.  We use
the pnorm function of R \cite{rTut2Tail},to find that the p-value for our 
test statistic is 0.32.  This is larger than the significance level
of 0.05 - it means that if the null hypothesis is true, then there
is a 0.32 probability of our observing the test statistic.

It is interesting to note that the R code for calcuating the p-value is
\begin{lstlist}
2*pnorm(1, lower.tail=FALSE)
\end{lstlist}
\printbibliography{}

\end{document}
