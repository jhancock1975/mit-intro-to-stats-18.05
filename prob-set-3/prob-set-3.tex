\documentclass[a4paper,11pt]{article}

\usepackage{amsmath}
\usepackage{amssymb}

% for proofs  environment
\usepackage{amsthm}

\usepackage[backend=bibtex]{biblatex}
\bibliography{prob-set-3}

% for probability trees
\usepackage{tikz}
\usetikzlibrary{trees}

% for plots
\usepackage{ pgfplots}
% inserted on suggestion in warning during compilation
\pgfplotsset{compat=1.9}

%for strikethrough text
\usepackage{soul}

%for R source code listing
\usepackage{listings}

%for block quotes
\usepackage{csquotes}

% For not indenting the first line of paragraphs:
\setlength{\parindent}{0pt}
% define the title
\author{John Hancock}
\title{MIT Introduction to Statistics 18.05 Problem Set 3 }
\begin{document}
% generates the title
\maketitle
% insert the table of contents
\tableofcontents
\section{References and License}
We are answering questions in the material from MIT OpenCourseWare
course 18.05, Introduction to Probability and Statistics.

In this document we are answering questions Orloff and Bloom ask in
\cite{probSet3}.

Please see the references section for detailed citation information.

The material for the course is licensed under the terms at
\url{http://ocw.mit.edu/terms}.

We use documentation in
to write \LaTeX source code for this document.
\section{Independence}

In this section we answer a problem in \cite{probSet2} that involves rolling
two six sided dice.

\subsection{Pairwise and mutual independence}
We define two events, $A$, and $B$ to be pairwise independent if
$P \left( A \cap B \right) = P \left( A \right) P \left(B \right)$.

For this problem Orloff and Bloom give us the definition of mutual independence
for three events, $A$, $B$, and $C$.  $A$, $B$, and $C$ are mutally independent
if
\begin{equation}
	P \left( A \cap B \cap C\right)
	= P \left(A \right) P \left(B \right) P \left( C \right)
\end{equation}

In this section, Orloff and Bloom give the following definitions for events
$A$, $B$, and $C$:
\begin{itemize}
	\item $A$ is the event that we roll an odd number with the first die.
	\item $B$ is the event that we roll an odd number with the second die.
	\item $C$ is the event that the sum of the numbers we roll is odd.
\end{itemize}

$A$, $B$, and $C$ are not mutually independent.  Whatever the
$A$, $B$, and $C$ probabilities of $A$, $B$, and $C$ are individually, the
probability of their intersection is $0$ since the sum of two odd numbers
is always an even number.



\printbibliography{}
\end{document}
