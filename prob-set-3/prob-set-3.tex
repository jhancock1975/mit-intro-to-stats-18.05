\documentclass[a4paper,11pt]{article}

\usepackage{amsmath}
\usepackage{amssymb}

% for proofs  environment
\usepackage{amsthm}

\usepackage[backend=bibtex]{biblatex}
\bibliography{prob-set-3}

% for probability trees
\usepackage{tikz}
\usetikzlibrary{trees}

% for Venn diagrams
\usetikzlibrary{shapes,backgrounds}
% for plots
\usepackage{ pgfplots}
% inserted on suggestion in warning during compilation
\pgfplotsset{compat=1.9}

%for strikethrough text
\usepackage{soul}

%for R source code listing
\usepackage{listings}

%for block quotes
\usepackage{csquotes}

% For not indenting the first line of paragraphs:
\setlength{\parindent}{0pt}
% define the title
\author{John Hancock}
\title{MIT Introduction to Statistics 18.05 Problem Set 3 }
\begin{document}
% generates the title
\maketitle
% insert the table of contents
\tableofcontents
\section{References and License}
We are answering questions in the material from MIT OpenCourseWare
course 18.05, Introduction to Probability and Statistics.

In this document we are answering questions Orloff and Bloom ask in
\cite{probSet3}.

Please see the references section for detailed citation information.

The material for the course is licensed under the terms at
\url{http://ocw.mit.edu/terms}.

We use documentation in \cite{vennDiagram}, \cite{nodePos}
to write \LaTeX source code for this document.
\section{Independence}

In this section we answer a problem in \cite{probSet2} that involves rolling
two six sided dice.

\subsection{Pairwise and mutual independence}
We define two events, $A$, and $B$ to be pairwise independent if
$P \left( A \cap B \right) = P \left( A \right) P \left(B \right)$.

For this problem Orloff and Bloom give us the definition of mutual independence
for three events, $A$, $B$, and $C$.  $A$, $B$, and $C$ are mutally independent
if
\begin{equation}
	P \left( A \cap B \cap C\right)
	= P \left(A \right) P \left(B \right) P \left( C \right)
\end{equation}

In this section, Orloff and Bloom give the following definitions for events
$A$, $B$, and $C$:
\begin{itemize}
	\item $A$ is the event that we roll an odd number with the first die.
	\item $B$ is the event that we roll an odd number with the second die.
	\item $C$ is the event that the sum of the numbers we roll is odd.
\end{itemize}

$A$, $B$, and $C$ are not mutually independent.  Whatever the
$A$, $B$, and $C$ probabilities of $A$, $B$, and $C$ are individually, the
probability of  $P \left( A \cap B \cap C\right)$ is $0$ since the sum of two
odd numbers is always an even number.

\subsection{Venn diagram}

Orloff and Bloom give the following Venn diagram:

\def\firstcircle{(0,0) circle (1.5cm)}
\def\secondcircle{(45:2cm) circle (1.5cm)}
\def\thirdcircle{(0:2cm) circle (1.5cm)}

\begin{center}
\begin{tikzpicture}
\begin{scope}[shift={(3cm,-5cm)}, fill opacity=0.3]
        \fill[red] \firstcircle;
        \fill[green] \secondcircle;
        \fill[blue] \thirdcircle;
        \draw \firstcircle node[below, opacity=1] {$0.225$} node[xshift=1cm, yshift=0.5cm, opacity=1] {$0.125$} node[left, yshift=0.5cm, opacity=1] {$A$};
        \draw \secondcircle node [above, opacity=1] {$0.225$}
				%intersection of a and b below:
					node[below, xshift=0.5cm, opacity=1]{$0.1$}
				%intersection of b and c below:
					node[below, xshift=-1.14cm, opacity=1]{$0.05$}
					node[left, yshift=0.5cm, opacity=1] {$B$};
        \draw \thirdcircle node [below, opacity=1] {$0.175$}
					node[left, xshift=-0.75cm, yshift=-0.5cm, opacity=1]{0.1}
					node[left, yshift=0.5cm, opacity=1] {$C$};
					node[xshift=]
    \end{scope}
\end{tikzpicture}
\end{center}

And ask us whether or not the events in the Venn diagram above are mutually
independent.

These events are not mutually independent because
\begin{equation}
	P \left(A \right) P \left(B \right) P \left( C \right) =
		0.225 \times 0.225 \times 0.175 = 0.008859375.
\end{equation}

However, in the Venn diagram above, Orloff and Bloom give us that
$P \left( A \cap B \cap C\right) = 0.125$

Therefore the events are not mutually independent.

\subsection{How many kids}
For this question we use the same assumptions about the probability of the
gender that a child is born with that Orloff and Bloom use in example 9 of
\cite{reading4a}.

We define the following events:

\begin{itemize}
\item $A$ is the event that the children in a family are both boys and girls.
\item $B$ is the event that at most one of the children is a girl.
\item $C_{i,b}$ is the event that child number $i$ is a boy.
\item $C_{i,g}$ is the event that child number $i$ is a girl.
\end{itemize}

Our goal is to construct a sample space such that $A$ and $B$ are independent.
The definition of independent events is in \cite{reading3}.

We rely on the same assumption that Orloff and Bloom make in \cite{reading4a}
regarding the probability of the genders of sequences of children.

Therefore we assume  $P\left( C_{i,b} \right) = 0.5$, and $P\left( C_{i,g} \right) = 0.5$,
independent of the event that any other child is a boy or a girl.

We write the following table to discover the number of children where 
$A$, and $B$ will meet the definition of independent events.

We fill in one cell in the table below for each possible
sequnce of three children in the family being boys
or girls.

\begin{center}
  \begin{tabular}{ | c | c | c | c | }
    \hline
	C_{1,b}C_{2,b}C_{3,b} & C_{1,b}C_{2,b}C_{3,g} & C_{1,b}C_{2,g}C_{3,b} & C_{1,b}C_{2,g}C_{3,g}   \\ \hline
    	C_{1,g}C_{2,b}C_{3,b} & C_{1,g}C_{2,b}C_{3,g} & C_{1,g}C_{2,g}C_{3,b} & C_{1,g}C_{2,g}C_{3,g}   \\ \hline
  \end{tabular}
\end{center}

In the table above there are $6$ sequences that are in $A$, so $P\left( A \right) = \frac{6}{8}$.

Also, there are $4$ sequences in $B$, so $P\left( B \right) = \frac{4}{8}$. 

Moreover, there are $3$ sequences where there is at most one girl, and
the children are both boys and girls.  Therefore $P\left( A \cap B \right) = \frac{3}{8}$.

$A$ and $B$ are indpendent since
\begin{equation}
P\left(A \right) P \left( B \right)
  = \left( \frac{6}{8} \right) \left( \frac{4}{8} \right)
  = \frac{24}{64} = \frac{3}{8}.
\end{equation}

Therefore $P\left(A \right) P\left(B \right) = P \left(A \cap B \right)$,
so $A$ and $B$ must be independent events.

We made these calculations assuming that there are $3$ children,
therefore the number of children we require in order for $A$, and
$B$ to be independent events is $3$.

\section{Dice}

In this section we will deal with problems that Orloff and
Bloom ask about the random variable $X$, that is equal to the value
we roll with a fair $4$-sided die, the random variable
$Y$, that is equal to the value we roll with a fair $6$ sided die,
and the random variable $Z$, that is equal to the average
of $X$ and $Y$.

\subsection{Standard deviation of $X$, $Y$, and $Z$}


We use the definition of variance and standard deviation in \cite{reading5a} to
calculate the standard deviations $\sigma \left( X \right)$, $\sigma \left( Y \right)$.

\begin{equation}
  \sigma \left(X \right) = 1.708
\end{equation}


\printbibliography{}
\end{document}
