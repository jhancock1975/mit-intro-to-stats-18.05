\documentclass[a4paper,11pt]{article}

\usepackage{amsmath}
\usepackage{amssymb}

% for proofs  environment
\usepackage{amsthm}

% for 3d plots
\usepackage{pgfplots}
\usepackage{pgfplotstable}
\usepgfplotslibrary{patchplots}

\usepackage[backend=bibtex]{biblatex}
\bibliography{reading7qub}

% for probability trees
\usepackage{tikz}
\usetikzlibrary{trees}

% for Venn diagrams
\usetikzlibrary{shapes,backgrounds}
% for plots
\usepackage{ pgfplots}
% inserted on suggestion in warning during compilation
\pgfplotsset{compat=1.9}

%for strikethrough text
\usepackage{soul}

%for R source code listing
\usepackage{listings}

%for block quotes
\usepackage{csquotes}

% For not indenting the first line of paragraphs:
\setlength{\parindent}{0pt}
% define the title
\author{John Hancock}
\title{MIT Introduction to Statistics 18.05 Reading 7B Questions}
\begin{document}
% generates the title
\maketitle
% insert the table of contents
\tableofcontents
\section{References and License}
We are answering questions in the material from MIT OpenCourseWare
course 18.05, Introduction to Probability and Statistics.

In this document we are answering questions Orloff and Bloom ask in
\cite{reading7qub}.

Please see the references section for detailed citation information.

The material for the course is licensed under the terms at
\url{http://ocw.mit.edu/terms}.

We use documentation in  \cite{logicNot} to write \LaTeX source code for this
document.

\section{Non-zero covariance}
The first question Orloff and Bloom ask is: if Cov$\left(X, Y \right)\neq 0$,
are $X$, and $Y$ independent?

In \cite{reading7b} Orloff and Bloom state that, if $X$, and $Y$ are
independent, then Cov$\left(X, Y \right)=0$.

We use symbolic logic to answer this question.  The validity of our answer
rests on Peter Williams' work in \cite{symbolicLog}

Let $p$ be the statement, "$X$, and $Y$ are independent", and $q$ be the
statement, "covariance is $0$."

Then, for this question, we know that $p$ implies $q$.

We use a result from \cite{contrapositive} that tells us that $p$ implies
$q$, if and only if $\neg q$ implies $\neg p$.

In this case $\neg q$ is the statement, "covariance $\neq 0$," and $\neg p$ is
the statement, "$X$, and $Y$ are not independent."

Therefore we know that if Cov$\left( X, Y \right) \neq 0$, then $X$, and $Y$
are not independent.

\section{Zero covariance}

For this question, Orloff and Bloom ask us a variation on the first, "Suppose
Cov$\left(X, Y\right)=0$ are $X$ and $Y$ independent?"

In \cite{reading7b} Orloff and Bloom state that if $X$ and $Y$ are independent,
then their covariance is 0.  They state this as a property of covariance.

However, in \cite{reading7b} Orloff and Bloom give an example where two
random variables are not independent, and have a covariance of $0$.

Therefore cases exist where two independent variables have a covariance of $0$,
and cases exist where two variables that are not independent have a covariance
of $0$.

Therefore the answer to this question is maybe.

\section{Joint-distributed discrete random variables}
For these problems, Orloff and Bloom give us the following joint distribution
of discrete random variables, $X$, and $Y$:

\begin{center}
  \begin{tabular}{ | c | c | c | c |}
    \hline
    $\frac{X}{Y}$ & 0    & 2   &      \\ \hline
    0             & 0.25 & 0.3 & 0.55 \\ \hline
    2             & 0.25 & 0.2 & 0.45 \\ \hline
                  & 0.5  & 0.5 &      \\ \hline
  \end{tabular}
\end{center}

\subsection{Independence}

The first question Orloff and Bloom have for us on the joint distriubtion is
to ask us whether or not $X$, and $Y$ are independent.

In \cite{reading7b} Orloff and Bloom state that variables $X$ and $Y$ of a
joint distribution are independent if each probability
$p \left(x_i, y_j \right)$ in their joint distribution table is the product of
its respective marginal probabilities
$p\left(x_i \right)$ and $p\left(y_j \right)$.

Therefore, if we find one probability in the table above that is not
equal to the product of its respective marginal probabilities, then $X$ and
$Y$ are not independent.

The joint probability for $X=0$, and $Y=0$ is $0.25$.  The respective marginal
probabilities for this joint probability are $p\left( x_1 \right)=0.55$ and
p $\left( y_1 \right) = 0.5$.  $0.5 \times 0.55 = 0.275$.  $0.275 \neq 0.25$,
so $X$ annd $Y$ are not independent.

\subsection{Covariance}

We use the formula for covariance of discrete random variables from
\cite{reading7b} to compute the covariance of $X$ and $Y$:

\begin{equation}\label{formCov}
\text{Cov}\left(X, Y \right) =
  \left( \sum_{i=1}^{n} \sum_{j=1}^{m} p\left(x_i, y_j \right) x_i, y_j \right)
      - \mu_{X}\mu_{Y}.
\end{equation}

In order to use equation \ref{formCov} we must compute $\mu_X$, and $\mu_Y$.

$\mu_X$ is the expected value of $X$, which is the sum of the products of
the values of $X$ and their respective probabilities.

$X$ takes values $0$, and $2$, so only the product of $2$ and the probability
of $X=2$ will contribute to the expected value of $X$.  There is a probability
of $0.3$ that $X$ will have the value $2$ when $Y=0$, or a probability of $0.2$
that $X$ will will have the value $2$ when $Y=2$. We apply the law of total
probability to compuute the probability of $X=2$.  The probability of $X=2$ is
therefore $0.2 + 0.3 = 0.5$. Therefore the expected value of $X$,
$\mu_X= 0.5 \times 2 = 1$.

We apply reasoning similar to what we do above to compute the expected value
$\mu_Y$ of $Y$. $\mu_Y=0.9$.

The next step in our calculation of Cov$\left( X, Y \right)$, is to compute
the value of the double sum

\begin{equation}
  \left( \sum_{i=1}^{n} \sum_{j=1}^{m} p\left(x_i, y_j \right) x_i, y_j \right)
  = \left( 0.25 \times 0 \times 0 \right) +
    \left( 0.3 \times 2 \times 0 \right) +
    \left( 0.25 \times 0 \times 0 \right) +
    \left( 0.2 \times 2 \times 2 \right)
\end{equation}

The terms on the right hand side of the equation above sum to $0.8$.

Now we know the values of the double sum in equation \ref{formCov} and $\mu_X$,
and $\mu_Y$, so we can calculate Cov$\left( X, Y \right) = -0.1$.

\printbibliography{}
\end{document}
