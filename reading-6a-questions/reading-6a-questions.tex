\documentclass[a4paper,11pt]{article}

\usepackage{amsmath}
\usepackage{amssymb}

% for proofs  environment
\usepackage{amsthm}

\usepackage[backend=bibtex]{biblatex}
\bibliography{reading-6a-questions}

% for probability trees
\usepackage{tikz}
\usetikzlibrary{trees}

% for Venn diagrams
\usetikzlibrary{shapes,backgrounds}
% for plots
\usepackage{ pgfplots}
% inserted on suggestion in warning during compilation
\pgfplotsset{compat=1.9}

%for strikethrough text
\usepackage{soul}

%for R source code listing
\usepackage{listings}

%for block quotes
\usepackage{csquotes}

% For not indenting the first line of paragraphs:
\setlength{\parindent}{0pt}
% define the title
\author{John Hancock}
\title{MIT Introduction to Statistics 18.05 Reading 6A Think Questions }
\begin{document}
% generates the title
\maketitle
% insert the table of contents
\tableofcontents
\section{References and License}
We are answering questions in the material from MIT OpenCourseWare
course 18.05, Introduction to Probability and Statistics.

In this document we are answering questions Orloff and Bloom ask in
\cite{reading6aQu}.

We use documentation in \cite{typesetDx} in order to write the \LaTeX code
for this document.

Please see the references section for detailed citation information.

The material for the course is licensed under the terms at
\url{http://ocw.mit.edu/terms}.

\section{The mean does not divde probability mass in half.}
The mean value of a continuous random variable $X$ does not necessarily divide
the area under the curve of its probability mass function because we may
have the case where more values $X$ occur where $X$ is greater (or less than)
the median value of $X$.

\section{Questions on $X \sim \text{U} \left( 0, 4 \right)$}

\subsection{Mean and variance of U$\left(0,4 \right)$}

We use the definition of the mean value of a continuous random variable from
\cite{reading6a}.

The mean, $\mu$, of U$\left( 0,4 \right)$ is
\begin{equation} \label{meanval}
  \mu = \int_0^{4} \frac{x}{4}\,dx.
\end{equation}

We use the power rule for integrals \cite{basicInt} to replace the right hand
side of \ref{meanval} with its anti-derivative:

\begin{equation}
  \mu =  \frac{x^2}{8} \bigg\rvert_0^4.
\end{equation}

Now we can evaluate the antiderivative over the interval $\left[ 0, 4 \right]$:


\begin{equation}
  \frac{x^2}{8} \bigg\rvert_0^4
  = \frac{16}{8} - \frac{0}{8} = 2.
\end{equation}

Therefore the mean value of U$\left(0,4 \right)$ is 2.

Now we turn to computing the variance of U$\left(0,4 \right)$.

In \cite{reading6a} Orloff and Bloom define the variance of a continuous
random variable $X$ with mean $\mu$ as
$E \left( \left(X - \mu \right) \right)$, where $E$ is the function for
computing the expected value of $X$.

Therefore
\begin{equation}
  \text{Var}\left( X \right) =
  \int_0^4 \frac{\left( x - \mu \right)^2}{4} \, dx
\end{equation}

Note:

\begin{equation}
\left( x - \mu \right)^2 = x^2 - 2x\mu + \mu^2.
\end{equation}

Therefore:

\begin{equation}
  \int_0^4 \frac{\left( x - \mu \right)^2}{4} \, dx
  =
  \int_0^4 \frac{x^2 - 2x\mu + \mu^2}{4} \, dx
\end{equation}

Now we can use theorem 6.1.1 from \cite{basicInt} to obtain:

\begin{equation}
  \int_0^4 \frac{x^2 - 2x\mu + \mu^2}{4} \, dx
  = \frac{1}{4} \left( \int_0^4 x^2\,dx - 2\int_0^4 \mu x \,dx
  +\int_0^4 \mu^2 \,dx \right).
\end{equation}

Next we apply the power rule for integrals \cite{basicInt} to the equation
above to get:

\begin{equation}
   \frac{1}{4} \left( \int_0^4 x^2\,dx - 2\int_0^4 \mu x \,dx
  +\int_0^4 \mu^2 \,dx \right) =
  \frac{1}{4} \left( \frac{x^3}{3} \bigg\rvert_0^4 - \mu x^2 \bigg\rvert_0^4
    + \mu^2 x \bigg\rvert_0^4 \right).
\end{equation}

We know from our work above that in the context of this problem, $\mu=2$, so:


\begin{equation}
  \frac{1}{4} \left( \frac{x^2}{2} \bigg\rvert_0^4 - \mu x^2 \bigg\rvert_0^4
    + \mu^2 x \bigg\rvert_0^4 \right) =
    \frac{1}{4} \left( \frac{x^3}{3} \bigg\rvert_0^4 - 2 x^2 \bigg\rvert_0^4
    + 4 x \bigg\rvert_0^4 \right).
\end{equation}

Now we can evaluate the anti-derivatives above:
\begin{equation}
    \frac{1}{4} \left( \frac{x^3}{3} \bigg\rvert_0^4 - 2 x^2 \bigg\rvert_0^4
    + 4 x \bigg\rvert_0^4 \right) =
    \frac{1}{4} \left( \frac{64}{3} - 32 + 16 \right)
    = \frac{1}{4} \left( \frac{64}{3} - \frac{96}{3} + \frac{48}{3} \right)
    = \frac{1}{4} \left(\frac{16}{3})
    =\frac{16}{12} \approx 1.333
\end{equation}


\printbibliography{}
\end{document}
