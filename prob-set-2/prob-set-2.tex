\documentclass[a4paper,11pt]{article}

\usepackage{amsmath}
\usepackage{amssymb}

% for proofs  environment
\usepackage{amsthm}

\usepackage[backend=bibtex]{biblatex}
\bibliography{prob-set-2}

% for probability trees
\usepackage{tikz}
\usetikzlibrary{trees}

%for strikethrough text
\usepackage{soul}

%for R source code listing
\usepackage{listings}

%for block quotes
\usepackage{csquotes}

% For not indenting the first line of paragraphs:
\setlength{\parindent}{0pt}
% define the title
\author{John Hancock}
\title{MIT Introduction to Statistics 18.05 Problem Set 2 }
\begin{document}
% generates the title
\maketitle
% insert the table of contents
\tableofcontents
\section{References and License}
We are answering questions in the material from MIT OpenCourseWare
course 18.05, Introduction to Probability and Statistics.

Please see the references section for detailed citation information.

The material for the course is licensed under the terms at 
\url{http://ocw.mit.edu/terms}.

We are answering the questions that Orloff and Bloom ask in
\cite{probSet2}.

We use documentation in \cite{latexSpecialChars} \cite{latexCases} to 
write \LaTeX source scode for this document.

\section{`Boy or girl' paradox}

In order to write this solution, we rely on the answer to this problem
in \cite{probSet2Ans}, and the treatment of the 'Boy or girl,' paradox
in \cite{boyGirlWiki}.

For these questions on the `Boy or girl paradox we deal with events
$B$, ``the child is a boy,'' and $G$, ``the child is a girl.''

We assume $B$, and $G$ have the same properties as the $B$ and $G$
events Orloff and Bloom analyze in example 9 of \cite{reading4}.
These properties are that $B$, and $G$ are independent, and they have
probability $\frac{1}{2}$.

We use these properties to define 4 more events, $BB$, $BG$, $GB$,
and $GG$.  These events are: ``the younger child is a boy, and the older
child is a boy,'' ``the younger child is a boy, and the older child is a
girl,'' ``the younger child is a girl, and the older child is a boy,''
``the younger child is a girl, and the older child is a girl,'' 
respectively.  We use the properties of $B$, and $G$, of example 9 to
compute that the probabilities of $BB$, $BG$, $GB$, $GG$, 
$P\left( BB \right)$, $P\left( BG \right)$, $P\left( GB \right)$, 
$P\left( GG \right)$, are all equal to $\frac{1}{4}$.

\subsection{Probability of girls}
The question Orloff and bloom quote is, ``Mr. Jones has two children. The
older child is a girl. What is the probability that both children are
girls?''

We can restate the question above as,  ``Given event $BG$
or event $GG$, what is the probability $GG$?'' 

We use the definition of conditional probability. We also use the law of
total probability to compute $P \left( GG \cup BG \right)$.

Therefore we write the equation:

\begin{equation}
P \left( GG \mid GG \cup BG \right) \\
 = \frac{ P \left( GG \cap \left( GG \cup BG \right) \right)}
  {P \left( GG \cup BG \right)}
\end{equation}

\begin{equation}
\frac{ P \left( GG \cap \left( GG \cup BG \right) \right)}
  {P \left( GG \cup BG \right)}
 = \frac{ P \left( GG \right) }
  {P \left( GG \cup BG \right)}
\end{equation}

\begin{equation}
 \frac{ P \left( GG \right) }
  {P \left( GG \cup BG \right)}
= \frac { \frac{1}{4} }
{ \frac{1}{2} }
\end{equation}

\begin{equation}
\frac { \frac{1}{4} }
{ \frac{1}{2} }
= \left( \frac{1}{4} \right) \left( \frac{2}{1} \right)
=\frac{1}{2}
\end{equation}

Therefore if Mr. Jones' older child is a girl, there is a 
probability of $\frac{1}{2}$ that the younger child is also a girl.


\subsection{Probability of boys}

In this section, Orloff and Bloom quote another question for us to 
answer here.

The question is, ``Mr. Smith has two children. At least one of them is a
boy. What is the probability that both children are boys?''

We use the definitions from the previous section for events, $BB$, $BG$,
$GB$ and $GG$. We use the probabilities we found in the first section
for these events as well.

The  author of this question is giving us that three possible events
have occurred: $BB$, $BG$, or $GB$.  Furthermore the question asks
for the conditional probability of $BB$.

We use the definition of conditional probability, and the law of total
probability to compute:

\begin{equation}
P \left( BB \mid BB \cup BG \cup GB \right) 
= \frac{ P \left( BB \cap \left( BB \cup BG \cup GB \right) \right) }
  { P \left( BB \cup BG \cup GB \right) }
\end{equation}

\begin{equation}
\frac{ P \left( BB \cap \left( BB \cup BG \cup GB \right) \right) }
  { P \left( BB \cup BG \cup GB \right) }
= \frac{ P \left( BB \right)}
  { P \left( BB \cup BG \cup GB \right) }
\end{equation}

\begin{equation}
 \frac{ P \left( BB \right)}
  { P \left( BB \cup BG \cup GB \right) }
= \frac{ \frac{ 1}{4} }
  { \frac{1}{4} + \frac{1}{4} + \frac{1}{4}}
= \left( \frac{1}{4} \right) \left( \frac{4}{3} \right)
= \frac{1}{3}
\end{equation}

If at least one of Mr. Smith's children is a boy, then there is a 
probability of $\frac{1}{3}$ that both children are boys.

\section{The blue taxi}

We define the following sets:
  \begin{itemize}
  \item $D+$, ``The car is blue.''
  \item $D-$, ``The car is green.''
  \item $T+$, ``The witness reports seing a blue car.''
  \item $T-$, ``The witness reports seing a green car.''
\end{itemize}

Orloff and Bloom give us the following probabilities:

\begin{itemize}
  \item $P \left( D+ \right) = 0.01$
  \item $P \left( D- \right) = 0.99$
  \item $P \left( T+ \mid D+ \right) = 0.99$
  \item $P \left( T+ \mid D- \right) = 0.02$
\end{itemize}

In order to make our case, we need to know 
$P \left( D+ \mid T+ \right)$. That is the probability that, given a 
blue car, the witness saw a blue car.  

This table summarizes the information we know.  Note the small 
ratio of blue taxis to all taxis: $\frac{1}{100}$.

\begin{center}
  \begin{tabular}{ | c | c | c |  }
    \hline
         & Green & Blue     \\ \hline
    Sees Blue & $P\left(T+ \mid D- \right) = 0.02$ & $P \left( T+ \mid D+ \right)=0.99$   \\ \hline
    Total & $P \left( D- \right) = 0.99$ &  $P \left( D+ \right) = 0.01$   \\ \hline
  \end{tabular}
\end{center}

We apply Bayes' theorem \cite{reading3} to $P \left(T+ \mid D+ \right)$
in order to compute $P \left( D+ \mid T+ \right)$.

\begin{equation} \label{blueCarBayes}
P \left( D+ \mid T+ \right)
  = \frac{ P \left( T+ \mid D+ \right) P \left( D+ \right) }
    { P \left( T+ \right) }
\end{equation}

We use the Law of total probability to rewrite the denominator of the
fraction on the right hand side of \ref{blueCarBayes}

\begin{equation}
\frac{ P \left( T+ \mid D+ \right) P \left( D+ \right) }
    { P \left( T+ \right) }
  = 
\frac{ P \left( T+ \mid D+ \right) P \left( D+ \right) }
    { P \left( T+ \cap D+ \right) + P \left( T+ \cap D- \right) }
\end{equation}

We now use the definition of conditional probability to rewrite the
probabilities in the denominator of the equation in the right hand side
of the equation above:

\begin{equation}
\frac{ P \left( T+ \mid D+ \right) P \left( D+ \right) }
    { P \left( T+ \right) }
  = 
\frac{ P \left( T+ \mid D+ \right) P \left( D+ \right) }
    { P \left( T+ \mid D+ \right) P \left( D+ \right) 
      + P \left( T+ \mid D- \right) P \left(D- \right)}
\end{equation}

The terms of the right hand side of the equation above are all in our
table, so we now have a way to compute $P \left( D+ \mid T+ \right)$:

\begin{equation}
\frac{ P \left( T+ \mid D+ \right) P \left( D+ \right) }
    { P \left( T+ \mid D+ \right) P \left( D+ \right) 
      + P \left( T+ \mid D- \right) P \left(D- \right)}
  = \frac{ 0.99 \times 0.01 }
  { 0.99 \times 0.01 + 0.02 \times 0.99 }
\end{equation}

Now we simplify the right hand side of the equation above to arrive at
a value for $P \left( D+ \mid T+ \right)$.
\begin{equation}
 \frac{ 0.99 \times 0.01 }
  {  \left(0.01 + 0.02 \right) \times 0.99 }
  = \frac{ 0.99 \times 0.01 }
  {  0.03 \times 0.99 }
  = \frac{1}{3}
\end{equation}

Therefore there is a $\frac{1}{3}$ that given a blue taxi, the witness
sees a blue taxi.  This is a less than $50\%$ chance that the witness
actually saw a blue taxi. Hence we have a reasonable doubt that the 
witness saw a blue taxi.

\section{Trees of cards}
\label{treesOfCards}
In this section we answer Orloff and Bloom's question in \cite{probSet2}
about the expected value of a random variable.

The random variable is the value of the sum of cards we draw from a 
hat.  

We refer to a card of rank one as an ace, and a card of rank two as
a two.

There are four aces, and four two's in the hat.

Rules govern the way we draw the cards from the hat.  The rules are: 
\begin{itemize}
\item If we draw an ace first, then we draw one more card. 
\item If we draw a two first, then we draw two more cards.
\end{itemize}

We assign one to the value of the ace, and two to the value of the two.

We assign the random variable $X$ the value of the sum of the values
of the cards we draw.

% Set the overall layout of the tree
\tikzstyle{level 1}=[level distance=1cm, sibling distance=7cm]
\tikzstyle{level 2}=[level distance=2cm, sibling distance=3cm]
\tikzstyle{level 3}=[level distance=2cm, sibling distance=1.5cm]

% Define styles for bags and leafs
\tikzstyle{bag} = [text width=4em, text centered]
\tikzstyle{end} = [circle, minimum width=3pt,fill, inner sep=0pt]

\begin{center}
\begin{tikzpicture}[grow=down, sloped]
\node[end] {}
    child {
        node[bag] {Draw Ace}        
           child {
                node[bag] {Draw Ace $X=2$}
                edge from parent node[above]  {$\frac{3}{7}$}
            }
            child {
                node[bag] {Draw Two $X=3$}
                edge from parent node[above]  {$\frac{4}{7}$}
            }
            edge from parent node[above]  {$\frac{4}{8}$}
    }
    child {
        node[bag] {Draw Two}        
        child {
          node[bag] {Draw Ace}
          child{
            node[bag] {Draw Ace $X=4$}
            edge from parent node[above] {$\frac{3}{6}$}
          }
          child{
            node[bag] {Draw Two $X=5$}
            edge from parent node[above] {$\frac{3}{6}$}
          }
          edge from parent node[above]  {$\frac{4}{7}$}
        }
        child {
          node[bag] {Draw Two}
          child{
            node[bag] {Draw Ace $X=5$}
            edge from parent node[above] {$\frac{4}{6}$}
          }
          child{
            node[bag] {Draw Two $X=6$}
            edge from parent node[above] {$\frac{2}{6}$}
          }
          edge from parent node[above]  {$\frac{3}{7}$}
         }
        edge from parent node[above]  {$\frac{4}{8}$}
    };
\end{tikzpicture}
\end{center}

We supplement terms we use when we write about trees with definitions 
of leaf, root from \cite{treeSlides}.  We also use definitions of 
vertex, node, edge, and path in \cite{graphSlides} when we write about
trees.

Recall from \cite{reading3} that the fractions on the edges of the
probability trees like the tree above are probabilities, and that the
nodes of the tree are events.  Nodes that are not directly connected to 
the root of the tree are unions of events.  Furthermore,  we learn in 
\cite{reading3} that the probabilities on the edges of probability trees 
that are not connected to the root of the tree are conditional 
probabilities, where the condition given is  the event that the node 
connected to the edge that is closer to the root of the tree represents. 

\cite{reading3} also allows us to multiply probabilities on paths from
the root of the probability tree to the leaves of a probability tree
to compute the probability of the events that are the leaf nodes of the
tree.

Now we are armed with the facts that enable us to compute the expected
value $E\left( X \right)$.

We use the definition of expected value in \cite{reading4b}:
\begin{equation}\label{expectedValCards}
\begin{split}
  E\left( X \right) = 
    2\left(\frac{4}{8} \right) \left( \frac{3}{7} \right)
  + 3 \left( \frac{4}{8} \right) \left( \frac{4}{7} \right)
  + 4 \left( \frac{4}{8} \right) \left( \frac{4}{7} \right) 
    \left( \frac{3}{6} \right) \\
  + 5 \left( \frac{4}{8} \right) \left( \frac{4}{7} \right)
    \left( \frac{3}{6} \right)
  + 5 \left( \frac{4}{8} \right) \left( \frac{3}{7} \right)
    \left( \frac{4}{6} \right)
  + 6 \left( \frac{4}{8} \right) \left( \frac{3}{7} \right)
    \left( \frac{2}{6} \right)
\end{split}
\end{equation}

We write a statement in the R programming language to calculate 
the value of the right hand side of equation \ref{expectedValCards}:

\begin{lstlisting}
> 2*(4/8)*(3/7) + 3*(4/8)*(4/7) + 4*(4/8)*(4/7)*(3/6) 
+ 5*(4/8)*(4/7)*(3/6) + 5*(4/8)*(3/7)*(4/6) + 6*(4/8)*(3/7)*(2/6)
[1] 3.714286
\end{lstlisting}

Note: we inserted a line break in the R statement above in order to
fit it onto the page; the reader will need to remove the line break
in order to reproduce the result.

Therefore $E \left( X \right) \approx 3.714$.

\section{Dice}
In this section, we answer problem for that Orloff and Bloom pose in
\cite{probSet2}.

\subsection{Probability Mass Function}

The first query Orloff and Bloom make is, "What is the pmf of $S$."

We find the definition of pmf in \cite{reading4b}.  Orloff and Bloom
give the definition of the random variable $S$ in \cite{probSet2}.
Please see \cite{reading4} for the definition of a random variable.

We use a table to give the pmf of $S$.

\begin{center}
  \begin{tabular}{ | c | c | c | c| }
    \hline
    k & 4 & 6 & 8    \\ \hline
    pmf $P\left( S=k \right)$ & $\frac{1}{4}$ & $\frac{1}{4}$ 
      & $\frac{1}{2}$ \\ \hline
  \end{tabular}
\end{center}

\subsection{Apply Bayes' Rule}
In this part of the question, Orloff and Bloom ask us to find the
conditional probabilities $P \left( S = k \mid R = 3\right)$, for
$k=4$, $k=6$, and $k=8$.

Orloff and Bloom also give a terminology note for this problem, 
writing, ``You are computing the pmf of ‘S given R = 3’.''
\cite{probSet2}

We will apply Bayes rule to compute $P \left( S = k \mid R = 3\right)$ 
because it is easy to compute $P \left( R=3 \mid S=k \right)$.

$P \left( R=3 \mid S=k \right)$ is the probability that we roll a $3$ 
with a $k$-sided die.

We assume the probabilities of the values we roll with any of the dice
Orloff and Bloom define in this problem are all equally likely.

We write a table to list the values of $P \left( R=3 \mid S=k \right)$,
for $k=4$, $k=6$, and $k=8$.

\begin{center}
  \begin{tabular}{ | c | c | c | c| }
    \hline
    k & 4 & 6 & 8    \\ \hline
    $p\left( R=3 \mid S=k \right)$ & $\frac{1}{4}$ & $\frac{1}{6}$ 
      & $\frac{1}{8}$ \\ \hline
  \end{tabular}
\end{center}

Let us apply Bayes theorem \cite{reading3} to 
$P \left( S=k \mid R=3 \right)$:

\begin{equation} \label{bayesApplyDice}
P \left( S=k \mid R=3 \right) =
  \frac{ P \left( R=3 \mid S=k \right) P \left( S=k \right)}
    { P \left( R=3 \right)}
\end{equation}

Equation \ref{bayesApplyDice} gives us a formula that we can apply
to the second row of the elements of table above to compute the
values of $P \left( S = k \mid R = 3\right)$, for
$k=4$, $k=6$, and $k=8$. We give these formulas in the table below:

\begin{center}
  \begin{tabular}{ | c | c | c | c| }
    \hline
    k & 4 & 6 & 8    \\ \hline
    $p\left( R=3 \mid S=k \right)$ & $\frac{1}{4}$ & $\frac{1}{6}$ 
      & $\frac{1}{8}$ \\ \hline
    $p\left( S=k \mid R=3 \right)$
     & $\frac{ P \left( R=3 \mid S=4 \right) P \left( S=4 \right)}
      { P \left( R=3 \right)}$
     & $\frac{ P \left( R=3 \mid S=6 \right) P \left( S=6 \right)}
      { P \left( R=3 \right)}$ 
    & $\frac{ P \left( R=3 \mid S=8 \right) P \left( S=8 \right)}
      { P \left( R=3 \right)}$ \\ \hline
  \end{tabular}
\end{center}

We are almost ready to compute $P \left( S = k \mid R = 3\right)$, for
$k=4$, $k=6$, and $k=8$, but we lack a value for 
$p \left( R =3 \right)$.

We will draw a probability tree, and use the law of total probability
to compute $P \left( R=3 \right)$.

\begin{center}
\begin{tikzpicture}[grow=down, sloped]
\node[end] {}
    child {
        node[bag] {$4$-Sided Die}        
           child {
                node[bag] {Roll 3}
                edge from parent node[above]  {$\frac{1}{4}$}
            }
            edge from parent node[above]  {$\frac{1}{4}$}
    }
    child {
        node[bag] {$6$-Sided Die}        
           child {
                node[bag] {Roll 3}
                edge from parent node[above]  {$\frac{1}{6}$}
            }
            edge from parent node[above]  {$\frac{1}{4}$}
    }
    child {
        node[bag] {$8$-Sided Die}        
           child {
                node[bag] {Roll 3}
                edge from parent node[above]  {$\frac{1}{8}$}
            }
            edge from parent node[above]  {$\frac{1}{2}$}
    };
\end{tikzpicture}
\end{center}

The probabilities on the edges connected to the root of the probability
tree above are from the probability mass function for $S$ that we
gave in the previous section.  These are the probabilities of the event
of selecting a $4$, $6$, or $8$ sided die.

The probabilities on the edges connected to the leaf nodes are the 
conditional probabilities we computed for 
$P \left( R=3 \mid S=k \right)$ in the table above.

Please note that we use the same definitions and operations on the 
elements of probability trees here that we use in section 
\ref{treesOfCards}.

The probability of the event of any one of the leaf nodes in the tree
above is the product of the probabilities that label the edges on the
path to one of the leaf nodes.

Hence, we apply the law of total probability to compute the total probability
of the events that are the leaf nodes of the tree above:

\begin{equation}
P \left( R=3 \right) = 
  \left( \frac{1}{4} \right) \left( \frac{1}{4} \right) +
  \left( \frac{1}{4} \right) \left( \frac{1}{6} \right) +
  \left( \frac{1}{2} \right) \left( \frac{1}{8} \right)
\end{equation}

In order to complete the calculation, we use the common denominator
of 48:

\begin{equation} 
  \left( \frac{1}{4} \right) \left( \frac{1}{4} \right) +
  \left( \frac{1}{4} \right) \left( \frac{1}{6} \right) +
  \left( \frac{1}{2} \right) \left( \frac{1}{8} \right)
= 
  \left( \frac{3}{48} \right) +
  \left( \frac{2}{48} \right) +
  \left( \frac{3}{48} \right)
\end{equation}

We add the fractions, and reduce to lowest terms:
\begin{equation} 
  \left( \frac{3}{48} \right) +
  \left( \frac{2}{48} \right) +
  \left( \frac{3}{48} \right)
= \frac{8}{48} = \frac{1}{6}
\end{equation}

Therefore $P \left(R = 3 \right) = \frac{1}{6}$

Now we can compute the values of $P \left( S = k \mid R = 3\right)$, for
$k=4$, $k=6$, and $k=8$. We give these values in the table below.
Please see our comments on the probability tree above for information on
how we know $P\left(S=k \right)$ for $k=4$, $k=6$, and $k=8$.

\begin{center}
  \begin{tabular}{ | c | c | c | c| }
    \hline
    k & 4 & 6 & 8    \\ \hline
    $p\left( R=3 \mid S=k \right)$ & $\frac{1}{4}$ & $\frac{1}{6}$ 
      & $\frac{1}{8}$ \\ \hline
    $p\left( S=k \mid R=3 \right)$
     & $\frac{ P \left( R=3 \mid S=4 \right) P \left( S=4 \right)}
      { P \left( R=3 \right)}$
     & $\frac{ P \left( R=3 \mid S=6 \right) P \left( S=6 \right)}
      { P \left( R=3 \right)}$ 
    & $\frac{ P \left( R=3 \mid S=8 \right) P \left( S=8 \right)}
      { P \left( R=3 \right)}$ \\ \hline
    $p\left( S=k \mid R=3 \right)$
     & $\frac{ \left( \frac{1}{4} \right) \left( \frac{1}{4} \right)}
          { \frac{1}{6} }$
     & $\frac{ \left( \frac{1}{6} \right) \left( \frac{1}{4} \right)}
          { \frac{1}{6} }$
     & $\frac{ \left( \frac{1}{8} \right) \left( \frac{1}{2} \right)}
          { \frac{1}{6} }$ \\ \hline
    simplify above & $\frac{3}{8}$ & $\frac{1}{4}$ 
      & $\frac{3}{8}$
      \\ \hline
  \end{tabular}
\end{center}

The last row of the table above gives the values for 
$P \left( S = k \mid R = 3\right)$, for
$k=4$, $k=6$, and $k=8$ that Orloff and Bloom require as a solution to
this problem.

In this problem, Orloff and Bloom ask a follow up question, 
``Which die is most likely if $R = 3$?''.  The largest value in the last
row in the table above is $\frac{3}{8}$, so the $4$-sided and the
$8$-sided die are equally likely if $R = 3$.

\subsection{Rolling a six}

This section is similar to the previous section, except that here
Orloff and Bloom are asking us, ``Which die is most likely if R = 6?''
\cite{probSet2}

This question is equivalent to: what is the maximum value of
$P \left( S=k \mid R = 6 \right)$ for values of $k=4$, $k=6$, and
$k=8$.

We use the probability mass function $P \left( S=k \right)$ for the 
random variable $S$ from the previous section.  

We draw a probability tree for $P \left( R=6 \right)$

\begin{center}
\begin{tikzpicture}[grow=down, sloped]
\node[end] {}
    child {
        node[bag] {$4$-Sided Die}        
            edge from parent node[above]  {$0$}
    }
    child {
        node[bag] {$6$-Sided Die}        
           child {
                node[bag] {Roll 6}
                edge from parent node[above]  {$\frac{1}{6}$}
            }
            edge from parent node[above]  {$\frac{1}{3}$}
    }
    child {
        node[bag] {$8$-Sided Die}        
           child {
                node[bag] {Roll 6}
                edge from parent node[above]  {$\frac{1}{8}$}
            }
            edge from parent node[above]  {$\frac{2}{3}$}
    };
\end{tikzpicture}
\end{center}

We use the rules governing probability trees from \cite{reading3} to
compute $P \left( R=6 \right)$:
\begin{equation} \label{totalProb6}
  P \left( R=6 \right) = 
    \left( \frac{1}{3} \right) \left( \frac{1}{6} \right)
  + \left(  \frac{2}{3} \right) \left( \frac{1}{8} \right)
\end{equation}

We can do some arithmetic to simplify the right hand side of equation
\ref{totalProb6}:

\begin{equation} 
    \left( \frac{1}{3} \right) \left( \frac{1}{6} \right)
  + \left(  \frac{2}{3} \right) \left( \frac{1}{8} \right)
  = \frac{1}{3} \left( \frac{1}{6} + \frac{2}{8} \right)
  = \frac{1}{3} \left( \frac{4}{24} + \frac{6}{24} \right)
  = \frac{10}{72} = \frac{5}{36}.
\end{equation}

Now we have all the probabilities necessary to fill out a table similar
to the last table in the previous section to determine the conditional 
probabilities $P \left( S=k \mid R=6 \right)$ for values of $k=4$, 
$k=6$, and $k=8$:

\begin{center}
  \begin{tabular}{ | c | c | c | c| }
    \hline
    k & 4 & 6 & 8    \\ \hline
    $p\left( R=6 \mid S=k \right)$ & $0$ & $\frac{1}{6}$ 
      & $\frac{1}{8}$ \\ \hline
    $p\left( S=k \mid R=6 \right)$
     & Bayes' Theorem Does Not Apply
     & $\frac{ P \left( R=6 \mid S=6 \right) P \left( S=6 \right)}
      { P \left( R=6 \right)}$ 
    & $\frac{ P \left( R=6 \mid S=8 \right) P \left( S=8 \right)}
      { P \left( R=6 \right)}$ \\ \hline
    $p\left( S=k \mid R=6 \right)$
     & Bayes' Theorem Does Not Apply'
     & $\frac{ \left( \frac{1}{6} \right) \left( \frac{1}{3} \right)}
          { \frac{5}{36} }$
     & $\frac{ \left( \frac{1}{8} \right) \left( \frac{2}{3} \right)}
          { \frac{5}{36} }$ \\ \hline
    simplify above & Bayes Theorem Does Not Apply & $\frac{2}{5}$ 
      & $\frac{3}{5}$
      \\ \hline
  \end{tabular}
\end{center}

In the table above, we see that the maximum value of 
$P \left( R=6 \mid S=k \right)$ for values of $k$, $k=4$, $k=6$, and
$k=8$, is $P \left( R=6 \mid S=8 \right) = \frac{3}{5}$.

\subsection{Rolling a seven}
The dice that is most likely if we roll a $7$ is one of the octahedral
dice because we cannot roll that value with the hexahedral or
tetrahedral dice.

\section{Seating arrangements and relative height}
Orloff and Bloom in this problem in \cite{probSet2} ask us to consider 
what would be the expected value of the number of people who are shorter
than their immediate neighbors, if the people are seated randomly
around the table, no people are the same height, and any height is
equally likely.  Orloff and Bloom also write that the table is circular.

We can use the same technique Orloff and Bloom use to solve the last
problem in \cite{slides4Ans} to answer this question.

We will assign the random variable $X$ the value of the number of
people seated at the table who are shorter than both of their
immediate neighbors.

To answer this question, we must compute the expected value, 
$E\left( X \right)$.

Borrowing the technique from \cite{probSet2}, we will define a set
of random variables 
\begin{equation}
  S = \left\{ X_{1}, X_{2}, \ldots, X_{n} \right\}
\end{equation}

Where 
\[
X_{i} = 
\begin{cases}
  1 & \text{if the } i^{th} \text{ person is shorter than 
    his/her immediate neighbors} \\
  0 & \text{otherwise}
\end{cases}
\]

The definition of $X$ guarantees us that
\begin{equation}
\sum_{i=1}^{n} X_{i} = X
\end{equation}
because $X$ is the number of people seated at the table who are shorter
than both of their immediate neighbors.

Orloff and Bloom state that no two people have the same height, so we
can model the seating of one person and his or her two immediate 
neighbors as a sampling without replacement of $n$ integers.

Let the three numbers we sample to model the heights of the $i^{th}$
person, one neighbor of the $i^{th}$ person, and the other neighbor of
the $i^{th}$ person be $h_{1}$, $h_{2}$, and $h_{3}$ respectively.

We know $h_{1}$, $h_{2}$, and $h_{3}$ are all different numbers, and
that there are $3! = 6$ ways to permute  $h_{1}$, $h_{2}$, and $h_{3}$.

There are two permutations where the smallest of $h_{1}$, $h_{2}$, 
and $h_{3}$ is the middle number.

Therefore there is a probability of $\frac{2}{6} = \frac{1}{3}$ that
the $i^{th}$ person will be shorter than his or her two immediate
neighbors.  

There is a $ 1 - \frac{1}{3} = \frac{2}{3}$ probability that the
$i^{th}$ person will not be shorter than both of his or her immediate
neighbors.

We defined the random variables $X_{i}$ to have the value 1 if the
$i^{th}$ person is shorter than both of his or her immediate neighbors,
and 0 otherwise.

Therefore, by the definition of expected value,
\begin{equation}
  E\left( X_{i} \right) = 1 \left( \frac{1}{3} \right)
    + 0 \left( \frac{2}{3} \right) = \frac{1}{3}
\end{equation}

In \cite{reading4b} Orloff and Bloom show that for any two random
variables$X$, and $Y$, $E \left( X + Y \right) = E \left( X \right) +
E \left(Y \right)$.  In \cite{hannySlides4} we show that this property
of expected values of random variables holds for finite length sums
of random variables.

Therefore
\begin{equation}
 E(X) = \sum_{i=1}^{n} X_{i} = \sum_{i=1}^{n} \frac{1}{3} = \frac{n}{3}
\end{equation}

The expected number of $n$ people seated at the table who are shorter
than both of their neighbors is $\frac{n}{3}$.


\section{R simulations and runs}
In this section we answer questions Orloff and Bloom ask in problem 6
of \cite{prob2Set}.

We use the R programming language and techniques in \cite{reading4R}.
We also used a technique for R programming we found in \cite{rleHint}.

\subsection{Random Sequence of 50 flips}
The first thing Orloff and Bloom ask us to do in problem is is to
write down a random sequence $S$of 50 flips.

Here is the sequence:
\begin{equation}
\begin{split}
S= (0, 0, 1, 0, 1, 1, 0, 0, 0, 1, 1, 1, 0, 1, 0, 1, 0, 0, 1, 1, 0, 0, \\
1, 1, 0, 1, 1, 1, 1, 1, 0, 1, 0, 0, 1, 1, 0, 0, 1, 0, 0, 0, 0, 0, 0, \\
0, 0, 1, 0, 1)
\end{split}
\end{equation}

\subsection{Longest Run}

Next, Orloff and Bloom ask us to find the length of the longest run
in $S$.  The longest runs is the subsequence of 8 0's that ends at the
fourth-to-last position of $S$.

\subsection{R simulation}

Orloff and Bloom give us the R code to simulate a trial of flipping
a fair coin 50 times.  They also supply the code for finding the
length of the longest run in the simulation.

It is our task to simulate $10,000$ trials of this experiment with a for
loop, and compute the average length of the longest run in the trials.

The R code below implements this task:

\begin{lstlisting}
trial <- function(){
  nflips=50;
  trial  = rbinom(nflips, 1, 0.5);
  maxRun = max(rle(trial)$lengths);
  return ()
}

experiment <- function(){
  sum = 0;
  for (i in 1 : 10000){
  sum = sum + trial();
  }
  return(sum/10000);
}
\end{lstlisting}

We run the experiment function and get the result 5.9906.

\subsection{Modify to compute probability of runs}

In this part, Orloff and Bloom ask us to make a small modification
to the R program above so that we can estimate the probability of
a run of length 8 in 50 or more flips.

Here is the modified code:

\begin{lstlisting}
trial <- function(){
  nflips=50;
  trial  = rbinom(nflips, 1, 0.5);
  maxRun = max(rle(trial)$lengths);
  return (sum(maxRun >= 8))
}
experiment <- function(){
  sum = 0;
  
  numTrials = 10000;
  for (i in 1 : numTrials){
  sum = sum + trial();
  }
  return(sum/numTrials);
}
experiment();
\end{lstlisting}

The output we get for this program is 0.1603.  Therefore we estimate
the probability of a run of length 8 in a 50 trials of a binomial random
variable with probability 0.5 to be 0.1603.

\printbibliography{}
\end{document}
