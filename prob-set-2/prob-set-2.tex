\documentclass[a4paper,11pt]{article}

\usepackage{amsmath}
\usepackage{amssymb}

% for proofs  environment
\usepackage{amsthm}

\usepackage[backend=bibtex]{biblatex}
\bibliography{prob-set-2}

% for probability trees
\usepackage{tikz}
\usetikzlibrary{trees}

%for strikethrough text
\usepackage{soul}

%for R source code listing
\usepackage{listings}

%for block quotes
\usepackage{csquotes}

% For not indenting the first line of paragraphs:
\setlength{\parindent}{0pt}
% define the title
\author{John Hancock}
\title{MIT Introduction to Statistics 18.05 Problem Set 2 }
\begin{document}
% generates the title
\maketitle
% insert the table of contents
\tableofcontents
\section{References and License}
We are answering questions in the material from MIT OpenCourseWare
course 18.05, Introduction to Probability and Statistics.

Please see the references section for detailed citation information.

The material for the course is licensed under the terms at 
\url{http://ocw.mit.edu/terms}.

We are answering the questions that Orloff and Bloom ask in
\cite{probSet2}.

We use documentation in  to write \LaTeX source code
for this document.

\section{Boy or girl paradox}
\subsection{Probability of girls}
The first question Orloff and Bloom pose in \cite{probSet2} is
on the probability of two similar situations.

The first situation is one where we are given that a man has
two children, one of which is a girl.

We are asked to calculate the probability that both children
are girls, given that one of the children is a girl.

We assume that the probability of a child being a girl is $\frac{1}{2}$,
and that the events of a child being a girl or a boy are independent.
Intuitively this is acceptable since knowledge that a woman gives birth
to a girl has no influence over the sex of the next baby she will give
birth to.

We represent the event, "the child is a girl," with a $1$, and
the event, "the child is a boy," with a $0$.

We represent the sequence of events such as, "the first child is  a boy, and the 
second child is a girl," as $\left( 0, 1 \right)$. 

We define a sample space, $\Omega$, that consists of sequences of these event.

The sample space of $\Omega$ events is small enough that we can write it out.

We can define $\Omega$
\begin{equation}
\Omega = \left\{ left( x_{1}, x_{2} \right) \mid x_{1}, x_{2} \in \left\{ 0 , 1 \right\} \right\}
  = \left\{ \left(0,0\right), \left(0,1\right), \left(1, 0 \right), \left(1, 1\right) \right\}
\end{equation}

Now, we are given that one of the children is a girl,
so the only outcomes in $Omega$ that meet this condition
are $left( 1,1 \right)$, and $\left( 1, 0 \right)$.

Each element of $\Omega$ has probability $\frac{1}{2}$.

These two outcomes constitute half of the sample space of $\Omega$,
and they are equally likely.


\subsection{Probability of boys}
\printbibliography{}
\end{document}
