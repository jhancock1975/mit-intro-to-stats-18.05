\documentclass[a4paper,11pt]{article}

\usepackage{amsmath}
\usepackage{amssymb}

% for proofs  environment
\usepackage{amsthm}

\usepackage[backend=bibtex]{biblatex}
\bibliography{prob-set-2}

% for probability trees
\usepackage{tikz}
\usetikzlibrary{trees}

%for strikethrough text
\usepackage{soul}

%for R source code listing
\usepackage{listings}

%for block quotes
\usepackage{csquotes}

% For not indenting the first line of paragraphs:
\setlength{\parindent}{0pt}
% define the title
\author{John Hancock}
\title{MIT Introduction to Statistics 18.05 Problem Set 2 }
\begin{document}
% generates the title
\maketitle
% insert the table of contents
\tableofcontents
\section{References and License}
We are answering questions in the material from MIT OpenCourseWare
course 18.05, Introduction to Probability and Statistics.

Please see the references section for detailed citation information.

The material for the course is licensed under the terms at 
\url{http://ocw.mit.edu/terms}.

We are answering the questions that Orloff and Bloom ask in
\cite{probSet2}.

We use documentation in  to write \LaTeX source code
for this document.

\section{`Boy or girl' paradox}

In order to write this solution, we rely on the answer to this problem
in \cite{probSet2Ans}, and the treatment of the 'Boy or girl,' paradox
in \cite{boyGirlWiki}.

For these questions on the `Boy or girl paradox we deal with events
$B$, ``the child is a boy,'' and $G$, ``the child is a girl.''

We assume $B$, and $G$ have the same properties as the $B$ and $G$
events Orloff and Bloom analize in example 9 of \cite{reading4}.
These properties are that $B$, and $G$ are independent, and they have
probability $\frac{1}{2}$.

We use these properties to define 4 more events, $BB$, $BG$, $GB$,
and $GG$.  These events are: ``the younger child is a boy, and the older
child is a boy,'' ``the younger child is a boy, and the older child is a
girl,'' ``the younger child is a girl, and the older child is a boy,''
``the younger child is a girl, and the older child is a girl,'' 
respectively.  We use the properties of $B$, and $G$, of example 9 to
compute that the probabilities of $BB$, $BG$, $GB$, $GG$, 
$P\left( BB \right)$, $P\left( BG \right)$, $P\left( GB \right)$, 
$P\left( GG \right)$, are all equal to $\frac{1}{4}$.

\subsection{Probability of girls}
The question Orloff and bloom quote is, ``Mr. Jones has two children. The
older child is a girl. What is the probability that both children are
girls?''

We can restate the question above as,  ``Given event $BG$
or event $GG$, what is the probability $GG$?'' 

We use the definition of conditional probability. We also use the law of
total probability to comupte $P \left( GG \cup BG \right)$.

Therefore we write the equation:

\begin{equation}
P \left( GG \mid GG \cup BG \right) \\
 = \frac{ P \left( GG \cap \left( GG \cup BG \right) \right)}
  {P \left( GG \cup BG \right)}
\end{equation}

\begin{equation}
\frac{ P \left( GG \cap \left( GG \cup BG \right) \right)}
  {P \left( GG \cup BG \right)}
 = \frac{ P \left( GG \right) }
  {P \left( GG \cup BG \right)}
\end{equation}

\begin{equation}
 \frac{ P \left( GG \right) }
  {P \left( GG \cup BG \right)}
= \frac { \frac{1}{4} }
{ \frac{1}{2} }
\end{equation}

\begin{equation}
\frac { \frac{1}{4} }
{ \frac{1}{2} }
= \left( \frac{1}{4} \right) \left( \frac{2}{1} \right)
=\frac{1}{2}
\end{equation}

Therefore if Mr. Jones' older child is a girl, there is a 
probabiility of $\frac{1}{2}$ that the younger child is also a girl.


\subsection{Probability of boys}

In this section, Orloff and Bloom quote another question for us to 
answer here.

The question is, ``Mr. Smith has two children. At least one of them is a
boy. What is the probability that both children are boys?''

We use the definitions from the previous section for events, $BB$, $BG$,
$GB$ and $GG$. We use the probabilities we found in the first section
for these events as well.

The  author of this question is giving us that three possible events
have occured: $BB$, $BG$, or $GB$.  Furthermore the question asks
for the conditional probability of $BB$.

We use the definition of conditional probability, and the law of total
probability to compute:

\begin{equation}
P \left( BB \mid BB \cup BG \cup GB \right) 
= \frac{ P \left( BB \cap \left( BB \cup BG \cup GB \right) \right) }
  { P \left( BB \cup BG \cup GB \right) }
\end{equation}

\begin{equation}
\frac{ P \left( BB \cap \left( BB \cup BG \cup GB \right) \right) }
  { P \left( BB \cup BG \cup GB \right) }
= \frac{ P \left( BB \right)}
  { P \left( BB \cup BG \cup GB \right) }
\end{equation}

\begin{equation}
 \frac{ P \left( BB \right)}
  { P \left( BB \cup BG \cup GB \right) }
= \frac{ \frac{ 1}{4} }
  { \frac{1}{4} + \frac{1}{4} + \frac{1}{4}}
= \left( \frac{1}{4} \right) \left( \frac{4}{3} \right)
= \frac{1}{3}
\end{equation}

If at least one of Mr. Smith's children is a boy, then there is a 
probability of $\frac{1}{3}$ that both children are boys.

\section{The blue taxi}

In order to solve this problem we will write a confusion matrix
\cite{confusionMatrix}.  This is the term we found for the kind of 
table Orloff and Bloom write in \cite{reading3} that organizes false
positive rates, false negative rates, etc. Into a table.

We define the following sets:
  \begin{itemize}
  \item $D+$, ``The car is blue.''
  \item $D-$, ``The car is green.''
  \item $T+$, ``The witness reports seing a blue car.''
  \item $T-$, ``The witness reports seing a green car.''
\end{itemize}

Orloff and Bloom give us the following sizes of the probabilities:

\begin{itemize}
  \item $P \left( D+ \right) = 0.01$
  \item $P \left( D- \right) = 0.99$
  \item $P \left( T+ \mid D+ \right) = 0.99$
  \item $P \left( T+ \mid D- \right) = 0.02$
\end{itemize}

In order to make our case, we need to know $P \left( D+ \mid T+ \right)$.

We are not given $P \left( D+ \mid T+ \right)$.  However, we can apply
Bayes' theorem to $P \left(T+ \mid D+ \right)$ in order to compute
$P \left( D+ \mid T+ \right)$.

We are not given $P \left(T+ \mid D+ \right)$ either, but we can
compute this value with the aid of a confusion matrix. Ironically the
confusion matrix lends us understanding.

We write several versions of the confusion matrix, first with only
symbolic values, fill in what we know. Then we start compute values
we do not know in order to complete a version of the confusion matrix
with numeric values.

\begin{center}
  \begin{tabular}{ | c | c | c | c | }
    \hline
         & $D+$ & $D-$ & Total   \\ \hline
    $T+$ & $P \left(T+ \mid D+ \right)$ & $P \left( T+ \mid D- \right) $ & $P \left( T+ \right)$ \\ \hline
    $T-$ & $P \left(T- \mid D+ \right)$ & $P \left( T- \mid D- \right) $ & $P \left( T- \right)$ \\ \hline
    total & $P \left( D+ \right)$ & $P \left( D- \right)$ & $P\left( D+ \cup D- \right)$ \\ \hline
  \end{tabular}
\end{center}

Now we fill in the values Orloff and Bloom give us in this question:

\begin{center}
  \begin{tabular}{ | c | c | c | c | }
    \hline
         & $D+$ & $D-$ & Total   \\ \hline
    $T+$ & $P \left(T+ \mid D+ \right)$ & $P \left( T+ \mid D- \right) $ & $P \left( T+ \right)$ \\ \hline
    $T-$ & $P \left(T- \mid D+ \right)$ & $P \left( T- \mid D- \right) $ & $P \left( T- \right)$ \\ \hline
    total & 0.01 & 0.99 & $P\left( D+ \cup D- \right)$ \\ \hline
  \end{tabular}
\end{center}
\printbibliography{}
\end{document}
