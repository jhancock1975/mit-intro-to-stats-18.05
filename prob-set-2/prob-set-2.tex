\documentclass[a4paper,11pt]{article}

\usepackage{amsmath}
\usepackage{amssymb}

% for proofs  environment
\usepackage{amsthm}

\usepackage[backend=bibtex]{biblatex}
\bibliography{prob-set-2}

% for probability trees
\usepackage{tikz}
\usetikzlibrary{trees}

%for strikethrough text
\usepackage{soul}

%for R source code listing
\usepackage{listings}

%for block quotes
\usepackage{csquotes}

% For not indenting the first line of paragraphs:
\setlength{\parindent}{0pt}
% define the title
\author{John Hancock}
\title{MIT Introduction to Statistics 18.05 Problem Set 2 }
\begin{document}
% generates the title
\maketitle
% insert the table of contents
\tableofcontents
\section{References and License}
We are answering questions in the material from MIT OpenCourseWare
course 18.05, Introduction to Probability and Statistics.

Please see the references section for detailed citation information.

The material for the course is licensed under the terms at 
\url{http://ocw.mit.edu/terms}.

We are answering the questions that Orloff and Bloom ask in
\cite{probSet2}.

We use documentation in  to write \LaTeX source code
for this document.

\section{Boy or girl paradox}
The first question Orloff and Bloom pose in \cite{probSet2} is
on the probability of two similar situations.

The first situation is one where we are given that a man has
two children, one of which is a girl.

We are asked to calculate the probability that both children
are girls.

We take the act of giving birth as trial of an 
experiment.

For each trial, we assign the variable $X$ the value 1 if a girl is
born. We assign the variable $X$ the value 0 if a
boy is born.

For each trial the events $X =1$ and $X = 0$ are
equally likely. $X$ fits the definition of a Bernoulli
random variable that Orloff and Bloom give in \cite{reading4}.

Therefore we can write $X \sim \text{Bernoulli} \left( \frac{1}{2} \right)$.

Furthermore, from \cite{reading4} we know $P\left( X = 1 \right) = \frac{1}{2}$.

In this question, we are asked to consider the
probability that two children are girls, that we
can model with two trials of our experiment.

We wish to know, for two trials, the probability
that $X=1$ for both trials.

These trials are independent events. The
probability of two independent events having the same
outcome is the product of the probability
of one event having the outcome.  In this case
the probability $X=1$ is $\frac{1}{2}$, so the
probability of $X=1$ for two trials is 
$\left( \frac{1}{2} \right) \left( \frac{1}{2} \right) = \frac{1}{4}$.


Hence the event of birth of a child 
\printbibliography{}
\end{document}
