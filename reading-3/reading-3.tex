\documentclass[a4paper,11pt]{article}

\usepackage{amsmath}
\usepackage{amssymb}

\usepackage[backend=bibtex]{biblatex}
\bibliography{biblio}

% for probability trees
\usepackage{tikz}
\usetikzlibrary{trees}

%for strikethrough text
\usepackage{soul}

%for R source code listing
\usepackage{listings}

% For not indenting the first line of paragraphs:
\setlength{\parindent}{0pt}
% define the title
\author{John Hancock}
\title{MIT Introduction to Statistics 18.05 Reading 3}
\begin{document}
% generates the title
\maketitle
% insert the table of contents
\tableofcontents
\section{References and License}
We are answering questions in the material from MIT OpenCourseWare
course 18.05, Introduction to Probability and Statistics.

Please see the references section for detailed citation information.

The material for the course is licensed under the terms at 
\url{http://ocw.mit.edu/terms}.

\section{What is $P\left(S_{2} \mid S_{1}^{\complement}\right)$}

$S_{1}$ = 'first card is a spade'
$S_{2}$ = 'second card is a spade'

Therefore $S_{2}^{\complement}$  = 'second card is not a spade

We will calculate

\begin{equation}
  P\left(S_{2} \mid S_{1}^{\complement}\right)
\end{equation}

That is, we will calculate the probability that the first card is a
spade given that the second card is not a spade.

We will apply the same method Orloff and Bloom use in \cite{reading3}, 
section 3, "Multiplication Rule."

If the first card is a spade then of the 51 cards remaining, 
$3 \times 13 = 39$ cards are not spades.

Therefore 
\begin{equation}
  P\left(S_{2} \mid S_{2}^{\complement}\right) = \frac{39}{51}
  \approx 0.765
\end{equation}

We use the multiplication rule from \cite{reading3} 
to compute the same probability.

We apply the multiplcation rule letting $A=S_{1}$ and $B=S_{2}^{\complement}$

\begin{equation}
P\left(S_{1} \mid S_{2}^{\complement}\right) \cdot = 
\frac { 
  P\left(S_{1} \cap S_{2}^{\complement}\right) 
}
{
  P\left(S_{2}^{\complement}\right){}
}
\end{equation}

First we calculate $P\left(S_{1} \cap S_{2}^{\complement}\right)$ .

There are


\printbibliography{}

\end{document}
