\documentclass[a4paper,11pt]{article}

\usepackage{amsmath}
\usepackage{amssymb}

\usepackage[backend=bibtex]{biblatex}
\bibliography{biblio}

% for probability trees
\usepackage{tikz}
\usetikzlibrary{trees}

%for strikethrough text
\usepackage{soul}

%for R source code listing
\usepackage{listings}

% For not indenting the first line of paragraphs:
\setlength{\parindent}{0pt}
% define the title
\author{John Hancock}
\title{MIT Introduction to Statistics 18.05 Reading 3 - Questions }
\begin{document}
% generates the title
\maketitle
% insert the table of contents
\tableofcontents
\section{References and License}
We are answering questions in the material from MIT OpenCourseWare
course 18.05, Introduction to Probability and Statistics.

Please see the references section for detailed citation information.

The material for the course is licensed under the terms at 
\url{http://ocw.mit.edu/terms}.

We are answering the questions in \cite{reading3Questions}.

We use documentation in \cite{latexSymbols}, \cite{latexMultiLine},
\cite{latexBold}, and \cite{latexForLatex} for properly writing the
\LaTeX source code for this document.
 
\section{Problem 1}
You roll two dice. Consider the following events. 

$A$ = 'first die is $3$'

$B$ = 'sum is $7$' 

$C$ = 'sum is greater than or equal to $7$'

\subsection{Compute $P\left(B\right)$; Dice Sum to 7}

We are rolling two dice so the sample space, $\Omega$, is 
$\left\{ \left( x,y \right) 
  \mid x, y \in \left\{ 1,2,3,4,5,6 \right\} \right\}$

Then $B$ is $ \left\{ \left( x, y \right) \in \Omega 
  \mid x + y = 7 \right\}$.

Therefore by inspection $B = \left\{ \left(1, 6 \right), 
  \left(6, 1 \right),
  \left(5, 2 \right),
  \left(2, 5 \right),
  \left(3, 4 \right),
  \left(4, 3 \right)
  \right\}$
  
There are $36$ sequences of integers $\left(x, y \right)$ for
$\left( x, y \right) \in \left\{1, 2, 3, 4, 5, 6 \right\}$.  There are $6$ elements
in $B$, so $P\left( B \right) = \frac{6}{36} \approx 0.1667$.

\subsection{Compute $P \left( B \mid A \right)$}

\begin{equation}  
  P \left( B \mid A \right) =
  \frac {P \left( B \cap A \right) } { P \left( A \right) }
\end{equation}

We defined elements of B in the previous section, and listed them out.

We define $A$:

\begin{equation} \label{BGivenA}
  A = \left\{ \left(x, y \right) \mid x = 3, y \in 
    \left\{ 1, 2, 3, 4, 5, 6 \right\} \right\}
\end{equation}

$A$ has six elements.

$B$ has one element where the first element of the sequence is $3$,
$\left(3, 4 \right)$.

Therefore $\left( A \cap B \right) = \left( 3,4 \right)$.{}

Now, we have all the information we need to calculate 
$P \left( B \mid A \right)$.{}

We Coninue from \ref{BGivenA}:

\begin{equation}  
  \frac {P \left( B \cap A \right) } { P \left( A \right) } =
  \frac {\left| \left( B \cap A \right) \right|}
    {\left| B \right|} = \frac{1}{6} \approx 0.167
\end{equation}

\subsection{Compute $P\left( B \mid C \right)$}

We define $B$ in the previous section.  $C$ is:

\begin{equation}
  C = \left\{ \left( x, y \right) \mid x + y \geq 7, x,y \in{}
    \left\{ 1, 2, 3, 4, 5, 6 \right\} \right\}
\end{equation}

We count the number of elements of $C$ to find a vaule for 
$\left| C \right|$.

\begin{multline}
  C = \left\{ \left( 1, 6 \right),
    \left( 2, 5 \right),
    \left( 2, 6 \right),
    \left( 3, 4 \right),
    \left( 3, 5 \right),
    \left( 3, 6 \right),
    \left( 4, 3 \right),
    \left( 4, 4 \right),
    \left( 4, 5 \right),
    \left( 4, 6 \right)\right\} \\ \cup \left\{ \left( 5, 2 \right),
    \left( 5, 3 \right),
    \left( 5, 4 \right),
    \left( 5, 5 \right),
    \left( 5, 6 \right),
    \left( 6, 1 \right),
    \left( 6, 2 \right),
    \left( 6, 3 \right),
    \left( 6, 4 \right),
    \left( 6, 5 \right),
    \left( 6, 6 \right) \right\}
\end{multline}


We count elements of $C$ listed above to find that $\left| C \right|
= 21$.  However, it behooves us to note that 
\begin{equation}
  \left| C \right| = \sum_{i=1}^{6} i = \frac{6 \times 7} {2} = 
    \frac{42}{2} = 21.
\end{equation}

Therefore:
\begin{equation}
  P\left( B \mid C \right) = 
  \frac{\left| \left( B \cap C \right) \right|} {\left| C \right|}
  = \frac{6}{21} \approx 0.286
\end{equation}
\printbibliography{}
\end{document}
