\documentclass[a4paper,11pt]{article}

\usepackage{amsmath}
\usepackage{amssymb}

\usepackage[backend=bibtex]{biblatex}
\bibliography{biblio}

% for probability trees
\usepackage{tikz}
\usetikzlibrary{trees}

%for strikethrough text
\usepackage{soul}

%for R source code listing
\usepackage{listings}

% For not indenting the first line of paragraphs:
\setlength{\parindent}{0pt}
% define the title
\author{John Hancock}
\title{MIT Introduction to Statistics 18.05 Reading 3 - Questions }
\begin{document}
% generates the title
\maketitle
% insert the table of contents
\tableofcontents
\section{References and License}
We are answering questions in the material from MIT OpenCourseWare
course 18.05, Introduction to Probability and Statistics.

Please see the references section for detailed citation information.

The material for the course is licensed under the terms at 
\url{http://ocw.mit.edu/terms}.

We are answering the questions in \cite{reading3Questions}.

We use documentation in \cite{latexSymbols}, \cite{latexMultiLine},
\cite{latexBold}, and \cite{latexForLatex} for properly writing the
\LaTeX source code for this document.
 
\label{prob1}
\section{Problem 1}
You roll two dice. Consider the following events. 

$A$ = 'first die is $3$'

$B$ = 'sum is $7$' 

$C$ = 'sum is greater than or equal to $7$'

\subsection{Compute $P\left(B\right)$; Dice Sum to 7}

We are rolling two dice so the sample space, $\Omega$, is 
$\left\{ \left( x,y \right) 
  \mid x, y \in \left\{ 1,2,3,4,5,6 \right\} \right\}$

Then $B$ is $ \left\{ \left( x, y \right) \in \Omega 
  \mid x + y = 7 \right\}$.

Therefore by inspection $B = \left\{ \left(1, 6 \right), 
  \left(6, 1 \right),
  \left(5, 2 \right),
  \left(2, 5 \right),
  \left(3, 4 \right),
  \left(4, 3 \right)
  \right\}$
  
There are $36$ sequences of integers $\left(x, y \right)$ for
$\left( x, y \right) \in \left\{1, 2, 3, 4, 5, 6 \right\}$.  There are $6$ elements
in $B$, so $P\left( B \right) = \frac{6}{36} \approx 0.1667$.

\label{defs}
\subsection{Compute $P \left( B \mid A \right)$} 

\begin{equation}  
  P \left( B \mid A \right) =
  \frac {P \left( B \cap A \right) } { P \left( A \right) }
\end{equation}

We defined elements of B in the previous section, and listed them out.

We define $A$:

\begin{equation} \label{BGivenA}
  A = \left\{ \left(x, y \right) \mid x = 3, y \in 
    \left\{ 1, 2, 3, 4, 5, 6 \right\} \right\}
\end{equation}

$A$ has six elements. Since all events are equally likely, 
$P\left( A \right) = \frac{6}{36} = \frac{1}{6}$.

$B$ has one element where the first element of the sequence is $3$,
$\left(3, 4 \right)$.

Therefore $\left( A \cap B \right) = \left( 3,4 \right)$. Since
all outcomes are equally likely, $P\left( A \cap B \right) 
  = \frac{1}{36}$.

Now, we have all the information we need to calculate 
$P \left( B \mid A \right)$.

We Continue from \ref{BGivenA}:

\begin{equation}  
  \frac {P \left( B \cap A \right) } { P \left( A \right) } =
  \frac {\frac{1}{36}} {\frac{1}{6}} = \frac{1}{6} \approx 0.167
\end{equation}

\subsection{Compute $P\left( B \mid C \right)$}

We define $B$ in the previous section.  $C$ is:

\begin{equation}
  C = \left\{ \left( x, y \right) \mid x + y \geq 7, x,y \in{}
    \left\{ 1, 2, 3, 4, 5, 6 \right\} \right\}
\end{equation}

We count the number of elements of $C$ to find a value for 
$\left| C \right|$.

\begin{multline}
  C = \left\{ \left( 1, 6 \right),
    \left( 2, 5 \right),
    \left( 2, 6 \right),
    \left( 3, 4 \right),
    \left( 3, 5 \right),
    \left( 3, 6 \right),
    \left( 4, 3 \right),
    \left( 4, 4 \right),
    \left( 4, 5 \right),
    \left( 4, 6 \right)\right\} \\ \cup \left\{ \left( 5, 2 \right),
    \left( 5, 3 \right),
    \left( 5, 4 \right),
    \left( 5, 5 \right),
    \left( 5, 6 \right),
    \left( 6, 1 \right),
    \left( 6, 2 \right),
    \left( 6, 3 \right),
    \left( 6, 4 \right),
    \left( 6, 5 \right),
    \left( 6, 6 \right) \right\}
\end{multline}


We count elements of $C$ listed above to find that $\left| C \right|
= 21$.  However, it behooves us to note that 
\begin{equation}
  \left| C \right| = \sum_{i=1}^{6} i = \frac{6 \times 7} {2} = 
    \frac{42}{2} = 21.
\end{equation}

Since all outcomes in our current sample space are equally
likely, $P \left( C \right) = \frac{21}{36}$.

Therefore:
\begin{equation}
  P\left( B \mid C \right) = 
  \frac{P\left( B \cap C \right)} {P\left( C \right)}
  = \frac{6}{21} \approx 0.286
\end{equation}

\subsection{$A$ and $B$ Are Independent} \label{bcindependent}
We see in  \ref{defs} $P \left( A \right) = \frac{6}{36} \approx 0.167$

We see in \ref{prob1} that $P \left(B \right) = \frac{6}{36} \approx 0.167$

$A$ and $B$ are independent if, and only if, 

$P\left( A \cap B \right) = P\left( A \right) \cdot P\left(B \right)$

We see in \ref{BGivenA} that $P\left(A \cap B \right)=\frac{1}{36}$

We need to check that 
\begin{equation}
  \frac{1}{36}
  = \left( \frac{6}{36} \right) \left( \frac{6}{36} \right)
\end{equation}

Now, 
\begin{equation}
  \left( \frac{6}{36} \right) \left( \frac{6}{36} \right)
  = \left( \frac{6}{6^{2}} \right) \left( \frac{6}{6^{2}} \right)
  =  \frac{6^{2}}{6^{4}}
  = 6^{2-4} = 6^{-2} = \frac{1}{6^{2}} = \frac{1}{36}
\end{equation}

Therefore $A$ and $B$ are independent. $\blacksquare$

\subsection{$B$ and $C$ are not independent}

We count 21 elements in $C$ in \ref{defs}.  Hence 
$\left| \Omega \right| = 36$.

Therefore $P \left( C \right) = \frac{21}{36} \approx 0.583$.

In \ref{prob1} we calculate $P \left( B \right) = \frac{1}{6}$

Thus, $P\left( B \right) \cdot P \left( C \right) = \frac{21}{216}$

However, $B \cap C$ is the set

\begin{equation}
  B \cap C = \left\{ \left( 3,4 \right), \left( 3, 5 \right), 
    \left( 3, 6 \right) \right\}
\end{equation}

And, $P \left( B \cap C \right) = \frac{3}{36} $.

Since
\begin{equation}
  P \left(B \cap C \right) \neq 
  P \left( B \right) \cdot P \left( C \right)
\end{equation}

$B$ and $C$ are not independent.

\section{Problem 2}

\subsection{$ P\left( S_{2} \mid S_{1}^{\complement} \right) $}

We calculate this in \cite{hancockReading3}.

\section{Problem 3}

\subsection{Probability Second Ball Red} \label{treeSection}

We use a probability tree to calculate this probability.

We define four events:
\begin{itemize}
  \item $R1$ is the event that we draw a red ball for the first drawing, 
  and place two red balls into the urn.

  \item $R2$ is the event that we draw a red ball for the second 
    drawing.

  \item $B1$ is the event that we draw a blue ball for the first 
    drawing.

  \item $B2$ is the event that draw a blue ball for the second drawing.
\end{itemize}

In addition, we define our starting condition $S$ to be that there
are 5 red balls, and 3 blue balls in the urn.

% Set the overall layout of the tree
\tikzstyle{level 1}=[level distance=3.5cm, sibling distance=3.5cm]
\tikzstyle{level 2}=[level distance=3.5cm, sibling distance=2cm]

% Define styles for bags and leafs
\tikzstyle{bag} = [text width=4em, text centered]
\tikzstyle{end} = [circle, minimum width=3pt,fill, inner sep=0pt]

\begin{tikzpicture}[grow=right, sloped]
\node[bag] {$S$}
    child {
        node[bag] {$R1$}        
           child {
                node[end, label=right:
                    {$R2$}] {}
                edge from parent
                node[above]  {$\frac{6}{9}$}
            }
            child {
                node[end, label=right:
                    {$B2$}] {}
                edge from parent
                node[above]  {$\frac{3}{9}$}
            }
            edge from parent 
            node[above]  {$\frac{5}{8}$}
    }
    child {
        node[bag] {$B1$}        
        child {
                node[end, label=right:
                    {$R2$}] {}
                edge from parent
                node[above]  {$\frac{5}{9}$}
            }
            child {
                node[end, label=right:
                    {$B2$}] {}
                edge from parent
                node[above]  {$\frac{4}{9}$}
            }
        edge from parent         
            node[above]  {$\frac{3}{8}$}
    };
\end{tikzpicture}

In this problem, Orloff and Bloom are asking us to calculate 
$P\left( R1 \cap R2 \right) \cup P \left( B1 \cap R2 \right)$,


We calculate this probability as the sum of the products of the
probabilities on the edges of the tree above, where the edges
are parts of paths that end in $R2$.

This sum of products is
\begin{equation}
  P \left( R1 \right) P\left( R2 \right) + 
    P \left( B1 \right) P \left( R2 \right) \\
   = \left( \frac{5}{8} \right)  \left( \frac{6}{9} \right) 
    + \left( \frac{3}{8} \right) \left( \frac{5}{9} \right) \\
  = \frac{45}{72} = \frac{5}{8} = 0.625
\end{equation}

Therefore the probability that the second ball is red is $0.625$.

\subsection{Probability First Ball Blue Given Second Ball Red}

We calculate the probability that the first ball is blue, given that
the second ball is red.

$R1$, $R2$, $B1$, and $B2$ have the same definitions that
we give in the previous section.

We use Bayes' Theorem \cite{reading3} to calculate
$P \left( B1 \mid R2 \right)$.

\begin{equation} \label{baysBR}
  P \left( B1 \mid R2 \right) = 
    \frac{ P \left( R2 \mid B1 \right) P\left( B1 \right) } 
      { P \left( R2 \right) }
\end{equation}

The tree in \ref{treeSection} shows 
$P\left( R2 \mid B1 \right) = \frac{5}{9}$.

We are given $P \left( B1 \right) = \frac{3}{8}$, and
$P \left( R2 \right) = \frac{5}{8}$.

We Substitute these values in the right hand side of the 
equation in \ref{baysBR} to get:

\begin{equation}
  P \left( B1 \mid R2 \right) 
  = \frac { \left( \frac{5}{9} \right) \left( \frac{3}{8} \right) }
      {\frac{5}{8}}
  = \left( \frac{5}{9} \right) \left( \frac{3}{8} \right) 
    \left( \frac{8}{5} \right) 
  = \frac{3}{9} \approx 0.333
\end{equation}


\printbibliography{}
\end{document}
