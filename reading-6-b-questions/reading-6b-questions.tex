\documentclass[a4paper,11pt]{article}

\usepackage{amsmath}
\usepackage{amssymb}

% for proofs  environment
\usepackage{amsthm}

\usepackage[backend=bibtex]{biblatex}
\bibliography{reading-6b-questions}

% for probability trees
\usepackage{tikz}
\usetikzlibrary{trees}

% for Venn diagrams
\usetikzlibrary{shapes,backgrounds}
% for plots
\usepackage{ pgfplots}
% inserted on suggestion in warning during compilation
\pgfplotsset{compat=1.9}

%for strikethrough text
\usepackage{soul}

%for R source code listing
\usepackage{listings}

%for block quotes
\usepackage{csquotes}

% For not indenting the first line of paragraphs:
\setlength{\parindent}{0pt}
% define the title
\author{John Hancock}
\title{MIT Introduction to Statistics 18.05 Reading 6A Think Questions }
\begin{document}
% generates the title
\maketitle
% insert the table of contents
\tableofcontents
\section{References and License}
We are answering questions in the material from MIT OpenCourseWare
course 18.05, Introduction to Probability and Statistics.

In this document we are answering questions Orloff and Bloom ask in
\cite{reading6bQu}.

Please see the references section for detailed citation information.

The material for the course is licensed under the terms at
\url{http://ocw.mit.edu/terms}.

\section{Histogram with a lot of values}
In the first question on reading 6b, Orloff and Bloom ask us if a histogram
for a random variable $X$ that follows some unknown distribution with $100,000$
values should be close to a graph of the probability density function of
whatever distribution $X$ follows.

We do not know anything about the distribution that $X$ follows, so it might
take a lot more than $100,000$ values for the histogram to converge on the
probability distribution function for $X$.

\section{Standardization}

For the second question, Orloff and Bloom give us a the mean and standard
deviation of a random variable $X$.  The mean $\mu$ of $X$ is 2 and the
standard deviation of $X$ is 2.

By definition \cite{reading6b}, the standardization of $X$ is
\begin{equation}
  Z = \frac{X-2}{2}
\end{equation}

\section{Mean}
For the third problem, Orloff and Bloom give us that $Y$ is a standard normal
random variable, and that $W=3Y+4$.  They then ask us for the mean of $W$.

In \cite{reading6a} Orloff and Bloom show that if $X$, and $Y$ are continuous
random variables defined on a sample space $\Omega$, and $a$, and $b$ are
constants, then:
\begin{equation} \label{expCont}
  E\left(aX +b \right)=aE\left(X \right) + b
\end{equation}

$Y$ is a standard normal random variable, so $Y$ has a mean of $0$
\cite{reading6a}.

Now we can substitute the information Orloff and Bloom give us for this problem
into equation \ref{expCont} to conclude that the mean of $W$ is:
\begin{equation}
  E\left( 3Y + 4 \right) = 3E\left(Y \right) + 4 = 3\times0 + 4 = 4.
\end{equation}

\section{Use Central Limit Theorem}
For this last problem, Orloff and Bloom ask us to apply the Central Limit Theorem
to estimate a probability.

Orloff and Bloom ask us to estimate the probability that we toss heads more than
40 times in 64 coin tosses.

We use the same technique Orloff and Bloom use in \cite{reading6b} to estimate
the probability.  Specifically we follow the technique Orloff and Bloom use
in Example 2 and Example 3 of \cite{reading6b}, and replace the numbers they
use with 40 and 64 to get started.

For this problem, $\mu=0.5\times64=32$, and
$\sigma=\sqrt{\text{Var}\left( S \right)}=\sqrt{\frac{64}{4}}=4$.

Let $S$ be the sum of the number of times we toss heads.

Then we wish to estimate $P\left( S > 40 \right)$.

If we apply the Central Limit Theorem like Orloff and Bloom do in
\cite{reading6b}, then we can write
\begin{equation}
  P\left( S > 40 \right) = P\left( \frac{S-32}{4} > \frac{40-32}{4} \right)
    \approx{P \left( Z > 2 \right)}.
\end{equation}

In \cite{reading6bQu}, Orloff and Bloom encourage us to use the rules of thumb
for standard normal probabilities to approximate $P\left(Z > 2 \right)$.

We know from \cite{reading6b}, Example 3, $P\left(Z > 2\right) \approx 0.025$.

Therefore our estimate of the probability of tossing heads 40 times in 64 coin
tosses is 0.025.

\printbibliography{}
\end{document}
