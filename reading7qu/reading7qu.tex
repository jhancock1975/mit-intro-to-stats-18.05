\documentclass[a4paper,11pt]{article}

\usepackage{amsmath}
\usepackage{amssymb}

% for proofs  environment
\usepackage{amsthm}

% for 3d plots
\usepackage{pgfplots}
\usepackage{pgfplotstable}
\usepgfplotslibrary{patchplots}

\usepackage[backend=bibtex]{biblatex}
\bibliography{reading7qu}

% for probability trees
\usepackage{tikz}
\usetikzlibrary{trees}

% for Venn diagrams
\usetikzlibrary{shapes,backgrounds}
% for plots
\usepackage{ pgfplots}
% inserted on suggestion in warning during compilation
\pgfplotsset{compat=1.9}

%for strikethrough text
\usepackage{soul}

%for R source code listing
\usepackage{listings}

%for block quotes
\usepackage{csquotes}

% For not indenting the first line of paragraphs:
\setlength{\parindent}{0pt}
% define the title
\author{John Hancock}
\title{MIT Introduction to Statistics 18.05 Reading 7a Questions}
\begin{document}
% generates the title
\maketitle
% insert the table of contents
\tableofcontents
\section{References and License}
We are answering questions in the material from MIT OpenCourseWare
course 18.05, Introduction to Probability and Statistics.

In this document we are answering questions Orloff and Bloom ask in
\cite{reading7qu}.

Please see the references section for detailed citation information.

The material for the course is licensed under the terms at
\url{http://ocw.mit.edu/terms}.

We use documentation in  to write \LaTeX source code for this
document.

\section{Joint pdf}
The first question Orloff and Bloom ask is what the value of the constant
$c$ is, where $f\left(x, y\right)$ is a pdf.  They give further details
on $f$:
\begin{itemize}
  \item $f$ is defined on $\left[0,1 \right] \times \left[0, 1 \right]$, and
  \item $f \left( x, y \right) = cxy$.
\end{itemize}

In order for $f$ to be a pdf:
\begin{equation}\label{doubleIntF}
  \int_{0}^{1} \int_{0}^{1} cxy \, dy \, dx = 1.
\end{equation}

Equation \ref{doubleIntF} comes from the definition and properties
of a join pdf that Orloff and Bloom state in \cite{reading7}.

We use properties of double integrals from \cite{doubleIntProp}, and
methods of integration in \cite{doubleIntEval} to replace the
integral on the left hand side of \ref{doubleIntF} with its
anti-derivative evaluated over the region
$\left[ 0, 1 \right] \times \left[ 0, 1 \right]$

\begin{equation}\label{antiDerF}
  c \frac{x^2}{2} \bigg\rvert_{x=0}^1 \frac{y^2}{2} \bigg\rvert_{y=0}^1 = 1.
\end{equation}

We evaluate the left hand side of  \ref{antiDerF} to get

\begin{equation}\label{antiDerF}
  c \frac{1}{4}  = 1.
\end{equation}

Therefore $c=4$.


\printbibliography{}
\end{document}
