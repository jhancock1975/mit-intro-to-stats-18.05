\documentclass{article}
\usepackage{makeidx}
\usepackage{amsmath}
\usepackage{amssymb}
\makeindex
\title{Reading 10b Questions}
\author{John Hancock}
\date{July 1, 2017}
\begin{document}
\maketitle \tableofcontents
\section{References and license}
We are answering questions in the material from MIT OpenCourseWare
course 18.05, Introduction to Probability and Statistics.

In this document we are answering questions Orloff and Bloom ask in
\cite{reading10bqu}.

\section{Maximum likelihood estimator}

According to \cite{reading10b} the maximum likelihood estimate for the
end points $a$, and $b$ of the interval for a set of numbers that
are values of a random variable that follows uniform distribution is
the maximum and minimum values in the set.

The data we have shows the uniformly distributed random varaible to
take values, $1, 3, 7$.  Furthermore, Orloff and Bloom tell us that 
the random variable has values from $1$ to $n$ in this problem.

Therefore the maximum likelihood estimate for $n$ is $7$.

\begin{thebibliography}{99}
\bibitem{reading10bqu}
Jeremy Orloff and Jonathan Bloom,
Reading Questions 10b,
Available at https://ocw.mit.edu/courses/mathematics/18-05-introduction-to-probability-and-statistics-spring-2014/readings/reading-questions-10b/ 
(Spring 2014)

\bibitem{reading10b}
Jeremy Orloff and Jonathan Bloom,
Maximum Likelihood Estimates Class 10, 18.05
Jeremy Orloff and Jonathan Bloom,
Available at   https://ocw.mit.edu/courses/mathematics/18-05-introduction-to-probability-and-statistics-spring-2014/readings/MIT18_05S14_Reading7a.pdf
 (Spring 2014)



\end{thebibliography}
\end{document}
