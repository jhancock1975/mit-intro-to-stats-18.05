\documentclass[a4paper,11pt]{article}
\usepackage{amsmath}
\setlength{\parindent}{0pt}
% define the title
\author{John Hancock}
\title{Udacity Intro To Statistics Problem Set 6}
\begin{document}
% generates the title
\maketitle
% insert the table of contents
\tableofcontents

\section{Probability Tables}
Both tables are valid because the probabilities add to 1:

(i)

\begin{center}
  \begin{tabular}{ | c | c | c | c  | c | c |}
    \hline
    outcomes & 1 & 2 & 3 & 4 & 5 \\ \hline
    probability & 1/5 & 1/5 & 1/5 & 1/5 & 1/5 \\ \hline
  \end{tabular}
\end{center}


\begin{equation}
  \frac{1}{5} + \frac{1}{5} + \frac{1}{5} + \frac{1}{5} + \frac{1}{5} = 
    \frac{5}{5} = 1
\end{equation}

(ii)

\begin{center}
  \begin{tabular}{ | c | c | c | c  | c | c |}
    \hline
    outcomes & 1 & 2 & 3 & 4 & 5 \\ \hline
    probability & 1/2 & 1/5 & 1/10 & 1/10 & 1/10 \\ \hline
  \end{tabular}
\end{center}

\begin{equation}
  \frac{5}{10} + \frac{2}{10} + \frac{1}{10} + \frac{1}{10} + \frac{1}{10} = 
    \frac{10}{10} = 1
\end{equation}

\section{Sample Space}
A sample space has exactly two outcomes. The first has probability $p$ 
and the second has probability $3p$. What is the value of $p$?

The sum of the probabilities of the outcomes must equal 1, 
so:

\begin{equation}
p + 3p = 1
\end{equation}

\begin{equation}
\implies 4p = 1
\end{equation}

\begin{equation}
\implies p = \frac{1}{4}
\end{equation}

\section{Union of Probabilities}
Given:
\begin{equation}
P\left(A\right)=0.4,
P\left(B\right)=0.5,
P\left(A \cap B \right)=0.3
\end{equation}

Since the probabilities have an intersection that is not empty we use
the inclusion-exclusion principle to calculate $p\left(A \cup B\right)$:

\begin{equation}
P\left( A \cup B \right) 
= \left( P\left( A \right) + P\left( B \right) \right) - P\left( A \cup B \right)
= (0.4 + 0.5) - 0.3 = 0.9 - 0.3 = 0.6
\end{equation}

\section{Size of an Event}
Our experiment consists of tossing a coin 3 times. Take the sample space
to be all sequences of 3 heads or tails.
\begin{equation}
S = \{HHH, HHT, HTH, HTT, THH, THT, TTH, TTT\}
\end{equation}
For each of the following give the size of the event.

(i) The event 'there are more heads than tails'.
\begin{equation}
\omega = \{HHH, HHT, HTH, THH\}
\end{equation}

\begin{equation}
\implies |\omega| = 4
\end{equation}

(ii) The event 'the first heads occurs after the first tails.'
\begin{equation}
\omega = \{ THH, THT, TTH\}
\end{equation}

\begin{equation}
\implies |\omega| = 3
\end{equation}
\end{document}
