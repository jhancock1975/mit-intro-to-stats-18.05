\documentclass[a4paper,11pt]{article}

\usepackage{amsmath}
\usepackage{amssymb}

% for proofs  environment
\usepackage{amsthm}

\usepackage[backend=bibtex]{biblatex}
\bibliography{reading5bThinkQuestions}

% for probability trees
\usepackage{tikz}
\usetikzlibrary{trees}

% for plots
\usepackage{ pgfplots}
% inserted on suggestion in warning during compilation
\pgfplotsset{compat=1.9}

%for strikethrough text
\usepackage{soul}

%for R source code listing
\usepackage{listings}

%for block quotes
\usepackage{csquotes}

% For not indenting the first line of paragraphs:
\setlength{\parindent}{0pt}
% define the title
\author{John Hancock}
\title{MIT Introduction to Statistics 18.05 Problem Set 2 }
\begin{document}
% generates the title
\maketitle
% insert the table of contents
\tableofcontents
\section{References and License}
We are answering questions in the material from MIT OpenCourseWare
course 18.05, Introduction to Probability and Statistics.

In this document we are answering questions Orloff and Bloom ask in
\cite{reading5b} after writing the word, "Think," in bold face font.

Please see the references section for detailed citation information.

The material for the course is licensed under the terms at
\url{http://ocw.mit.edu/terms}.

We use documentation in \cite{latexIntegrals} to write \LaTeX source code for
this document.

\section{Total area under pdf}

The first question Orloff and Bloom ask in \cite{reading5b} is, ``What is the
total area under the pdf $f \left( x \right)?$''

In \cite{reading5b}, Orloff and Bloom state that it is a property of
probability density functions $f$ that

\begin{equation}
  \int_{-infty}^{infty} {f \left(x \right)} dx = 1
\end{equation}

Therefore, the total area under the pdf $f \left( x \right)$ is 1.

\section{Pdf's textem{vs} probabilities}
In \cite{reading5b}, Orloff and bloom ask, ``: In the previous example f(x)
takes values greater than 1. Why does this not violate the rule that
probabilities are always between 0 and 1?''

This does not violate the rule that probabilities are always between 0 and 1
because the value of a pdf is not a probability.  The value of the integral
of a pdf over an interval is a probability.  Orloff and Bloom state this above
where they ask this question in \cite{reading5b}.

\section{Probabilities of points}
In \cite{reading5b} Orloff and Bloom ask three related questions about CDF's
for the probabilities of intervals of length 0 for continuous random variables.

The value of any integral of a continuous function over a region is the
value of the antiderivative evaluated at the end of the region, minus the value
of the antiderivative at the beginning of the region.  Therefore, if the
beginning of the region is the same as the end of the region, the value of the
integral will be 0.  Since the probablity is defined to be the value of the
integral of the pdf, the probability of a pdf having the value of a specific
point is always 0.

However, this does not mean that the pdf never attains the value that it is
defined on for a point. The value of a function at a point, and the value of
the integral of a function over the interval that begins and ends at the same
point are totally different things.

\printbibliography{}
\end{document}
