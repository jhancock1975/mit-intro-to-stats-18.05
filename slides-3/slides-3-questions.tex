\documentclass[a4paper,11pt]{article}

\usepackage{amsmath}
\usepackage{amssymb}

\usepackage[backend=bibtex]{biblatex}
\bibliography{biblio}

% for probability trees
\usepackage{tikz}
\usetikzlibrary{trees}

%for strikethrough text
\usepackage{soul}

%for R source code listing
\usepackage{listings}

% For not indenting the first line of paragraphs:
\setlength{\parindent}{0pt}

% define the title
\author{John Hancock}
\title{Answers To Questions in Conditional Probability, Independence, 
Bayes Theorem 18.05}

\begin{document}
% generates the title
\maketitle
% insert the table of contents
\tableofcontents
\section{References and License}
The material for the course is licensed under the terms at 
\url{http://ocw.mit.edu/terms}.

In this document, we are answering the questions in \cite{slides3}.

We use documentation in for properly writing the
\LaTeX source code for this document.
 
\label{prob1}
\section{Probability At Least 3 Heads, Given First Toss Tails}

We are tossing a coin four times. Therefore we define the sample space
\begin{equation}
  \Omega = \left\{ \left( x_{1}, x_{2}, x_{3}, x_{4} \right), 
    x_{1}, x_{2}, x_{3}, x_{4} \in \left\{H, T \right\} \right\}
\end{equation}

$ \left| \Omega \right| = 16$

We assume all outcomes are equally likely.

$A$ is the event that at we toss heads at least 3 times.

$B$ is the event that we toss tails the first time.


We use the definition of conditional probability to cacluate
$P \left( A \mid B \right)$.

\begin{equation} \label{defCondProb}
P \left( A \mid B \right) = \frac{ P \left( A \cap B \right) } 
  { P \left( B \right) } \text{, Provided  } P \left( B \right) \neq 0
\end{equation}

$P \left(A \right) = \frac{5}{16}$ since there are 5 elements of $\Omega$
that represent the event that we toss heads at least three times, and
we assume all outcomes are equally likely.

These are:$\left(T,H,H,H \right)$, $\left(H,T,H,H \right)$, 
$\left(H,H,T,H \right)$, $\left(H,H,H,T \right)$, 
$\left(H,H,H,H \right)$.

The elements of $B$ are $ \left(T, T, T, T \right) $,
$ \left(T, T, T, H \right) $, $ \left(T, T, H, T \right) $,
$ \left(T, T, H, H \right) $, $ \left(T, H, T, T \right) $,
$ \left(T, H, T, H \right) $, $ \left(T, H, H, T \right) $,
$ \left(T, H, H, H \right) $.

By inspection $ A \cap B $ is the element $\left( T, H, H, H \right)$.

We substitute values into \ref{defCondProb} to get
\begin{equation}
  P \left( A \mid B \right) = \frac {\frac{1}{16}} {\frac{8}{16}} = 
    \frac{1}{8}
\end{equation}

\section{Probability First Toss Tails, Given At Least 3 Heads}

We use \ref{defCondProb} and definitions of the sets $\Omega$, $A$, 
$A \cap B$, and $B$ that we define in section \ref{prob1}. In addition
we assume all outcomes are equally likely.

We use \ref{defCondProb} to get
\begin{equation}
P \left( B \mid A \right) = \frac{P \left( B \cap A \right)}
  { P \left( A \right) } \text{, Provided } P\left( A \right) \neq 0
\end{equation}

The $\cap$ operator is commutative, so $P \left( A \cap B \right)
 = P \left( B \cap A \right)$, and we discover 
 $P \left( A \cap B \right)$ in section \ref{prob1}.
 
Therefore,

\begin{equation}
P \left( B \mid A \right) = \frac{ \frac{1}{16} }{ \frac{5}{16} }
  = \frac{1}{5}
\end{equation}

\section{Probability Urn}
\subsection{Probability Second Ball Red}

Orloff and Bloom ask the following question in \cite{slides3}, "What is 
the probability the second ball is red?"

We are given the probability tree below:

% Set the overall layout of the tree
\tikzstyle{level 1}=[level distance=3.5cm, sibling distance=3.5cm]
\tikzstyle{level 2}=[level distance=3.5cm, sibling distance=2cm]

% Define styles for bags and leafs
\tikzstyle{bag} = [text width=4em, text centered]
\tikzstyle{end} = [circle, minimum width=3pt,fill, inner sep=0pt]

\begin{tikzpicture}[grow=right, sloped]
\node[end, label=] {}
    child {
        node[bag] {$R_{1}$}        
           child {
                node[end, label=right:
                    {$R_{2} \cap R_{1} $}] {}
                edge from parent
                node[above]  {$ P \left( R_{2} \mid R_{1} \right) = \frac{4}{7}$}
            }
            child {
                node[end, label=right:
                    {$R_{2} \cap G_{2}$}] {}
                edge from parent
                node[above]  {$ P\left( G_{2} \mid R_{1} \right) = \frac{3}{7}$}
            }
            edge from parent 
            node[above]  {$ P \left( R_{1} \right) = \frac{5}{7}$}
    }
    child {
        node[bag] {$G_{1}$}        
        child {
                node[end, label=right:
                    {$R_{2} \cap G_{1}$}] {}
                edge from parent
                node[above]  {$ P \left( R_{2} \mid G_{1} \right)  = \frac{6}{7}$}
            }
            child {
                node[end, label=right:
                    {$G_{2} \cap G_{1}$}] {}
                edge from parent
                node[above]  {$ P \left( G_{2} \mid G_{1} \right) = \frac{1}{7}$}
            }
        edge from parent         
            node[above]  {$ P\left( G_{1} \right) = \frac{2}{7}$}
    };
\end{tikzpicture}


We use the law of total probability to write an equation for 
$P \left( R_{2} \right)$.

\begin{equation} \label{R2Prob}
  P \left( R_{2} \right) = P \left( R_{1} \cap R_{2} \right) +
    P\left( G_{1} \cap R_{2} \right)
\end{equation}

Now we can use the definition of conditional probability to rewrite
\ref{R2Prob}:

\begin{equation} \label{R2ProbApplyDef}
  P \left( R_{2} \right) = P \left( R_{2} \mid R_{1} \right) 
    P \left( R_{1} \right) +
    P\left( R_{2} \mid G_{1} \right) P \left( R_{1} \right)
\end{equation}

Orloff and Bloom give us the probabilities in \ref{R2ProbApplyDef} in 
the probability tree above, so we can use them to compute 
$P\left( R_{2} \right)$.

\begin{equation} \label{R2PlugNumbers}
  P \left( R_{2} \right) = \left( \frac {4}{7} \right) 
    \left( \frac{5}{7} \right) +
    \left( \frac {6}{7} \right) \left( \frac{2}{7} \right)
  = \frac{20}{49} + \frac{12}{49} = \frac{32}{49} \approx 0.653
\end{equation}

\subsection{Probability First Ball Red, Given second Ball Red}

To answer the question, "What is the probability the first ball was red 
given the second ball was red?" \cite{slides3}

This question is asking for $P \left( R_{1} \mid R_{2} \right)$.

Since we know $P\left( R_{2} \mid R_{1} \right)$, we apply Bayes Theorem
to compute $P \left( R_{1} \mid R_{2} \right)$.

\begin{equation} \label{bayesR1GivenR2}
P \left( R_{1} \mid R_{2} \right) = 
  \frac{ P \left( R_{2} \mid R_{1} \right) P \left( R_{1} \right)}
    {P \left( R_{2} \right)}
\end{equation}

Bloom and Orloff give us the values for all of the probabilities in the
right hand side of \ref{bayesR1GivenR2} in the probability tree above.

Therefore 

\begin{equation} \label{bayesR1GivenR2}
P \left( R_{1} \mid R_{2} \right) = 
  \frac{ \left( \frac{4}{7} \right) \left( \frac{5}{7} \right)} 
    {\frac{32}{49}} 
  = \left( \frac{20}{49} \right) 
      \left( \frac{49}{32} \right)
  = \frac{20}{32} \approx 0.625
\end{equation}

\section{Concept Questions on Probability Trees}
In this section, we answer questions about the probability tree that
Orloff and Bloom give in \cite{slides3}.

\begin{tikzpicture}[grow=down, sloped]
\node[end, label=] {}
    child {
        node[bag] {$A_{1}$}
        child {
          node[bag] {$B_{1}$}
          child {
            node[end, label=below: {$C_{1}$}]{}
            edge from parent node[above] {}
          }
          child {
            node[end, label=right: {$C_{2}$}]{}
            edge from parent node[above]  {}
          }     
          edge from parent node[above] {}
        }
        child {
          node[bag] {$B_{2}$}
          child {
            node[end, label = left: {$C_{1}$}]{}
            edge from parent node[above] {z}
          }
          child {
            node[end, label = below: {$C_{2}$}]{}
            edge from parent node[above]  {}
          }
          edge from parent node[above]  {y}
        }
        edge from parent node[above] {x}
    }
    child {
        node[bag] {$A_{2}$}
        child {
          node[bag] {$B_{1}$}
          child {
            node[end, label=below: {$C_{1}$}]{}
            edge from parent node[above] {}
          }
          child {
            node[end, label=right: {$C_{2}$}]{}
            edge from parent node[above]  {}
          }
          edge from parent node[above] {}
        }
        child {
          node[bag] {$B_{2}$}
          child {
            node[end, label = left: {$C_{1}$}]{}
            edge from parent node[above] {}
          }
          child {
            node[end, label = below: {$C_{2}$}]{}
            edge from parent node[above]  {}
          }
          edge from parent node[above]  {}
        }
        edge from parent node[above]  {}
    };
\end{tikzpicture}

The edge labeled, "x," in the figure above represents 
$P \left( A_{1} \right)$.

The edge labeled, "y," in the figure above represets
$P \left( B_{2} \mid A_{1} \right)$.

The edge labeled, "z," in the figure above represents 
$P \left( C_{1} \mid A_{1} \cap B_{2} \right)$.  We read section 5.1,
titled, "Shorthand vs. precise trees," in \cite{slides3} in order to 
understand what the edge labeled, "z," represents.  This section 
explains that nodes having a distance of two or more edges to the
root of the tree represent the probabilities of intersections of sets
of outcomes.

\printbibliography{}
\end{document}
