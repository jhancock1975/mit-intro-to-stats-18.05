\documentclass[a4paper,11pt]{article}

\usepackage{amsmath}
\usepackage{amssymb}

\usepackage[backend=bibtex]{biblatex}
\bibliography{biblio}

% for probability trees
\usepackage{tikz}
\usetikzlibrary{trees}

%for strikethrough text
\usepackage{soul}

%for R source code listing
\usepackage{listings}

% For not indenting the first line of paragraphs:
\setlength{\parindent}{0pt}

% define the title
\author{John Hancock}
\title{Answers To Questions in Conditional Probability, Independence, 
Bayes Theorem 18.05}

\begin{document}
% generates the title
\maketitle
% insert the table of contents
\tableofcontents
\section{References and License}
The material for the course is licensed under the terms at 
\url{http://ocw.mit.edu/terms}.

In this document, we are answering the questions in \cite{slides3}.

We use documentation in for properly writing the
\LaTeX source code for this document.
 
\label{prob1}
\section{Probability At Least 3 Heads, Given First Toss Tails}

We are tossing a coin four times. Therefore we define the sample space
\begin{equation}
  \Omega = \left\{ \left( x_{1}, x_{2}, x_{3}, x_{4} \right), 
    x_{1}, x_{2}, x_{3}, x_{4} \in \left\{H, T \right\} \right\}
\end{equation}

$ \left| \Omega \right| = 16$

We assume all outcomes are equally likely.

$A$ is the event that at we toss heads at least 3 times.

$B$ is the event that we toss tails the first time.


We use the definition of conditional probability to cacluate
$P \left( A \mid B \right)$.

\begin{equation} \label{defCondProb}
P \left( A \mid B \right) = frac{ P \left( A cap B \right) } 
  { P \left( B \right) }, Provided P \left( B \right) \neq 0
\end{equation}

$P\left(A \right) = frac{5}{16}$ since there are 5 elements of $\Omega$
that represent the event that we toss heads at least three times, and
we assume all outcomes are equally likely.

These are:$\left(T,H,H,H \right)$, $\left(H,T,H,H \right)$, 
$\left(H,H,T,H \right)$, $\left(H,H,H,T \right)$, 
$\left(H,H,H,H \right)$.

The elements of $B$ are $ \left(T, T, T, T \right) $,
$ \left(T, T, T, H \right) $, $ \left(T, T, H, T \right) $,
$ \left(T, T, H, H \right) $, $ \left(T, H, T, T \right) $,
$ \left(T, H, T, H \right) $, $ \left(T, H, H, T \right) $,
$ \left(T, H, H, H \right) $.

By inspection $ A \cap B $ is the element $\left( T, H, H, H \right)$.

We substitute values into \ref{defCondProb} to get
\begin{equation}
  P \left( A \mid B \right) = \frac {\frac{1}{16}} {\frac{8}{16}} = 
    frac{1}{8}
\end{equation}

\section{Probability First Toss Tails, Given At Least 3 Heads}

We use \ref{defCondProb} and definitions of the sets $\Omega$, $A$, 
$A \cap B$, and $B$ that we define in section \ref{prob1}. In addition
we assume all outcomes are equally likely.

We use \ref{defCond} to get
\begin{equation}
P \left( B \mid A \right) = frac{P \left( B \cap A \right)}
  { P \left( A \right) }, Provided P\left( A \right) \neq 0
\end{equation}

The $\cap$ operator is commutative, so $P \left( A \cap B \right)
 = P \left( B \cap A \right)$, and we discover 
 $P \left( A \cap B \right)$ in the previous section \ref{prob1}.
 
Therefore,

\begin{equation}
P \left( B \mid A \right) = frac{ \frac{1}{16} }{ \frac{5}{16} }
  = \frac{1}{5}
\end{equation}

\section{Probability Second Ball Red}
 
\printbibliography{}
\end{document}
