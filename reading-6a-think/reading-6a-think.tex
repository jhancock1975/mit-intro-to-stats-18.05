\documentclass[a4paper,11pt]{article}

\usepackage{amsmath}
\usepackage{amssymb}

% for proofs  environment
\usepackage{amsthm}

\usepackage[backend=bibtex]{biblatex}
\bibliography{reading-6a-think}

% for probability trees
\usepackage{tikz}
\usetikzlibrary{trees}

% for Venn diagrams
\usetikzlibrary{shapes,backgrounds}
% for plots
\usepackage{ pgfplots}
% inserted on suggestion in warning during compilation
\pgfplotsset{compat=1.9}

%for strikethrough text
\usepackage{soul}

%for R source code listing
\usepackage{listings}

%for block quotes
\usepackage{csquotes}

% For not indenting the first line of paragraphs:
\setlength{\parindent}{0pt}
% define the title
\author{John Hancock}
\title{MIT Introduction to Statistics 18.05 Reading 6A Think Questions }
\begin{document}
% generates the title
\maketitle
% insert the table of contents
\tableofcontents
\section{References and License}
We are answering questions in the material from MIT OpenCourseWare
course 18.05, Introduction to Probability and Statistics.

In this document we are answering questions Orloff and Bloom ask in
\cite{reading6a}.

Please see the references section for detailed citation information.

The material for the course is licensed under the terms at
\url{http://ocw.mit.edu/terms}.

\section{Why the median is not the mean}

In \cite{reading6a} Orloff and Bloom give us the exponential distribution,
and show that the median value of the exponential distribution
$\text{exp}\left(\lambda \right)$ is
$\frac{ \text{ln} \left(2 \right)} {\lambda}$

On the other hand, they also show that the mean value of
$\text{exp}\left(\lambda \right)$ is $\frac{1} {\lambda}$.

Orloff and Bloom then ask us why the mean value is not equal to the median
value.

They give us a hint to look at a graph of the pdf of
$\text{exp}\left(\lambda \right)$, and they mention that the median is to
the left of the mean.

Here is a graph of $\text{exp}\left(\lambda \right)$:

\begin{tikzpicture}
      \draw[scale=1,domain=0:3.5,smooth,variable=\x,black] plot ({\x}, { exp(-1*\x)});
      \draw [dotted, gray] (0,0) grid (3.5,2);
\end{tikzpicture}

If $X \sim \text{exp}\left(\lambda \right)$ the median value of $X$ is the
point on the $x$ axis of the graph above where half the area under the curve
is to the left of $x$.

However, if we think about the mean value of $X$ as a sum of an infinite
number of terms, there will be more terms in the sum where values of $x$
are to the right of the median value.  Therefore the median value is
to the left of the mean in this case.

\printbibliography{}
\end{document}
