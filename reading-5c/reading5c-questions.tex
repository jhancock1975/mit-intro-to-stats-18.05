\documentclass[a4paper,11pt]{article}

\usepackage{amsmath}
\usepackage{amssymb}

% for proofs  environment
\usepackage{amsthm}

\usepackage[backend=bibtex]{biblatex}
\bibliography{reading5c-questions}

% for probability trees
\usepackage{tikz}
\usetikzlibrary{trees}

% for plots
\usepackage{ pgfplots}
% inserted on suggestion in warning during compilation
\pgfplotsset{compat=1.9}

%for strikethrough text
\usepackage{soul}

%for R source code listing
\usepackage{listings}

%for block quotes
\usepackage{csquotes}

% For not indenting the first line of paragraphs:
\setlength{\parindent}{0pt}
% define the title
\author{John Hancock}
\title{MIT introduction to statistics 18.05  reading 5c questions}
\begin{document}
% generates the title
\maketitle
% insert the table of contents
\tableofcontents
\section{References and License}
We are answering questions in the material from MIT OpenCourseWare
course 18.05, Introduction to Probability and Statistics.

In this document we are answering questions Orloff and Bloom ask in
\cite{reading5CQ}. We rely on the material in \cite{reading5c} to answer
these questions.

Please see the references section for detailed citation information.

The material for the course is licensed under the terms at
\url{http://ocw.mit.edu/terms}.

\section{Use r pnorm function to calculate probability}
Orloff and Bloom ask us to use the pnorm function of the R programming language
to calculate $P\left( Z < 1.5 \right)$.

The on-line documentation for the pnorm function we accessed with
\begin{lstlisting}
help(pnorm)
\end{lstlisting}
states that $\text{pnorm}\left(x, \mu, \sigma, lower.tail \right)$
returns $P \left( Z \leq x \right)$ if lower.tail has the value TRUE, and
$P \left( Z > x \right)$ otherwise.

Orloff and bloom are asking us to calculate
$P \left( Z < 1.5 \right)$, but strictly speaking, we cannot use the pnorm
function to calculate this value because pnorm returns either
$P \left(  Z \leq x \right)$ or $P \left( Z > x \right)$.

However, we call the pnorm function and get the output listed below:
\begin{lstlisting}
	> pnorm(1.5, 0, 1)
	[1] 0.9331928
\end{lstlisting}

When we enter this value into the checker for this problem, the checker reports
that this is the correct value.  This begs the question: for a random variable
Z that follows the normal distribution, is
\begin{equation}
	P \left( Z < x \right) = P \left(Z \leq x \right)?
\end{equation}

For the second reading question, Orloff and Bloom ask us to calculate
$P \left( -1 < Z < 1 \right)$.

Since pnorm returns $P \left( Z \leq x \right)$, we will use the fact that the
area under the curve over the interval $ \left( -1, 1 \right)$ is equal
to the area under the curve from $\left(-\infty, 1 \right)$ minus the area
under the curve from $\left( -\infty, -1 \right)$.

Therefore, we invoke the pnorm function twice to calculate the probability
$P \left( -1 \leq Z \leq 1 \right)$.

\begin{lstlisting}
> pnorm(1, 0,  1) - pnorm(-1, 0, 1)
[1] 0.6826895
\end{lstlisting}

Once again, when we enter the result above into the checker for this question,
the checker indicates that this is the correct answer.  However, this is
$P \left( -1 \leq Z \leq 1 \right)$, not
$P \left( -1 < Z < 1 \right)$ that Orloff and Bloom are asking for.
\printbibliography{}
\end{document}
